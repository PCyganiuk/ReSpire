\chapter{Projekt rozwiązania}

\section{Projekt architektury systemu \Hania\ \Ola}

\section{Projekt logiki aplikacji \Hania\ \Ola}

\section{Projekt interfejsu użytkownika (figma) \Hania\ \Ola}
Wstępny projekt interfejsu użytkownika został zaprojektowany w Figmie.

W narzędziu Canva zostało zaprojektowane logo aplikacji - połączenie dwóch różnych czcionek oraz dodatek grafiki podmuchu i detalu symbolizującego cząsteczkę powietrza nad literą “i” oddaje ducha ReSpire.

W oparciu o kolorystykę elementów w logo zostały wybrane 3 główne kolory aplikacji - są to odcienie niebieskiego. Kolor ten jest kojarzony z poczuciem spokoju, harmonii, stabilizacji i bezpieczeństwa.( tu cytat). Dzięki temu aplikacja jest spójna kolorystycznie, a także wywołuje przyjemne odczucia w użytkowniku, zapewniając spokój i harmonię. 

Jedna z czcionek \todoJakub{Chyba czcionka użyta? Użyliśmy kilka?} użytych w logo  - “Glacial Indifference” została również wykorzystana w UI aplikacji, zapewniając jej unikatowy charakter.

Białe kafelki na stronie głównej z asymetrycznymi zaokrągleniami oraz symbolami podmuchu stanowią prosty, ale modernistyczny element aplikacji. Przycisk dodawania własnego treningu nawiązuje do detalu w logo, czyniąc aplikację wyjątkową.

\section{Projekt struktury danych (treningi) \Jakub}