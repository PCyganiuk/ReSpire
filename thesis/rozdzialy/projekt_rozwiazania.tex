\chapter{Projekt rozwiązania}

\section{Projekt architektury systemu \Hania\ \Ola}

\section{Projekt logiki aplikacji \Hania\ \Ola}

\section{Projekt interfejsu użytkownika \Hania\ \Ola}
\subsection{Projekt w Figmie}
Wstępny projekt interfejsu użytkownika został zaprojektowany w Figmie.

\begin{figure}[H]
\centering
\includegraphics[scale=0.4]{obrazki/projekt_rozwiazania/figma_home.png}
\caption{Pierwszy projekt aplikacji ReSpire - strona główna}
\label{img/respire_home}
\end{figure}

\begin{figure}[H]
\centering
\includegraphics[scale=0.45]{obrazki/projekt_rozwiazania/figma_training.png}
\caption{Pierwszy projekt aplikacji ReSpire - przebieg treningu}
\label{img/respire_training}
\end{figure}

\begin{figure}[H]
\centering
\includegraphics[scale=0.44]{obrazki/projekt_rozwiazania/figma_settings.png}
\caption{Pierwszy projekt aplikacji ReSpire - edycja treningu}
\label{img/respire_settings}
\end{figure}

\newpage
\subsection{Identyfikacja wizualna aplikacji}
W celu zapewnienia spójności wyglądu oraz estetyki aplikacji ReSpire przygotowana została identyfikacja wizualna obejmująca logo, kolory oraz czcionkę.

Logo aplikacji zostało zaprojektowane w narzędziu \textit{Canva}, a część jego elementów była narysowana w edytorze graficznym \textit{AdobeFresco}. Połączenie dwóch różnych czcionek o odmiennych grubościach oraz dodatek grafiki podmuchu i detalu symbolizującego cząsteczkę powietrza nad literą \textit{i} oddaje ducha ReSpire. Logo używane w ekranie ładowania aplikacji oraz ikonie posiada napis umieszczony w dwóch liniach, jak przedstawiono na rysunku \ref{img/respire_logo}. Pomysł umieszczenia go na pasku na ekranie głównym spowodował, że powstała także druga wersja, w której całość mieści się w jednej linii, co pokazano na rysunku \ref{img/logo_w_jednej_linii}.
\begin{figure}[H]
    \centering
    \begin{minipage}{0.48\textwidth}
        \centering
        \includegraphics[scale=0.45]{obrazki/projekt_rozwiazania/logo_biale_tlo.png}
        \caption{Logo ReSpire w oryginalnej wersji}
        \label{img/respire_logo}
    \end{minipage}
    \hfill
    \begin{minipage}{0.48\textwidth}
        \centering
        \includegraphics[scale=0.45]{obrazki/projekt_rozwiazania/logo_poziom.png}
        \caption{Logo przystosowane do użycia na ekranie głównym aplikacji}
        \label{img/logo_w_jednej_linii}
    \end{minipage}
\end{figure}


W oparciu o kolorystykę elementów w logo zostały wybrane 3 główne kolory aplikacji. Są to~odcienie niebieskiego. Kolor ten jest kojarzony z poczuciem spokoju i bezpieczeństwa \cite{blue}. Dzięki temu aplikacja jest spójna kolorystycznie, a także wywołuje przyjemne odczucia w użytkowniku, zapewniając wyciszenie i harmonię. 
Poniżej przedstawiono listę kolorów zapisanych w~dwóch powszechnie stosowanych systemach zapisu barw oraz ich wizualizację:
\definecolor{colA}{HTML}{1A93A8}
\definecolor{colB}{HTML}{32B7CF}
\definecolor{colC}{HTML}{7BDEF0}

\noindent
\colorbox{colA}{\phantom{XXXX}} \quad \textbf{HEX:} \#1A93A8, \textbf{RGB:} (26, 147, 168) \\[3pt]
\colorbox{colB}{\phantom{XXXX}} \quad \textbf{HEX:} \#32B7CF, \textbf{RGB:} (50, 183, 207) \\[3pt]
\colorbox{colC}{\phantom{XXXX}} \quad \textbf{HEX:} \#7BDEF0, \textbf{RGB:} (123, 22, 240)

Jedna z dwóch czcionek użytych w logo - \textit{Glacial Indifference} \cite{font} - została również wykorzystana w~aplikacji, zapewniając jej unikatowy charakter. Jest to darmowy, otwartoźródłowy krój pisma zaprojektowany przez \textit{Hanken Design Co.} z~licencją \textit{SIL OPEN FONT LICENSE Version 1.1} \cite{font_license} pozwalającą na użytek zarówno prywatny, jak i komercyjny. Przykładowy tekst oraz znaki zapisane użytą czcionką przedstawione zostały na rysunku \ref{img/czcionki}. Wykorzystana została także domyślna czcionka \textit{Fluttera} dla aplikacji na systemie \textit{Android} - \textit{Roboto}.
\begin{figure}[H]
    \centering
    \includegraphics[scale=0.25]{obrazki/projekt_rozwiazania/czcionka_pogrubiona.jpg}
    \includegraphics[scale=0.25]{obrazki/projekt_rozwiazania/czcionka_regularna_tekst.jpg}
    \caption{Czcionka \textit{GlacialIndeifference} użyta w logo oraz aplikacji - po lewej w wersji pogrubionej, a~po~prawej w standardowej}
    \label{img/czcionki}
\end{figure}

Elementy charakterystyczne dla wyglądu ReSpire przedstawione zostały na rysunku \ref{img/kafelek_i_przycisk}. Białe kafelki z nazwami treningów na stronie głównej z asymetrycznymi zaokrągleniami oraz symbolami podmuchu stanowią prosty, ale modernistyczny element aplikacji. Przycisk dodawania własnego treningu nawiązuje do detalu cząstki powietrza w logo, czyniąc aplikację wyjątkową. 
\begin{figure}[H]
    \centering
    \includegraphics[scale=0.25]{obrazki/projekt_rozwiazania/Kafelek_i_przycisk.png}
    \caption{Kafelek z nazwą treningu oraz przycisk dodawania nowego poniżej}
    \label{img/kafelek_i_przycisk}
\end{figure}


Wszytskie ikony używane w aplikacji pochodzą z otwartoźródłowej biblioteki \textit{Material Icons} \cite{icons} należącej do \textit{Google}. Są intuicyjne i minimalistyczne, dzięki czemu wygląd aplikacji jest spójny oraz przejrzysty dla użytkownika. Można je wykorzystywać na podstawie licencji \textit{Apache License, Version 2.0} \cite{font_license}.
\begin{figure}[H]
    \centering
    \includegraphics[scale=0.35]{obrazki/projekt_rozwiazania/ikony.png}
    \caption{Przykładowe ikony z biblioteki \textit{Material Icons}}
    \label{img/ikony}
\end{figure}

W aplikacji zastosowano lekkie, łatwo skalowalne animacje w formacie Lottie \cite{animations}. Umożliwia on zapis w formie plików JSON, są bazowane na grafice wektorowej i idealnie nadaje się do aplikacji mobilnych, zapewniając wysoką płynność i niskie wykorzystanie zasobów. Wykorzystane zostały dwie darmowe animacje - fala \cite{wave_animation} wykonana przez \textit{CosmoYo} oraz łódka \cite{boat_animation}, której autorem jest \textit{Sergey Riznyk} na podstawie licencji \textit{Lottie Simple License (FL 9.13.21)} \cite{animations_license}. Zostały one dostosowane do potrzeb aplikacji poprzez między innymi: zmianę kolorów, usunięcie niektórych elementów, spowolnienie tempa odtwarzania czy zmianę orientacji. Animacje ożywiają warstwę prezentacji, dopełniając przestrzenie ekranu głównego i strony ze szczegółami treningu. Dzięki powolnemu tempu wprowadzają spokój.


\section{Projekt struktury danych (treningi) \Jakub}
\begin{figure}[H]
    \centering
    \includegraphics[width=\textwidth]{obrazki/projekt_rozwiazania/training_class_diagram.png}
    \caption{Diagram klas dla treningu oddechowego}
    \label{img/class_diagram}
\end{figure}