\chapter{Krótka instrukcja obsługi}

\begin{flushright}
  \small{
        \parbox{8cm}{
                {\small
                        \textit{Do what you think is interesting,
                        do something that you think is fun and worthwhile,
                        because otherwise you won't do it well anyway.}
                }
                \vspace{.5cm}\hfill{---Brian W. Kernighan}
        }
 }
\end{flushright}

Pakiet \verb|SzablonPG| powstał, aby ułatwić pisanie prac dyplomowych na Politechnice Gdańskiej. Ułatwienie to ma polegać na podaniu
szablonu \LaTeX'owego pracy dyplomowej, który będzie maksymalnie zgodny z wytycznymi dla autorów
podanych w Zarządzeniu Rektora 22/2018 z 20 czerwca 2018 roku. W ten sposób zamiast ,,walczyć'' z \LaTeX'em lub starać się
złożyć pracę w MS Word można skupić się na treści. Korzystanie z \LaTeX'a szczególnie w przypadku prac
matematycznych wydaje się być wskazane z uwagi na łatwość składania i jakość złożonych formuł
matematycznych.


\section{Jak użyć szablonu}

Szablon dostarcza klasy \verb|SzablonPG|, którą należy użyć zamiast standardowych klas np.:
\verb|article|, \verb|report|. Plik \verb|szablonPG.cls| \textbf{musi być} umieszczony w tym samym
katalogu, co główny plik pracy. Na początku głównego pliku pracy piszemy
\begin{verbatim}
    \documentclass{szablonPG}   % zamiast standardowych klas np.: article
\end{verbatim}
Plik \verb|SzablonPracyPG_main.tex| pokazuje jak może wyglądać praca złożona z użyciem tej klasy.

\section{Pisanie w \LaTeX'u}

Aby wygodnie pisać w \LaTeX'u potrzebujemy:
\begin{enumerate}
 \item dystrybucji \TeX'a np.: MikTeX lub TeXLive (Linux)
 \item edytora np.: LED, WinEdit, Kile (Linux)
\end{enumerate}

Podstawowym źródłem wiedzy o programowaniu jest dostarczana z każdą dystrybucją dokumentacja:
\begin{enumerate}
 \item A Gentle Introduction to TEX,
 \item The Not So Short Introduction to LATEX
\end{enumerate}

Internet jest pełen wprowadzeń, tutoriali i przykładów. Nieocenionym źródłem wiedzy są strony:
\begin{verbatim}
http://google.com/
http://faq.gust.org.pl/
http://tex.stackexchange.com/
\end{verbatim}

\section{Organizacja projektu}

Pisanie w jednym pliku dużego projektu, powoduje, że w pewnym momencie ciężko odszukać wśród setek
linii kodu tą właściwą. System \LaTeX umożliwia podzielenie kodu źródłowego na osobne pliki, służą
do tego polecenia \verb|\input| oraz \verb|\include|. Opiszemy sposób korzystania z tego drugiego
polecenia.

Na początek dzielimy naszą pracę na logiczne fragmenty. W przypadku pracy dyplomowej mogą to być
np.: rozdziały. Następnie tworzymy osobne pliki \TeX'owe dla każdego rozdziału i~jeden plik główny.

W pliku głównym umieszczamy wszystkie deklaracje, dołączamy pakiety, których chcemy użyć i sekwencję
\verb|\begin{document} ... \end{document}|. W tej sekwencji za pomocą polecenia \verb|\include|
dołączamy pliki rozdziałów. W plikach rozdziałów piszemy \textbf{tylko kod rozdziału}.

Załóżmy, że nasza praca składa się ze \textit{Wstępu, Rozdziału 1 i Rozdziału 2}. Tworzymy wtedy
pliki (nazwy plików przykładowe): \verb|praca_glowny.tex|, \verb|praca_wstep.tex|,
\verb|praca_rozdzial1.tex|, \verb|praca_rozdzial2.tex|. UWAGA! \LaTeX\ nie lubi spacji w nazwach
plików. Przykładowo, pliki mogłyby wyglądać tak:
\begin{multicols}{2}
\begin{verbatim}
% praca_glowny.tex
\documentclass{SzablonPracyPG}
%inne pakiety, definicje itp.
\begin{document}
\tableofcontents
  \include{praca_wstep}
  \include{praca_rozdzial1}
  \include{praca_rozdzial2}
% bibilografia etc.
\end{document}
%EOF
\end{verbatim}
\columnbreak
\begin{verbatim}
% praca_wstep.tex
\chapter*{Wstęp}

To jest wstęp.
\end{verbatim}
\begin{verbatim}
% praca_rozdzial1.tex
\chapter{Tytuł pierwszego rozdziału}

Treśc pierwszego rozdziału.
\end{verbatim}
\begin{verbatim}
% praca_rozdzial2.tex
\chapter{Tytuł drugie rozdziału}

Treść drugiego rozdziału.
\end{verbatim}
\end{multicols}

Pliki nie muszą być umieszczone w tym samy katalogu co plik główny, trzeba wtedy poprzedzić
nazwę pliku ścieżką do katalogu w którym znajduje się plik. Przykładowo można w tym celu użyć następującego polecenia: \verb|\input{rozdzialy/praca_rozdzial1.tex|. Wszystkie edytory \TeX'owe wspierają
tworzenie projektów, czyli prace w ten sposób, pomagając w ten sposób zapanować nad strukturą pracy.
Dla długich projektów przydatne jest także polecenie \verb|\includeonly|. 

\section{Strona tytułowa i oświadczenie}


Najprościej - umieszczamy w katalogu \textbf{meta} pliki oświadczenia oraz strony tytułowej wygenerowane z serwisu MojaPG, zapisane po uzupełnieniu danych w formacie pdf. Następnie w głównym pliku pracy dyplomowej dodajemy pliki do całego dokumentu przy użyciu poleceń:
\begin{verbatim}
%------------------------------------------------------------
%  Dodanie strony tytułowej wygenerowanej z MojaPG oraz 
%  						oświadczenia
\includepdf{meta/strona_tytulowa.pdf}
\includepdf{meta/oswiadczenie.pdf}
%------------------------------------------------------------
\end{verbatim}

Polecenia te znajdują się już w przykładowym pliku głównym pracy \verb|SzablonPracyPG_main.tex|.

\section{Język polski i kodowanie}

W przypadku, gdy:
\begin{enumerate}
 \item plik \verb|SzablonPracyPG_main.tex| nie chce się kompilować,
 \item kod źródłowy tego przykładu zawiera ,,krzaczki'',
 \item widoczny po skompilowaniu tekst: nie zawiera polskich liter,
 \item zamiast polskich liter wyświetla ,,krzaczki'',
\end{enumerate}
to jedną z przyczyn może być niewłaściwe kodowanie polskich znaków. Domyślnie przyjmujemy kodowanie
windowsowe: \verb|cp1250|. Zmiana jest możliwa
poprzez ręczną edycje linii
\begin{center}
\verb|\RequirePackage[cp1250]{inputenc}|
\end{center}
Możliwe wartości to:
\begin{enumerate}
 \item \verb|\RequirePackage[utf8]{inputenc}| (dla kodowania utf8)
 \item \verb|\RequirePackage[latin2]{inputenc}| (dla kodowania iso8859-2)
\end{enumerate}
Inną możliwością jest zmiana kodowania pliku w używanym edytorze.

\subsection{Kilka słów wyjaśnienia}

Jedną z wielkich bolączek typografii komputerowej jest/był brak jednolitego formatu kodowania
polskich znaków (lub ogólniej znaków innych niż występujących w angielskich tekstach). We wczesnych
latach pojawiło się \textbf{bardzo dużo} niezgodnych ze sobą systemów. Pewną standaryzacje
wprowadziły DOS i Windows (strony kodowe cpXXXX) i upowszechnienie WWW (warianty isoXXXX-X).
Obecnie coraz powszechniejsze jest kodowanie w standardzie utfX.

\section{Matematyka}

\subsection{Środowiska twierdzeń}
W klasie zdefiniowane są następujące środowiska matematyczne;

\begin{itemize}
  \item Definicja
\begin{definicja}
\label{def:pierwsza}
Liczbę naturalną, nazywamy liczbą pierwszą, jeżeli jej jedynymi dzielnikami jest $1$ i ona sama.
\end{definicja}
\begin{verbatim}
\begin{definicja}
Liczbę naturalną, nazywamy liczbą pierwszą, jeżeli jej jedynymi dzielnikami
jest $1$ i ona sama.
\end{definicja}
\end{verbatim}

  \item Lemat
\begin{lemat}
Niech $p$ będzie liczbą pierwszą. Jeżeli $p|ab$, to $p|a$ lub $p|b$.
\end{lemat}
\begin{verbatim}
\begin{lemat}
Niech $p$ będzie liczbą pierwszą. Jeżeli $p|ab$, to $p|a$ lub $p|b$.
\end{lemat}
\end{verbatim}

  \item Twierdzenie
\begin{twierdzenie}
\label{tw:ilepierwszych}
Liczb pierwszych jest nieskończenie wiele.
\end{twierdzenie}

\begin{verbatim}
\begin{twierdzenie}
Liczb pierwszych jest nieskończenie wiele.
\end{twierdzenie}
\end{verbatim}

  \item Wniosek
\begin{wniosek}
Ciąg liczb pierwszych jest nieograniczony.
\end{wniosek}

\begin{verbatim}
\begin{wniosek}
Ciąg liczb pierwszych jest nieograniczony.
\end{wniosek}
\end{verbatim}

\item Przykład
\begin{przyklad}

\end{przyklad}

\end{itemize}


Prowadzona numeracja środowisk jest jest ciągła, aby ułatwić odnajdywanie numerów w tekście.
Wszystkie środowiska złożone są w ten sam sposób, modyfikacja jest możliwa poprzez utworzenie
własnego stylu i zadeklarowanie jego użycia przed zdefiniowaniem środowiska (patrz.
dokumentacja pakietów \verb|ams*| i plik klasy).

Do twierdzeń, definicji itd. możemy odwoływać się korzystając z konstrukcji \verb|\label| --
\verb|\ref|, np.; Definicja \ref{def:pierwsza} jest bardzo ważna. Twierdzenie
\ref{tw:ilepierwszych} było znane już Euklidesowi.
\begin{verbatim}
Definicja \ref{def:pierwsza} jest bardzo ważna.
Twierdzenie \ref{tw:ilepierwszych} było znane juz Euklidesowi.
\end{verbatim}


\section{Wyliczenia}

Wyliczenia numerowane tworzymy za pomocą następującego polecenia:
\begin{verbatim}
\begin{enumerate}
    \item Jeżeli $2|n$, to $n$ jest liczbą parzystą.
    \item Liczb parzystych jet nieskończenie wiele.
    \item Istnieje dokładnie jedna parzysta liczba pierwsza.
\end{enumerate}
\end{verbatim}

    \begin{enumerate}
        \item Jeżeli $2|n$, to $n$ jest liczbą parzystą.
        \item Liczb parzystych jet nieskończenie wiele.
        \item Istnieje dokładnie jedna parzysta liczba pierwsza.
    \end{enumerate}

Jeżeli chcemy uniknąć wcięcia akapitowego bezpośrednio po wyliczeniu należy zastosować polecenie
\verb|\noindent|. Do list numerowanych można dodać dodatkowy parametr \verb|label=|, np.:

\begin{verbatim}
\begin{definicja}
\emph{Grupą} nazywamy parę $(G,\circ)$, jeżeli spełnione są następujące aksjomaty:
\begin{enumerate}[label=(G$_\arabic*$),leftmargin=*]
 \item działanie $\circ$ jest łączne,
 \item działanie $\circ$ posiada element neutralny,
 \item każdy element posiada element odwrotny.
\end{enumerate}
\end{definicja}
\end{verbatim}

\begin{definicja}
\emph{Grupą} nazywamy parę $(G,\circ)$, jeżeli spełnione są następujące aksjomaty:
\begin{enumerate}[label=(G$_\arabic*$),leftmargin=*]
 \item działanie $\circ$ jest łączne,
 \item działanie $\circ$ posiada element neutralny,
 \item każdy element posiada element odwrotny.
\end{enumerate}
\end{definicja}

\noindent Wyliczenia nienumerowane tworzymy następująco:

\begin{verbatim}
\begin{itemize}
    \item Jeżeli $2|n$, to $n$ jest liczbą parzystą.
    \item Liczb parzystych jet nieskończenie wiele.
    \item Istnieje dokładnie jedna parzysta liczba pierwsza.
\end{itemize}
\end{verbatim}

    \begin{itemize}
        \item Jeżeli $2|n$, to $n$ jest liczbą parzystą.
        \item Liczb parzystych jet nieskończenie wiele.
        \item Istnieje dokładnie jedna parzysta liczba pierwsza.
    \end{itemize}

\section{Obrazki}


Temat umieszczenia obrazków w \LaTeX'u jest zagadnieniem troszkę skomplikowanym, szczególnie jeżeli
chcemy na obrazkach umieścić formuły i symbole matematyczne. Istnieją wyspecjalizowane pakiety do
tworzenia obrazków: \verb|pstricks| oraz \verb|tikz|. Jednak ich użycie może wymagać sporo
dodatkowej nauki.

Do utworzenia obrazków może służyć dowolny program graficzny z możliwością
eksportu do formatu \verb|*.eps| lub \verb|*.pdf|. Polecane programy posiadające możliwością
eksportu do wspomnianych formatów i umieszczania formuł \TeX'owych:
\begin{itemize}
    \item LATEXDraw
    \item Inkscape
\end{itemize}

\begin{figure*}[!b]
  % wyśrodkowanie zawartości pola obrazka
  \begin{center}
    % okienko skalujące:
    %  pierwszy argument szerokość, drugi wysokość,
    %  jeden z nich może być zastąpiony ! - zachowanie proporcji obrazka
    %  w taki sposób możemy skalować także inne obiekty np. tekst
    \resizebox{0.5\textwidth}{!}{
      % wstawienie obrazka
      \includegraphics{Rysunki/KulaEuklidesowa}
    }
    % opis obrazka
    \caption[Skrócony opis do spisu obrazków]{Opis pod obrazkiem. Może być bardzo długi
i zawierać wiele istotnych informacji. Może mówić co przedstawia obrazek, podawać parametry, wzory
itp.}
    % etykieta
    \label{kula_eulidesowa}
  \end{center}
\end{figure*}


\noindent Po tworzeniu obrazków możemy je dołączyć w taki sposób:
\begin{verbatim}
\begin{figure*}[!h]
  % wyśrodkowanie zawartości pola obrazka
  \begin{center}
    % okienko skalujące:
    %  pierwszy argument szerokość, drugi wysokość,
    %  jeden z nich może być zastąpiony ! - zachowanie proporcji obrazka
    %  w taki sposób możemy skalować także inne obiekty np. tekst
    \resizebox{0.5\textwidth}{!}{
      % wstawienie obrazka
      \includegraphics{Rysunki/KulaEuklidesowa}
    }
    % opis obrazka
    \caption[Skrócony opis do spisu obrazków]{Opis pod obrazkiem. Może być bardzo długi
i zawierać wiele istotnych informacji. Może mówić co przedstawia obrazek, podawać parametry, wzory
itp.}
    % etykieta
    \label{etykietka_obrazka}
  \end{center}
\end{figure*}
\end{verbatim}

Proponowanym rozwiązaniem jest także umieszczenie obrazków w odrębnym katalogu znajdującym się w głównym folderze pracy dyplomowej. Przykładowo w niniejszym szablonie utworzono katalog \verb|rysunki|, z którego dołączyć można obrazek za pomocą standardowego sposobu pokazanego powyżej, dodając ścieżkę dostępu do obrazka np. : 
\begin{verbatim}
\includegraphics[scale=0.5]{rysunki/obrazek1.jpg}
\end{verbatim}


Do obrazków możemy odwoływać się używając konstrukcji \verb|\label| -- \verb|\ref|, np.: Rysunek
\ref{kula_eulidesowa} przedstawia kulę w metryce euklidesowej.

\begin{verbatim}
Rysunek \ref{kula_eulidesowa} przedstawia kulę w metryce euklidesowej.
\end{verbatim}

Spis obrazków dołączamy do pracy za pomocą polecenia \verb|\listoffigures|. Sporym problemem może
być umieszczenie obrazka w konkretnym miejscu na stronie. Jeżeli \LaTeX nie potrafi umieścić
obrazka tam gdzie chcemy, to może go przenieść na następną stronę lub na koniec pliku. Prostym
rozwiązaniem jest zmiana wielkości obrazka. W tym wypadku najlepiej spytać Google. Ogólna porada
jest taka: pozycjonowanie obrazków należy wykonać dopiero po napisani całego tekstu, gdyż dodanie
jednej linii tekstu może całkowicie zmienić pozycję obrazka.

Możliwe jest także umieszczenie kilku obrazków na jednym rysunku.

\begin{figure*}[!h]
\centering
\subfloat[$X=\mathbb{R}^2$, metryka taksówkowa]
{
    \resizebox{0.3\textwidth}{!}{
      \includegraphics{Rysunki/KulaEuklidesowa}
    }
}
\subfloat[$X=\mathbb{R}^2$, metryka taksówkowa]
{
    \resizebox{0.3\textwidth}{!}{
      \includegraphics{Rysunki/KulaTaksowkowa}
    }
}
\subfloat[$X=\mathbb{R}^2$, metryka maksimum]
{
    \resizebox{0.3\textwidth}{!}{
      \includegraphics{Rysunki/KulaMaksimum}
    }
}
\caption{Kule w różnych metrykach}
\end{figure*}

\begin{verbatim}
\begin{figure*}[!b]
\centering
\subfloat[$X=\mathbb{R}^2$, metryka taksówkowa]
{
    \resizebox{0.3\textwidth}{!}{
      \includegraphics{Rysunki/KulaEuklidesowa}
    }
}
\subfloat[$X=\mathbb{R}^2$, metryka taksówkowa]
{
    \resizebox{0.3\textwidth}{!}{
      \includegraphics{Rysunki/KulaTaksowkowa}
    }
}
\subfloat[$X=\mathbb{R}^2$, metryka maksimum]
{
    \resizebox{0.3\textwidth}{!}{
      \includegraphics{Rysunki/KulaMaksimum}
    }
}
\caption{Kule w różnych metrykach}
\end{figure*}
\end{verbatim}


\section{Dodawanie listingu kodu}

W razie potrzeby zamieszczenia listingu kodu omawianego w pracy dyplomowej lub krótkich fragmentów tzw. pseudokodu, można wykorzystać pakiet \verb|listings|. Pakiet ten zawarto w dołączonym do szablonu pakiecie \verb|listing_schemat.sty| który znajduje się w głównym katalogu. Pakiet ten dodaje się przez odkometowanie poniższego polecenia na początku pliku głównego \textbf{SzablonPracyPG\_main.tex}:
\begin{verbatim}
%-------------------- Dodatkowe pakiety ---------------------
\usepackage{listing_schemat}
%------------------------------------------------------------
\end{verbatim}

Przykład  zamieszczenia listingu kodu w pracy przedstawiono poniżej.
\begin{verbatim}
\lstset{style=python}
\begin{lstlisting}[caption={Przykładowy listing w języku Python}, label=pyton]
%%%%%%% KOD ŹRÓDŁOWY %%%%%%%
#importing the time module
import time
#welcoming the user
name = raw_input("What is your name? ")
print "Hello, " + name, "Time to play hangman!"
#wait for 1 second
time.sleep(1)
%%%%%%%%%%%%%%%%%%%%%%%%%%%%
\end{lstlisting}
\end{verbatim}

Po skompilowaniu, otrzymuje się listing w postaci przedstawionej poniżej.
\lstset{style=python}
\begin{lstlisting}[caption={Przykładowy listing w języku Python}, label=pyton]
#importing the time module
import time
#welcoming the user
name = raw_input("What is your name? ")
print "Hello, " + name, "Time to play hangman!"
#wait for 1 second
time.sleep(1)
\end{lstlisting}

Wstawienie listingu rozpoczyna się poleceniem \verb|\lstset{style=NAZWA_JĘZYKA}|, które ustawia określony sposób formatowania listingu zależnie od wybranego języka. Nazwę języka należy wpisać małymi literami. Opracowane formaty języków znajdują się w pliku pakietu \textbf{listing\_schemat}. Następnie do polecenia \verb|\begin{listings}| podaje się argumenty wejściowe w kwadratowych nawiasach, co przedstawione zostało poniżej. 
\begin{verbatim}
\begin{lstlisting}[caption={Podpis wstawianego listingu}, label=Odnośnik do listingu]
\end{lstlisting}
\end{verbatim}
Podanie wymienionych argumentów wejściowych umożliwia podpisywanie poszczególnych listingów oraz odwoływanie się do nich w tekście przy użyciu polecenia \verb|\ref{}|.

Istnieje także możliwość dołączania kodu źródłowego, który został zapisany w osobnym pliku. W tym celu należy użyć poniższego polecenia.
\begin{verbatim}
\begin{lstlisting}[caption={Podpis wstawianego listingu}, label=Odnośnik do listingu]
\lstinputlisting[language=Python, firstline=10, lastline=20]{plik_zrodlowy.py}
\end{lstlisting}

\end{verbatim}

W pliku pakietu \textbf{listing\_schemat.sty} można zdefiniować własny styl formatowania listingu, zgodny z używanym językiem programowania. Parametry stylu formatowania zostały opisane w pliku, co umożliwia również ewentualną zmianę kolorów składni kodu lub innych atrybutów.

Do pakietu dołączono również środowisko \verb|tikz|, które umożliwia generowanie przejrzystych schematów blokowych, grafów, diagramów oraz wykresów. Tworzenie diagramów poprzez opracowanie kodu w edytorze tekstu może być uciążliwe, jednak istnieją graficzne generatory kodu np. \textbf{TikzEdt} dostępny pod adresem: \textbf{http://www.tikzedt.org/}.

\section{Dodawanie bibliografii}

Bibliografię dodajemy za pomocą następujących poleceń

\begin{verbatim}
\bibliographystyle{plain}                       % styl bibliografii
\begin{thebibliography}{3}                      % początek środowiska
\addcontentsline{toc}{chapter}{Bibliografia}    % dodaje bibliografię do spisu treści
\small              % spisy i bibliografie składamy mniejszym stopniem pisma
% przykładowy wpis
    \bibitem{Duda}      % \bibitem{etykieta}
A. Duda,\emph{Wprowadzenie do topologii}, PWN, Warszawa 1986
% następna pozycja
    \bibitem{EngeSiek}
R. Engelking, K. Sieklucki, \emph{Geometria i topologia. Część II. Topologia}, PWN, Warszawa 1980
% następna pozycja
    \bibitem{Patk}
H. Patkowska, \emph{Wstęp do topologii}, PWN, Warszawa 1979
% następna pozycja
    \bibitem{Siek}
K. Sieklucki, \emph{Geometria i topologia. Część I. Geometria}, PWN, Warszawa 1979
% następna pozycja
    \bibitem{Rutkowski}
Rutkowski J., \emph{Algebra Abstrakcyjna w zadaniach},  Wydawnictwo Naukowe PWN, Warszawa 2005
\end{thebibliography}                           % koniec środowiska
\end{verbatim}

Do pozycji bibliograficznych możemy odwoływać się w tekście korzystając z polecenia \verb|\cite|,
np.: Głównym źródłem jest książka A. Dudy \cite{Duda}. Przykłady podane w tym rozdziale pochodzą z
książki \cite{Rutkowski}.

\begin{verbatim}
Głównym źródłem jest książka A. Dudy \cite{Duda}.
Przykłady podane w tym rozdziale pochodzą z książki \cite{Rutkowski}.
\end{verbatim}

\section{Uwagi o typografii}

W polskiej typografii (i nie tylko w polskiej) przyjęło się stosować następujące zasady:
\begin{itemize}
    \item na końcu wiersza nie mogą zostać słowa jednoliterowe (dwu-, trzy-) (tzw. sierota)
    \item akapit nie może kończyć się bardzo krótkim wierszem: pojedyncze słowo, przeniesiona
część słowa lub krótkich słów, orientacyjnie, mniej niż 3 długości wcięcia akapitowego (tzw. wdowa)
\tiny{<-- wdowa}\normalsize
    \item strona nie może kończyć się pojedynczym wierszem nowego akapitu (tzw. szewc)
    \item strona nie może zaczynać się ostatnim wierszem poprzedniego akapitu (tzw. bękart)
    \item unikać podwójnych wyróżnień, nie należy wyróżnianego fragmentu tekstu wyróżniać więcej niż
jedną metodą naraz (np.: \textbf{\textit{podwójne wyróżnienie}})
\begin{verbatim}
https://pl.wikipedia.org/wiki/Zasada_unikania_podwójnych_wyróżnień
\end{verbatim}
\end{itemize}

Niestosowanie się do tych zasad nie umniejsza wartości pracy, jednak zmniejsza jej estetykę a w
przypadku pojedynczych wierszy utrudnia czytanie. W tym dokumencie można znaleźć przykłady takich
sytuacji.


Aby usunąć wspomniane błędy zwykle wystarczy zmienić szyk zdania lub użyć innego słowa. W
ostateczności(!) można dodać pusty wiersz. Aby usunąć pojedyncze litery na końcu wiersza można
dodać tzw. twardą spację, umieszczając pomiędzy wyrazami znak tyldy \textasciitilde.

\begin{przyklad}
długiesłowo i krótkie i długiesłowo i krótkie i długiesłowo i krótkie i długiesłowo i krótkie i
długiesłowo i krótkie i długiesłowo i krótkie i długiesłowo i krótkie i długiesłowo i krótkie i
\end{przyklad}

\begin{przyklad}
długiesłowo i krótkie i długiesłowo i krótkie i długiesłowo i krótkie i długiesłowo i~krótkie i
długiesłowo i krótkie i długiesłowo i krótkie i długiesłowo i krótkie i długiesłowo i krótkie i
\end{przyklad}


Podobnie jak w przypadku obrazków, poprawki typograficzne powinny być wprowadzane po zakończeniu
pisania pracy, gdyż nawet najmniejsza zmiana w poprzednim wierszu może skutkować poważnymi
przesunięciami całego tekstu.


Jako ciekawostka: norma PN-83/P-55366 (nieobowiązująca od kilku lat) podaje serię wskazówek i
zaleceń typograficznych.