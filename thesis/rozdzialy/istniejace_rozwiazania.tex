\chapter{Aktualny stan wiedzy (\Hania,\ \Ola,\ \Jakub,\ \Karol)}
Celem niniejszego rozdziału jest przegląd wybranych aplikacji mobilnych dedykowanych treningowi oddechowemu oraz przedstawienie naukowego uzasadnienia dla ich stosowania. W~kolejnych podrozdziałach opisano kontekst medyczny wykorzystania technologii mobilnych w rehabilitacji oddechowej, a następnie przeanalizowano indywidualne aplikacje, zwracając uwagę na~ich specyfikę funkcjonalną, zastosowane rozwiązania technologiczne, interfejs użytkownika oraz ograniczenia.

\section{Kontekst naukowy i skuteczność aplikacji mobilnych (\Karol)}
W dobie cyfryzacji opieki zdrowotnej, aplikacje mobilne stają się istotnym narzędziem wspierającym tradycyjne metody rehabilitacji. Przeglądy systematyczne wskazują, że samodzielna rehabilitacja pulmonologiczna wspierana przez aplikacje mobilne może być skuteczną alternatywą dla rehabilitacji stacjonarnej, szczególnie w przypadku pacjentów z przewlekłą obturacyjną chorobą płuc (POChP) \cite{MobileApp_Rehab}. Aplikacje te pozwalają na przezwyciężenie barier takich jak brak dostępu do specjalistycznych ośrodków, koszty czy problemy z transportem. Wykazano, że interwencje oparte na aplikacjach mogą prowadzić do poprawy wydolności fizycznej i łagodzenia objawów chorobowych, oferując jednocześnie spersonalizowane plany treningowe, co jest kluczowe dla utrzymania zaangażowania pacjenta.

\section{Istniejące rozwiązania wspierające trening oddechu (\Jakub)}\label{sec:istniejace_rozwiazania}
\begin{itemize}
    \item Wim Hof Method firmy WHM Services
    \item Breathwrk: Breathing Exercises firmy Breathwrk Inc.
    \item STAmina Apnea Trainer firmy Squarecrowd Apps SIA
    \item Breathe firmy Havabee
\end{itemize}

\section{Wim Hof Method firmy WHM Services (\Hania)}
Na szczególną uwagę wśród dostępnych rozwiązań zasługuje aplikacja Wim Hof Method \cite{WHM} stworzona przez firmę WHM Services. Wykorzystuje ona techniki oddechowe opracowane przez holenderskiego sportowca i trenera, Wima Hofa. Wyróżnia się ona kompleksowym podejściem, łącząc ćwiczenia z ekspozycją na zimno oraz medytacją. 

\subsection{Cel aplikacji}
Celem aplikacji jest udostępnienie użytkownikom dostępu do regularnych treningów oddechowych zgodnie z założeniami metody Wim Hofa. Została zaprojektowana z~myślą o~osobach szukających sposobów na poprawę zdrowia psychofizycznego, zwłaszcza poprzez codzienne ćwiczenia oddechowe i ekspozycję na zimno. Ze względu na zróżnicowany poziom wiedzy technicznej oraz wiedzy z zakresu zdrowia wśród użytkowników, kluczową cechą aplikacji jest prosty i intuicyjny interfejs oraz stopniowe, prowadzące użytkownika krok po kroku wdrożenie w proces ćwiczeń.

\subsection{Funkcjonalności}
Aplikacja podzielona jest na trzy główne zakładki: \textit{Results} (pol. Rezultaty), \textit{Home} (pol. Strona główna) oraz \textit{Community} (pol. Społeczność).

Celem zakładki \textit{Results} (pol. Rezultaty) jest przedstawienie wszystkich postępów użytkownika. Na podstronie \textit{Bar Chart} (pol. Wykres słupkowy) znajdują się statystyki dotyczące wykonanych \textit{Breathing sessions} (pol. sesji oddechowych), \textit{Cold showers} (pol. zimnych pryszniców) czy \textit{Ice baths} (pol. kąpieli w lodzie). Poniżej znajduje się także wykres słupkowy i podsumowanie ilościowe dotyczące konkretnego rodzaju ćwiczenia, które można wybrać z dostępnej listy. Aby zarejestrować zadanie, nie jest konieczne wykonywanie go wraz z aplikacją, można to zrobić, klikając w przycisk \textit{Log exercise} (pol. Zarejestruj ćwiczenie).

Na podstronie \textit{Calendar} (pol. Kalendarz) dostępny jest widok miesiąca z możliwością przeglądania szczegółów poszczególnych dni. Użytkownik może edytować lub usuwać zarejestrowane aktywności, jednak dodawanie nowych jest niedostępne. Istnieje także opcja dzielenia się wynikami ze społecznością.

Podstrona \textit{Badges} (pol. Odznaki) motywuje użytkownika do dalszej pracy poprzez system osiągnięć i nagród za regularność i stopień trudności ćwiczeń. Na rysunku \ref{img/whm_results} przedstawiono podstrony \textit{Bar Chart}, \textit{Calendar} i \textit{Badges} zakładki \textit{Results} aplikacji Wim Hof Method.

\begin{figure}[ht]
\centering
\includegraphics[scale=0.25]{obrazki/istniejace_rozwiazania/whm/whm_results.png}
\caption{Zakładka \textit{Results} (\textit{Bar Chart, Calendar, Badges}) w aplikacji Wim Hof Method}
\label{img/whm_results}
\end{figure}

Zakładka \textit{Home} (pol. Strona główna) oferuje użytkownikowi dostęp do różnych typów ćwiczeń: \textit{Breathing Exercises} (pol. Trening oddechowy), \textit{Cold Exposure} (pol. Ekspozycja na zimno) — obejmującej m.in. zimne prysznice i kąpiele w lodzie — oraz trening \textit{Power of the Mind} (pol. Siły umysłu), który w praktyce stanowi zestaw ćwiczeń fizycznych. Oprócz głównych modułów treningowych, użytkownik ma dostęp do treści dodatkowych, takich jak nagrania audio (np. internetowe publikacje Wima Hofa), sesje medytacyjne, materiały internetowe wyjaśniające zasady działania metody oraz różnego rodzaju wyzwania. Interfejs zakładki \textit{Home} (pol. Strona główna) przestawiono na rysunku \ref{img/whm_home}.

Zakładka z ćwiczeniami oddechowymi umożliwia użytkownikowi skorzystanie z gotowych zestawów treningowych oraz ich edycję. Po wybraniu konkretnego ćwiczenia dostępny jest szereg opcji personalizacji, takich jak: tempo oddechu, liczba rund, liczba oddechów przed wstrzymaniem, wybór podkładu muzycznego oraz wyłączenie głosu prowadzącego. 

Po uruchomieniu sesji użytkownik widzi animację w formie sześciokąta, który sygnalizuje pozostały czas do ukończenia etapu oddechu. Wyświetlane są także liczba rund i liczba oddechów. W dowolnym momencie istnieje możliwość przerwania ćwiczenia i powrotu do ekranu głównego — postęp zostaje mimo to zapisany jako ukończona sesja. Dodatkowo, podwójne kliknięcie w trakcie ćwiczenia pozwala na przejście do fazy wstrzymania oddechu. Proces treningu został przedstawiony na~rysunku \ref{img/whm_training}.

Zakładka \textit{Community} (pol. Społeczność) umożliwia śledzenie aktywności innych użytkowników aplikacji. Aby uzyskać dostęp do postępów konkretnej osoby, należy ją wyszukać i zaobserwować. Użytkownik może również udostępniać własne wyniki za pośrednictwem spersonalizowanego profilu, w którym można zamieścić takie informacje jak imię i nazwisko, zdjęcie profilowe oraz krótki opis. W profilu widoczne są również: lista obserwowanych i obserwujących użytkowników, zdobyte odznaki oraz liczba ukończonych treningów. Intrefejs zakładki \textit{Community} (pol. Społeczność) został przedstawiony na~rysunku \ref{img/whm_community}.

\begin{figure}[H]
    \centering
    \begin{minipage}{0.48\textwidth}
        \centering
        \includegraphics[scale=0.25]{obrazki/istniejace_rozwiazania/whm/whm_home.png}
        \caption{Zakładka \textit{Home} w aplikacji Wim Hof Method}
        \label{img/whm_home}
    \end{minipage}
    \hfill
    \begin{minipage}{0.48\textwidth}
        \centering
        \includegraphics[scale=0.25]{obrazki/istniejace_rozwiazania/whm/whm_community.png}
        \caption{Zakładka \textit{Community} w aplikacji Wim Hof Method}
        \label{img/whm_community}
    \end{minipage}
\end{figure}

\begin{figure}[H]
    \centering
    \includegraphics[scale=0.25]{obrazki/istniejace_rozwiazania/whm/whm_training.png}
    \caption{Trening oddechowy w aplikacji Wim Hof Method}
    \label{img/whm_training}
\end{figure}

\subsection{Ograniczenia}
Pomimo szerokiego zakresu funkcjonalności, aplikacja posiada również pewne ograniczenia, które z perspektywy użytkownika mogą mieć istotne znaczenie.

Pierwszym z nich jest dostęp do wielu funkcji wyłącznie w wersji płatnej \textit{Premium}. Ograniczenia te dotyczą m.in. wyboru rodzaju ćwiczeń, uczestnictwa w sesjach medytacyjnych oraz podejmowania wyzwań. Choć aplikacja jest dostępna i użyteczna także w wersji podstawowej, dla bardziej zaawansowanych użytkowników może nie oferować wystarczającego wsparcia rozwojowego. Warto jednak zauważyć, że model subskrypcyjny stanowi źródło przychodu.
\par
Drugim ograniczeniem jest brak możliwości tworzenia własnych, w pełni spersonalizowanych treningów. Aplikacja koncentruje się na ściśle określonych założeniach metody Wima Hofa, co ogranicza elastyczność i możliwość dostosowania ćwiczeń do indywidualnych potrzeb. Dostępna personalizacja ogranicza się głównie do zmiany długości ćwiczeń, liczby rund czy tempa oddychania. Dla bardziej zaawansowanych użytkowników może to stanowić istotne utrudnienie.


\section{Breathwrk: Breathing Exercises firmy Breathwrk Inc. (\Ola)}
Breathwrk \cite{Breathwrk} to aplikacja, która w 2022 roku zwyciężyła w kategorii \textit{Najlepsza aplikacja do rozwoju osobistego} w \textit{Google Google Play’s Best of 2022}. Jest reklamowana jako aplikacja numer jeden do ćwiczeń oddechowych. Posiada wysoką ocenę zarówno w \textit{Sklepie Play} --- 4.1, jak i~w~\textit{App Store} --- 4.8. Według twórców posiada około dwa miliony użytkowników.

\subsection{Cel aplikacji}
Aplikacja skupia się na pomocy użytkownikowi poprzez trening oddechowy w czterech głównych obszarach: śnie, stresie, koncentracji oraz energii. Oferuje setki ćwiczeń oraz zajęć, czyli sesji z doświadczonymi instruktorami, którzy przeprowadzają użytkownika  przez starannie dobraną serię ćwiczeń. Breathwrk ma także służyć jako pomoc w budowaniu nawyku regularnych treningów oddechowych, dzięki przypomnieniom według ustalonych harmonogramów. Elementem wyróżniającym są także codziennie dodawane, nowe zajęcia immersyjne oferujące ciekawe wrażenia wizualne. Interesującym dodatkiem są również wibracje telefonu podczas treningu, jednak ustawienie to nie jest dostępne na wszystkich urządzeniach. Możliwy jest dostęp offline do aplikacji, a jej językiem głównym jest angielski.

\subsection{Funkcjonalności}
Aplikacja składa się z czterech głównych zakładek dostępnych na pasku w~dole aplikacji: \textit{Now} (pol. Teraz), \textit{Explore} (pol. Odkrywaj), \textit{Schedule} (pol. Zaplanuj), \textit{Me} (pol. Ja).

W prawym, górnym rogu znajduje się okrągła ikona ze zdjęciem profilowym lub wybranym z~biblioteki zdjęciem zwierzęcia, liczbą oznaczającą poziom rankingowy oraz postęp w osiągnięciu kolejnego poziomu rankingowego. Poziom rankingowy można zwiększać poprzez wykonywanie ćwiczeń. Po kliknięciu na ikonę pokazuje się okno \textit{Shortcuts} (pol. Skróty) z dwoma zakładkami. W jednej z nich znajdują się treningi oznaczone jako ulubione, a w drugiej ostatnio wykonane treningi. Pokazuje się także ikona ołówka, po kliknięciu którego możliwa jest edycja zdjęcia profilu. Zakładki te przedstawione zostały na rysunku \ref{img/wrk_shortcuts}.

\begin{figure}[H]
    \centering
    \includegraphics[scale=0.55]{obrazki/istniejace_rozwiazania/breathwrk/wrk_shortcuts.png}
    \caption{Okno \textit{Shortcuts} w aplikacji Breathwrk}
    \label{img/wrk_shortcuts}
\end{figure}

Główna zakładka aplikacji to \textit{Now}. Na samej górze widoczny jest streak, a także zakładka \textit{favourites} (pol. ulubione).
Środkową część zajmuje karuzela z treningami oddechowymi, które są obecnie trendujące lub zostały wybrane dla użytkownika. W prawym górnym rogu dostępna jest opcja dodania treningu do ulubionych poprzez kliknięcie ikony serca, a także ikona informacyjna, która otwiera okno z opisem treningu. Dostępny jest ogólny opis treningu, propozycje użycia treningu oraz opis biologicznych procesów, które aktywuje trening. U góry okna umieszczone zostały piktogramy symbolizujące w jaki sposób wykonywać wdechy i wydechy oraz czas trwania poszczególnych etapów. Na rysunku \ref{img/wrk_opis} przedstawiono widok opisu trenigu.

\begin{figure}[H]
    \centering
    \includegraphics[scale=0.50]{obrazki/istniejace_rozwiazania/breathwrk/wrk_opis.png}
    \caption{Podzakładki strony opisu treningu}
    \label{img/wrk_opis}
\end{figure}

Po uruchomieniu treningu pokazuje się domyślna animacja --- koło. Po kliknięciu na środek ekranu pokazuje się czas do zakończenia treningu, ikona informacji,odnosząca do opisu treningu, przycisk pauzy oraz rezygnacji z treningu. Możliwe jest też dodanie treningu do ulubionych i jego udostępnienie. Dostępne są także opcje konfiguracji treningu --- wybór animacji (koło, małpka lub linia), dodanie nawyku, włączenie instrukcji głosowych, włączenie i wybór dźwięku w tle, a także wibracje telefonu. Widok tej strony przedstawiony został na rysunku \ref{img/wrk_trening}.
Po zakończeniu treningu pokazuje się okno ze statystykami oraz gratulacjami. Pokazuje się także przycisk do ponowienia treningu z wybranymi wcześniej ustawieniami.

\begin{figure}[H]
    \centering
    \includegraphics[scale=0.48]{obrazki/istniejace_rozwiazania/breathwrk/wrk_trening.png}
    \caption{Strona treningu w aplikacji Breathwrk}
    \label{img/wrk_trening}
\end{figure}
\begin{figure}[H]
    \centering
    \includegraphics[scale=0.85]{obrazki/istniejace_rozwiazania/breathwrk/wrk_immersje.png}
    \caption{Przykładowe treningi immersyjne w aplikacji Breathwrk}
    \label{img/wrk_immersje}
\end{figure}

Na dole zakładki znajduje się podsekcja \textit{Daily Classes} (pol. Codzienne Ćwiczenia) z zestawem 3 ćwiczeń immersyjnych, które zmieniają się co 24 godziny. Ćwiczenia te  charakteryzują się nietypowymi wrażeniami wizualnymi --- efekt promieniowania czy animacja wody w basenie. Przykładowe ćwiczenie przedstawione zostały na rysunku \ref{img/wrk_immersje}.

\textit{Explore}  to zakładka będąca biblioteką treningów, ćwiczeń, wyzwań, nawyków (opcja dla użytkowników  \textit{Breathwrk Premium}) oraz testów. Dostępne są także zakładki, po których można szybko filtrować treść: \textit{All}, \textit{Calming}, \textit{Nighttimel}, \textit{Energizing}, \textit{Perform} oraz \textit{Health}. 
Rysunek \ref{img/wrk_discover} przedstawia wygląd tej zakładki.

\begin{figure}[H]
    \centering
    \includegraphics[scale=0.5]{obrazki/istniejace_rozwiazania/breathwrk/wrk_discover.png}
    \caption{Zakładka \textit{Discover} w aplikacji Breathwrk)}
    \label{img/wrk_discover}
\end{figure}

\begin{figure}[H] 
    \centering
    \includegraphics[scale=0.5]{obrazki/istniejace_rozwiazania/breathwrk/wrk_me.png}
    \caption{Zakładka \textit{Me}}
    \label{img/wrk_me}
\end{figure}


Celem zakładki \textit{Schedule} jest zachęcenie użytkownika do systematycznego używania aplikacji. Możliwe jest ustawienie pojedynczego przypomnienia na trening lub regularnych przypomnień w wybrane dni, tak aby stworzyć nawyk.

 
Zakładka \textit{Me} składa się z trzech sekcji: statystyk użytkownika, testów udoskonalających oddech oraz tabeli najlepszych użytkowników, ocenianych na podstawie miesięcznego lub tygodniowego czasu spędzonego na aktywnościach oddechowych w aplikacji. Zakładka przedstawiona została na rysunku \ref{img/wrk_me}

\subsection{Ograniczenia}
Aplikacja jest przedstawiana jako umożliwiająca pełną personalizację ćwiczeń oddechowych, mimo to zawiera ograniczenia.

W aplikacji brakuje przede wszystkim możliwości wyboru długości trwania poszczególnych kroków. Możliwy jest jedynie wybór czasu trwania całego treningu, jednak jest on także ograniczony do trzech predefiniowanych wyborów dla mniejszej ilości interwałów. Dla większej liczby interwałów można je wybrać z dokładnością do jednego interwału, jest to natomiast funkcja dostępna jedynie w płatnym planie \textit{Breathwrk Premium}

Kolejnym ograniczeniem jest brak możliwości tworzenia własnych treningów. Do wyboru pozostają wyłącznie treningi oferowane przez aplikację. Dla użytkowników wersji \textit{Breathwrk Premium} wybór gotowych treningów i zajęć jest szeroki, natomiast dla pozostałych użytkowników jest mocno ograniczony --- po jednym lub nawet braku treningu spośród kilku z każdej z kategorii.

\section{STAmina Apnea Trainer firmy Squarecrowd Apps SIA (\Karol)} 
Kolejnym ważnym rozwiązaniem jest aplikacja STAmina Apnea Trainer \cite{Squarecrowd}, stworzona przez firmę Squarecrowd Apps SIA. To narzędzie mobilne wspomagające trening wstrzymywania oddechu, szczególnie przydatne dla freediverów, sportowców oraz osób praktykujących techniki oddechowe i relaksacyjne.

\subsection{Cel aplikacji}
STAmina Apnea Trainer to narzędzie przeznaczone przede wszystkim dla osób ćwiczących freediving, łowiectwo podwodne oraz inne dyscypliny wodne, w których kluczowe znaczenie ma umiejętność długiego wstrzymywania oddechu. Głównym założeniem aplikacji jest zapewnienie kompleksowego wsparcia w treningu \textit{static apnea} (pol. bezdech statyczny), wykorzystującego różnorodne tablice treningowe (między innymi ukierunkowane na tolerancję na niedotlenienie --- O$_2$, nadmiar dwutlenku węgla --- CO$_2$, a także kombinacje, techniki relaksacyjne i możliwość tworzenia własnych schematów).

\subsection{Funkcjonalności}
Aplikacja udostępnia pięć gotowych szablonów treningowych, które pozwalają systematycznie rozwijać poszczególne aspekty wstrzymywania oddechu.
Pierwszy z nich koncentruje się na niedoborze O$_2$, czyli stopniowym wydłużaniu czasu wstrzymania przy stałym czasie odpoczynku. 
Drugi skupia się na tolerancji CO$_2$, w którym użytkownik zmniejsza przerwy między kolejnymi sesjami wstrzymania oddechu, by przyzwyczaić organizm do wyższego poziomu CO$_2$.
Kolejny schemat, nazwany \textit{Wonka}, wprowadza pauzę po pierwszym odczuwalnym skurczu przepony, co pomaga uczyć się właściwej techniki. 
Schemat \textit{Mix} (pol. Mieszany) łączy oba podejścia (O$_2$/CO$_2$), zapewniając wszechstronne podejście do treningu, natomiast \textit{Pranayama} oferuje techniki oddechowe o charakterze relaksacyjnym, przydatne zarówno przed jak i po głównej części treningu. Na rysunku \ref{img/sta_schematy} przedstawiono gotowe schematy dostępne w~aplikacji.

\begin{figure}[ht]
\centering
\includegraphics[scale=0.25]{obrazki/istniejace_rozwiazania/STA/STA_schematy.png}
\caption{Gotowe schematy w aplikacji STAmina Apnea Trainer}
\label{img/sta_schematy}
\end{figure}

 Poza gotowymi programami każdy użytkownik może stworzyć trening \textit{Custom} (pol. własny) z pełną konfigurowalnością, dobierając liczbę powtórzeń, czasy wstrzymywania i przerw według indywidualnych potrzeb. Na podstawie dotychczasowego \textit{Personal Best} (pol. rekordu) aplikacja automatycznie generuje sesje w trzech stopniach zaawansowania --- \textit{Easy} (pol. łatwy), \textit{Normal} (pol. normalny), \textit{Hard} (pol. trudny), co ułatwia progresję i dostosowanie wysiłku do aktualnych możliwości. Tworzenie treningu przedstawiono na rysunku \ref{img/sta_generate}.
 
 Aby wspomóc systematyczne korzystanie, wbudowany system powiadomień przypomina o zaplanowanych treningach. Wszystkie sesje są zapisywane, a użytkownik może obserwować swoje postępy dzięki szczegółowym przeglądom rekordów i trendów w czasie. Dodatkowym wsparciem jest nawigacja głosowa prowadzona przez nagrania profesjonalnych lektorów w wersjach męskiej i żeńskiej, dostępna w kilku językach (m.in. angielskim, francuskim, niemieckim, włoskim i rosyjskim), co pozwala skupić się wyłącznie na ćwiczeniu, bez konieczności patrzenia na ekran. Integracja z Apple Health umożliwia rejestrowanie tętna i poziomu SpO$_2$ za pomocą Apple Watch lub innych urządzeń Bluetooth, co dostarcza cennych danych biometricznych podczas treningów. Przedstawiono to na rysunku \ref{img/sta_heart}.
 
 Synchronizacja z chmurą gwarantuje \textit{backup} (pol. kopia zapasowa) i przywracanie danych pomiędzy urządzeniami, co jest istotne dla osób korzystających z aplikacji na różnych sprzętach. Aplikacja oferuje także możliwość śledzenia momentu wystąpienia skurczów oddechowych, co pomaga w lepszym zrozumieniu własnych granic i rozwoju techniki. Dla lepszego komfortu dostępne są wibracyjne alerty oraz \textit{square breath} (pol. tryb oddechu pudełkowego), wspierające relaksację i stabilizację oddechu.

  \begin{figure}[H]
    \centering
    \begin{minipage}{0.48\textwidth}
        \centering
        \includegraphics[scale=0.15]{obrazki/istniejace_rozwiazania/STA/STA_generate.png}
\caption{Tworzenia treningu w aplikacji STAmina Apnea Trainer}
\label{img/sta_generate}
    \end{minipage}
    \hfill
    \begin{minipage}{0.48\textwidth}
        \centering
        \includegraphics[scale=0.15]{obrazki/istniejace_rozwiazania/STA/STA_serce.png}
        \caption{Wykresy poziomu SpO$_2$ w aplikacji STAmina Apnea Trainer}
        \label{img/sta_heart}
    \end{minipage}
\end{figure}
 
 Zakładka z ogólnymi pomocami dotyczącymi treningów oddechowych oraz freedivingu dostarcza wiedzy teoretycznej i praktycznych wskazówek, co pomaga użytkownikom zarówno początkującym, jak i bardziej zaawansowanym.

\subsection{Ograniczenia}
Mimo bogatego zestawu funkcji, aplikacja ma też swoje wady. Interfejs jest już nieco przestarzały, co może obniżać komfort użytkowania i sprawiać, że poruszanie się po aplikacji wydaje się mniej intuicyjne w porównaniu ze współczesnymi standardami. Brak wbudowanego przewodnika lub samouczka powoduje, że nowi użytkownicy mogą mieć trudności z rozpoczęciem treningów. Pomocne mogłoby okazać się interaktywne wskazówki dla początkujących czy wprowadzenie krok po kroku.

\section{Breathe firmy Havabee (\Jakub)}
Kolejną aplikacją wartą uwagi jest Breathe \cite{Havabee} firmy Havabee. Wyróżnia się spośród innych swoim minimalizmem i łatwością obsługi. 

\subsection{Cel aplikacji}
Celem aplikacji jest umożliwienie użytkownikom, niezależnie od wieku i zaawansowania technologicznego, wykonywania treningów oddechowych. Została ona zaprojektowana z~myślą o~osobach niewymagających wiele, co potwierdzają proste treningi dostarczone razem z aplikacją.

\subsection{Funkcjonalności}
Aplikacje w swoim przejrzystym interfejsie zawiera jedynie trzy podstrony: \textit{Domyślnie} (Strona główna), \textit{Własne} (Strona konfiguracji treningów) i \textit{Postęp} (Strona ze statystykami).

Podstrona \textit{Domyślnie} zawiera cztery zdefiniowane przez twórców aplikacji treningi, takie jak: \textit{Oddychanie równoważne}, \textit{Oddychanie pudełkowe}, \textit{Oddychanie sposobem 4-7-8} i \textit{Test wstrzymania oddechu}. Aby dowiedzieć się więcej o danym treningu, wystarczy kliknąć na ikonę informacji. Możliwa jest również zmiana trwania schematów poprzez kliknięcie w przycisk \textit{Zmień czas trwania}. Gotowe schematy zostały przedstawione na rysunku \ref{img/havabee_menu}. 

\begin{figure}[H]
    \centering
    \begin{minipage}{0.48\textwidth}
        \centering
        \includegraphics[scale=0.11]{obrazki/istniejace_rozwiazania/Havabee/Havabee_menu.jpg}
\caption{Schematy dostępne w~aplikacji Breathe}
\label{img/havabee_menu}
    \end{minipage}
    \hfill
    \begin{minipage}{0.48\textwidth}
        \centering
        \includegraphics[scale=0.10]{obrazki/istniejace_rozwiazania/Havabee/Havabee_trening.jpg}
        \caption{Trening w aplikacji Breathe}
        \label{img/havabee_trening}
    \end{minipage}
\end{figure}

Po wybraniu interesującego nas schematu przenoszeni jesteśmy do strony treningu. Aby przygotować użytkownika do ćwiczeń, w pierwszej kolejności odbywa się krótka sesja pozwalająca skupić się na własnym oddechu i przejść w stan świadomej pracy oddechowej. Po niej przechodzimy do właściwego ćwiczenia, które prowadzone jest zarówno wizualnie, jak i w formie audio. Liczba odbytych cykli wyświetlana jest w postaci zapełniającej się obwódki kółka, jednak brak bezpośredniego wyświetlania wartości liczbowych.
Dostępne jest również ustawienie poziomu dźwięku, jednakże brakuje możliwości zatrzymania treningu. Można jedynie powrócić do okna startowego, w dodatku bez ostrzeżenia użytkownika o natychmiastowym zakończeniu sesji. Trening został przedstawiony na rysunku \ref{img/havabee_trening}.

Celem kolejnej zakładki jest umożliwienie użytkownikom tworzenie własnych wzorów oddychania. Po kliknięciu w przycisk plusa (+), zostajemy przeniesieni na stronę konfiguracji treningu. Możliwe jest nadanie treningowi nazwy, a także wybranie czasu trwania każdej z faz oddechu (Wdech, Wstrzymanie, Wydech, Wstrzymanie) i liczby cykli w przedziale od jednego do dwustu pięćdziesięciu. Na rysunku \ref{img/havabee_custom} przedstawiono tę zakładkę. 

Na ostatniej stronie możemy podejrzeć czas, jaki spędziliśmy w aplikacji oraz liczbę odbytych sesji w danym dniu, tygodniu, miesiącu i roku. Poniżej ogólnych statystyk znajduje się także osobna sekcja do podglądu odbytych treningów wstrzymywania oddechu, na której można zaobserwować przeciętny i najdłuższy czas trwania takiego oddechu w tych samych ramach czasowych.

\begin{figure}[H]
    \centering
    \includegraphics[scale=0.20]{obrazki/istniejace_rozwiazania/Havabee/havabee_custom.png}
    \caption{Tworzenie własnych treningów w aplikacji Breathe}
    \label{img/havabee_custom}
\end{figure}

\subsection{Ograniczenia}
Wspomniana aplikacja, mimo swojej prostej szaty graficznej oraz oferowanych funkcjonalności, nie jest w stanie sprostać oczekiwaniom wszystkich użytkowników. Każdy z dostępnych wzorców treningowych umożliwia jedynie powtarzanie określonego cyklu oddechowego z góry ustaloną liczbę razy. Stanowi to istotne ograniczenie dla osób, które w swoich treningach wykorzystują zróżnicowane cykle oddechowe lub takie, które składają się z mniejszej liczby faz. Dodatkowym mankamentem jest brak możliwości ustawienia czasu wstrzymania oddechu na zero sekund lub bardzo krótkich wdechów, co skutkuje nieprawidłowym działaniem dźwięku towarzyszącego ćwiczeniu. Koniecznie należy wspomnieć o braku możliwości zatrzymania treningu i mało przejrzystego wyświetlania liczby cykli podczas treningu.

\section{Podsumowanie analizowanych aplikacji \Hania}

W tabeli \ref{tab:podsumowanie_aplikacji} zestawiono kluczowe cechy omówionych aplikacji. Pozwala to na szybkie porównanie ich funkcjonalności, zakresu personalizacji oraz ograniczeń. Zestawienie uwzględnia zarówno aplikacje o rozbudowanej strukturze, jak i rozwiązania minimalistyczne, skupione na prostych schematach oddechowych.

Z przeprowadzonej analizy wynika, że dostępne aplikacje wyraźnie dzielą się na dwa \linebreak typy --- jeden stawia na doświadczenie i wygląd, a drugi na szeroką możliwość konfiguracji. \linebreak W obu podejściach pojawiają się ograniczenia --- pierwsze oferują mało elastyczności, a drugie często mają uproszczoną formę. Żadna nie łączy pełnej personalizacji z nowoczesnym, lekkim podejściem, co pokazuje lukę i uzasadnia stworzenie bardziej elastycznego rozwiązania.

\begin{table}[H]
\centering
\hspace*{-1cm}
\renewcommand{\arraystretch}{1.4}
\begin{tabular}{|p{1.7cm}|p{4.7cm}|p{4.2cm}|p{4.5cm}|}
\hline
\textbf{Aplikacja} & \textbf{Najważniejsze funkcje} & \textbf{Personalizacja} & \textbf{Główne ograniczenia} \\ \hline

\textbf{Wim Hof \linebreak Method} &
Treningi oddechowe, \newline 
ekspozycja na zimno, \newline 
medytacje, statystyki, \newline 
odznaki, społeczność. &
Zmiana tempa oddechu, \newline
liczby rund, oddechów, \newline
tła dźwiękowego. \newline
Brak tworzenia własnych \newline
treningów. &
Wiele funkcji tylko w wersji \newline
Premium; \newline
ograniczona elastyczność \newline
konfiguracji.\\ 
\hline
\textbf{Breathwrk} &
Setki ćwiczeń, \newline 
zajęcia z instruktorami, \newline 
animacje, tryby immersyjne, \newline 
przypomnienia, statystyki. &
Wybór animacji, dźwięków,\newline
czasu trwania treningu, \newline 
podstawowe ustawienia \newline  
kroków (ograniczone). &
Brak możliwości tworzenia \newline 
własnych treningów; \newline 
ograniczenia czasu \newline 
trwania ćwiczeń; \newline 
wiele treści wyłącznie\newline 
w planie Premium. \\ 
\hline
\textbf{STAmina \linebreak Apnea \linebreak Trainer} &
Tablice CO$_2$/O$_2$, \newline 
bezdech statyczny, \newline 
śledzenie rekordów, \newline 
integracja z Apple Health, \newline 
analiza SpO$_2$, \newline 
głosowe instrukcje. &
Pełne tworzenie treningów; \newline  
regulacja czasów trwania \newline  
każdej fazy; \newline 
generowanie sesji \newline 
na podstawie rekordów. &
Starszy interfejs; \newline 
brak samouczka; \newline 
wyższy próg wejścia \newline 
dla początkujących. \\ 
\hline
\textbf{Breathe \linebreak (Havabee)} &
Schematy oddechowe, \newline
tworzenie własnych wzorców, \newline
statystyki sesji, \newline
intuicyjny interfejs. &
Możliwość definiowania faz \newline
oddechu i liczby cykli; \newline
tworzenie własnych \newline
treningów. &
Brak zatrzymania \newline
treningu; \newline
brak złożonych sekwencji \newline
oddechowych; \newline
ograniczenia minimalnych \newline
czasów faz; \newline
mniej czytelne wizualizacje postępu. \\ 
\hline
\end{tabular}
\caption{Porównanie funkcjonalności analizowanych aplikacji oddechowych}
\label{tab:podsumowanie_aplikacji}
\end{table}
