\chapter{Analiza wymagań}

W niniejszym rozdziale przedstawione zostały wymagania projektowanej aplikacji. Użyto w tym celu trzech modeli - przypadków użycia, klas oraz sterowania dźwiękami.

\section{Model przypadków użycia \Hania\ \Ola}

cel po co robimy: określić główne funkcje systemu, aktorów oraz relacje między nim
zakres: wszystkie funkcje dostępne dla użytkownika końcowego
Podzieliliśmy na 3 diagramy
diagram 1 został przedstawiony na rysunku \ref{img/przypadki_uzycia_1}

\begin{figure}[ht]
\centering
\includegraphics[scale=0.37]{obrazki/analiza_wymagan/przypadki_uzycia_1.png}
\caption{Główne przypadki użycia dla aplikacji ReSpire}
\label{img/przypadki_uzycia_1}
\end{figure}

opis diagramu 1, opis przypadków użycia
diagram 2 został przedstawiony na rysunku \ref{img/przypadki_uzycia_2}

\begin{figure}[ht]
\centering
\includegraphics[scale=0.42]{obrazki/analiza_wymagan/przypadki_uzycia_2.png}
\caption{Przypadki użycia dla aplikacji ReSpire - edycja treningu}
\label{img/przypadki_uzycia_2}
\end{figure}

opis diagramu 2, opis przypadków użycia
diagram 3 został przedstawiony na rysunku \ref{img/przypadki_uzycia_3}

\begin{figure}[ht]
\centering
\includegraphics[scale=0.38]{obrazki/analiza_wymagan/przypadki_uzycia_3.png}
\caption{Przypadki użycia dla aplikacji ReSpire - odtwarzanie treningu}
\label{img/przypadki_uzycia_3}
\end{figure}

opis diagramu 3, opis przypadków użycia

\section{Modele klas \Jakub}
\subsection{Model serwisów}
\subsection{Model treningu}

\section{Model sterowania dźwiękami \Hania\ \Ola}

oś czasu z zaznaczonymi zdarzeniami
cel: żeby pokazać logikę ich odtwarzania w czasie
opis diagramu i logiki
diagram przedtawiono na rysunku ref{}
<diagram>

\section{\textcolor{red}{Ewentualnie inne}}