\chapter{Testowanie aplikacji i wnioski}
\todoHania{SIEROTY!!!!}
W tym rozdziale przedstawione zostały sposoby testowania tworzonej aplikacji mobilnej. Terminologia i podział technik testowych są zgodne z aktualnym syllabusem ISTQB (\textit{ang. International Software Testing Qualifications Board}) Foundation Level (wersja 4.0) oraz polskimi odpowiednikami zawartymi w Glosariuszu SJSI. Ze względu na mały zespół i charakter projektu większość czynności miała charakter manualny i opierała się na wzajemnej weryfikacji członków zespołu i opiekuna pracy. Opis przeprowadzonego testowania podzielono na dwie części - testowanie statyczne i testowanie dynamiczne.

\section{Testowanie statyczne}
Testowanie statyczne stanowiło jeden z kluczowych elementów zapewnienia jakości w niniejszym projekcie. W odróżnieniu od testowania dynamicznego, wszystkie czynności statyczne realizowano bez konieczności uruchamiania kodu wykonawczego. Zamiast tego analizie poddane były artefakty wytwarzane na każdym etapie, w tym: kod źródłowy, modele i diagramy oraz wymagania. Dzięki temu możliwe było wczesne wykrycie błędów, niezgodności, problemów z jakością czy naruszenia standardów. 

Główną formę testowania statycznego stosowanego w projekcie stanowiły przeglądy kodu (\textit{ang. Code Review}). Każdy fragment kodu, niezależnie od tego, czy była to nowa funkcjonalność czy drobna poprawka, musiał przejść nieformalny przegląd przez co najmniej jedną, a najczęściej dwie inne osoby z zespołu. Czynności te odbywały się na bieżąco po każdym przyroście przed dołączeniem (\textit{ang. merge}) do głównej gałęzi roboczej \textit{dev}. Ich celem było wykrycie błędów przed połączeniem z już działającym kodem.

Często odbywały się również przeglądy techniczne, których celem było podjęcie decyzji w sprawie problemu technicznego. W spotkaniach tego typu brał udział cały zespół, co skutkowało większą liczbą wykreowanych pomysłów i prowadziło do osiągnięcia konsensusu. Zdarzało się jednak, iż przegląd odbywał się w formie przeglądu parami (\textit{ang. pair reviewing}), gdy problem wymagał osób dysponujących specyficznymi kwalifikacjami technicznymi lub problem nie wymagał obecności wszystkich członków zespołu.

Wszystkie inne artefakty, takie jak wymagania, modele czy diagramy były wielokrotnie omawiane i poprawiane w trakcie wspólnych sesji. Często dochodziło do kilku iteracji, aż cały zespół w pełni rozumiał i akceptował przedstawione rozwiązanie. Szczególną uwagę zwracano na spójność, brak sprzeczności oraz jednoznaczność opisów.

Dzięki małemu zespołowi i kulturze otwartej komunikacji każdy mógł w dowolnym momencie zweryfikować kod innego członka zespołu i zadać pytanie, zasugerować zmiany lub wskazać potencjalny problem. Ta nieformalna, ale niezwykle częsta sytuacja była jednym z najskuteczniejszych mechanizmów wczesnego wykrywania defektów.