\chapter{Podsumowanie \Karol}

\section{Osiągnięte rezultaty}
W trakcie trwania naszego inżynierskiego projektu dyplomowego udało nam sie zrealizować w pełni funkcjonalną aplikację umożliwiającą kompleksową konfigurację skomplikowanych treningów oddechowych. Aplikacja umożliwia importowanie gotowych treningów lub grup treningów, tworzenie treningów, elastyczną konfigurację ich struktury oraz wielowarstwową edycję audio dla każdego elementu treningu. Następnie nasze rozwiązanie umożliwia odtworzenie treningu z wykorzystaniem wszystkich skofngiurowanych wskazówek audiowizualnych lub eksport jednego lub wybranej przez użytkownika grupy treningów do pliku JSON.

\section{Napotkane problemy i ogarniczenia}
Niestety, z powodu braku dostarczenia nam modelu lub serwisu z modelem do ewaluacji treningu przez nagrania audio użytkownika oraz decyzji o skupieniu się na rozbudowie konfigurowalności, nie udało nam się zaimplementować sprzężenia zwrotnego określającego jakość wykonania przez użytkownika treningu.

\section{Wnioski na przyszłość}

\section{Porównanie opracowanego rozwiązania z istniejącymi rozwiązaniami}
Wśród ograniczeń zidentyfikowanych w istniejących rozwiązaniach (rozdział \ref{sec:istniejace_rozwiazania}) wskazywano utrudnioną edycję i tworzenie treningów, nawet w aplikacjach deklarujących pełną personalizację. Część z nich nie umożliwiała definiowania długości trwania faz oddechowych, inne narzucały sztywne szablony treningów, nie umożliwiając ich edycji lub ograniczały funkcjonalność użytkownikom bez subskrypcji. Jako mankamenty rozwiązań wymieniano również przestarzały lub nieintuicyjny interfejs, problemy z poprawnym odtwarzaniem dźwięków oraz brak możliwości zatrzymania treningu.

ReSpire oferuje nowoczesny i czytelny interfejs, obsługuje wstrzymywanie sesji oraz eliminuje główny problem związany z ograniczoną elastycznością. Treningi są w pełni konfigurowalne przy pomocy zaawansowanego edytora, który umożliwia dobór typów i długości faz oddechowych oraz ich dowolne zestawianie. Mimo to aplikacja nadaje się zarówno dla osób doświadczonych, jak i początkujących. Zaoferowane gotowe treningi pozwalają na połączenie potrzeb obydwu grup bez konieczności rezygnowania z możliwości konfiguracji. Elementem wyróżniającym ReSpire jest również rozbudowana oprawa dźwiękowa. Użytkownik może wybierać spośród różnych poziomów szczegółowości, ustawiając dźwięki od faz w poszczególnych etapach po muzykę w tle treningu.

Opracowana aplikacja rozwiązuje problemy zauważone w istniejących rozwiązaniach. Choć oferuje przestrzeń do dalszego rozwoju, stanowi przykład podejścia innowacyjnego --- wprowadzającego nowe, bardziej elastyczne możliwości.

\section{Możliwości rozwoju}
Aplikacja jest obecnie w pełni funkcjonalna i stabilna, natomiast posiada perspektywy dalszego rozwoju. Jedną z nich jest integracja z modelem oceniającym w czasie rzeczywistym na podstawie nagrań audio dokładności wykonywanych treningów i wyświetlanie ich w formie przyjaznych statystyk. System można by także wzbogacić o mechanizm powiadomień, aby użytkownik pamiętał o regularnych treningach oddechowych, a także wprowadzić system gratyfikacji motywujący do używania ReSpire. Kolejnym potencjalnym obszarem rozwoju jest wprowadzenie innych wersji językowych --- obecnie aplikacja obsługuje jedynie polski oraz angielski. Pomimo tego, że aplikacja posiada silną i spójną identyfikację wizualną, warto rozważyć także dodanie motywu ciemnego lub innych motywów kolorystycznych. Etapem pozwalającym udostępnić aplikację szerszemu gronu użytkowników byłoby wprowadzenie jej do sklepu \textit{GooglePlay} i/lub \textit{Apple App Store}, co z~kolei stanowiłoby podstawę do dodania funkcjonalności społeczności do ReSpire --- na przykład udostępniania treningów między użytkownikami bezpośrednio przez platformę.
