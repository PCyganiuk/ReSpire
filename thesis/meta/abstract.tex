\chapter*{Abstract}

\noindent \textbf{Context:} In light of the increasing popularity of breathing techniques in sport, stress reduction, and pulmonary rehabilitation, the need to develop a highly configurable tool to facilitate users' practice has been identified. An analysis of the state of the art and existing market solutions revealed that currently available tools often impose rigid exercise patterns, thereby limiting the scope for individual customisation of training to meet user needs.


\vspace{0.5cm}
\noindent \textbf{Objective:} The primary aim of this engineering thesis is to design and implement a cross-platform mobile application capable of facilitating highly configurable breathing exercises. The scope of work included requirements analysis, architecture and interface design, and software implementation.

\vspace{0.5cm}
\noindent \textbf{Methods:} The application was developed using the Flutter framework and the Dart language, which enabled support for both Android and iOS systems via a single codebase. A modular software architecture was employed alongside Hive, a local non-relational database, to ensure data persistence. The design phase was preceded by a comparative analysis of existing solutions available on the market.


\vspace{0.5cm}
\noindent \textbf{Results:} The project resulted in the creation of 'ReSpire', an application featuring a comprehensive editor that allows for the construction of multi-stage training sequences. An audiovisual user guidance system was implemented, integrating animated visualisations of phase progression with a multi-layered audio track that supports background music, voice instructions, and a binaural beats generator.


\vspace{0.5cm}
\noindent \textbf{Conclusions:} The developed solution effectively combines the ease of use for beginners with the flexibility required by advanced users, successfully overcoming the limitations of competing products identified during the analysis stage.

\vspace{1.5cm}
\noindent \textbf{Keywords:} breathing training, digital health, flexibility, configurability, Flutter, mobile application, breathing phases, audio setting, training personalization
\vspace{0.5cm}
