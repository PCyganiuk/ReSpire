\chapter{Założenia projektowe \Karol}

\section{Dla kogo rozwiązanie jest przeznaczone?  \Karol}

\section{Jakie korzyści ma dostarczać? \Karol}
\textbf{//na razie z RPI}

ReSpire to aplikacja mobilna wspierająca trening oddechowy. Umożliwia użytkownikom projektowanie i zarządzanie spersonalizowanymi sesjami treningu oddechowego. Dzięki szerokim możliwościom personalizacji, ReSpire pozwala na tworzenie unikalnych planów treningowych i ich realizację w intuicyjny sposób. 
Aplikacja wychodzi naprzeciw potrzebie złożonej konfiguracji ćwiczeń oddechowych. Służy ona jak najlepszemu dopasowaniu treningów do użytkowników i ich potrzeb, zapewniając dużą elastyczność i swobodę. To wyróżnia ją od innych aplikacji dostępnych na rynku, które nie pozwalają na dostosowywanie w tak znacznym stopniu. Jednocześnie zaspokaja również potrzeby osób, których nie interesuje personalizacja i wolą skorzystać       z gotowych, prostszych treningów oddechowych.

Celem projektu jest stworzenie tworzenie intuicyjnej i funkcjonalnej aplikacji mobilnej, umożliwiającej użytkownikom skuteczny trening oddechowy, która ma pomagać w poprawie kontroli oddechu, redukcji stresu oraz zwiększeniu świadomości oddechowej. Aplikacja zostanie opublikowana w Google Play i App Store, aby dotrzeć do jak najszerszego grona użytkowników w Polsce. Projekt stanowi część pracy inżynierskiej.

Zakres projektu obejmuje pełny cykl tworzenia aplikacji mobilnej ReSpire, od fazy projektowania po wdrożenie i publikację. 

W ramach prac zostanie zaprojektowany i zaimplementowany intuicyjny interfejs użytkownika, który umożliwi łatwe zarządzanie sesjami treningu oddechowego. Kluczowym aspektem aplikacji będzie szeroka personalizacja poprzez wybór parametrów takich jak typ kroku (wdech, wydech, przerwa), długość kroku, inkrementacja czy powtórzeń. Projekt zakłada również opracowanie mechanizmów interaktywnych, takich jak wizualizacje wspomagające kontrolę oddechu.
Zwieńczeniem projektu będzie publikacja aplikacji w sklepach Google Play i App Store, co umożliwi szeroką dostępność wśród użytkowników w Polsce, jak i za granicą.

\section{Jakie mają być główne funkcjonalności? \Karol}

\section{Jacy użytkownicy są przewidziani? \Karol}

\section{Czy i ew. z jakimi systemami ma współpracować? \Karol}

\section{Pozostałe założenia \Karol}