\chapter{Uwagi od strony technicznej}

\section{Opcje}

\begin{enumerate}
 \item \verb|strict| -- domyślnie, klasa stara się jak najściślej wypełniać zalecenia
 \item \verb|nostrict| -- drobne modyfikacje typograficzne
 \begin{itemize}
  \item zmniejszenie wcięcia akapitowego z 1.25cm na 1.5em
 \end{itemize}
\end{enumerate}

\section{Wymagane pakiety}
Lista pakietów, które są wymagane do kompilacji (większość nich jest zapewne zainstalowana
domyślnie)
\begin{enumerate}
  \item \verb|polski| -- polonizacja \TeX'a
  \item \verb|fontenc| -- kodowanie znaków
  \item \verb|inputenc| -- kodowanie znaków
  \item \verb|helvet| -- wybiera font podobny do Arial
  \item \verb|geometry| -- ustawienie marginesów
  \item \verb|indentfirst| -- wcięcie pierwszego akapitu
  \item \verb|fancyhdr| -- paginacja
  \item \verb|titlesec| -- tytularia
  \item \verb|titletoc| -- formatowanie spisu treści
  \item \verb|enumitem| -- wyliczenia numerowane i nienumerowane
  \item \verb|amsmath,amssymb,amsthm| -- standardowe pakiety matematyczne
  \item \verb|graphicx| -- dołączanie obrazków
  \item \verb|subfig| -- wiele obrazków na jednym rysunku
  \item \verb|caption| -- format podpisu pod obrazkiem
  \item \verb|tikz| -- generowanie schematów blokowych i innych rysunków
  \item \verb|listings| -- umieszczanie listingów kodu w pracy
\end{enumerate}

\section{Fonty}

Wymaganym fontem jest Arial. Ponieważ taki font nie jest łatwo dostępny w \LaTeX'u więc korzystamy
z fonta zastępczego w pakiecie \verb|helv|. Wymagany font matematyczny nie został podany. Używamy
zatem fontu z pakietu \verb|mathpazo|.
\begin{verbatim}
\usepackage{helvet}
\usepackage{mathpazo}
\renewcommand{\familydefault}{\sfdefault}
\end{verbatim}

Inną wersje fontu bezszeryfowego można uzyskać poprzez zrezygnowanie z pakietu \verb|helv|.

Przykładowy: $\sin(x)+ay^2$.
\[
 \sin(x)+ay^2
\]
