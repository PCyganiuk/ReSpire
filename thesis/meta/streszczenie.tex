\chapter*{Streszczenie}

\noindent \textbf{Kontekst:} W obliczu rosnącej popularności technik oddechowych w sporcie, redukcji stresu oraz rehabilitacji pulmonologicznej, zidentyfikowano potrzebę stworzenia wysoce konfigurowalnego narzędzia wspierającego w ich praktykowaniu. Analiza aktualnego stanu wiedzy oraz istniejących rozwiązań pokazała, że obecnie dostępne rozwiązania często narzucają sztywne schematy ćwiczeń, ograniczając możliwość indywidualnego dostosowania treningu do potrzeb użytkownika. 

\vspace{0.5cm}
\noindent \textbf{Cel:} Głównym celem pracy inżynierskiej jest opracowanie rozwiązania oraz implementacja wieloplatformowej mobilnej aplikacji umożliwiającej przeprowadzanie treningu oddechowego umożliwiającej przeprowadzenie wysoce konfigurowalnych treningów oddechowych. Zakres prac obejmował analizę wymagań, projekt architektury i interfejsu oraz implementację oprogramowania.

\vspace{0.5cm}
\noindent \textbf{Metody:} Aplikacja została zrealizowana w oparciu o platformę Flutter w języku Dart, co umożliwiło wsparcie dla zarówno systemu Android jak i iOS przy wykorzystaniu wspólnego kodu źródłowego. Zastosowano modularną architekturę oprogramowania oraz lokalną, nierelacyjną bazę danych Hive do zapewnienia trwałości danych. Projektowanie poprzedzono analizą porównawczą istniejących rozwiązań na rynku.

\vspace{0.5cm}
\noindent \textbf{Wyniki:} W ramach prac stworzono aplikację “ReSpire”, wyposażoną w kompleksowy edytor umożliwiający tworzenie wieloetapowych sekwencji treningowych. Zaimplementowano audiowizualny system przeprowadzania użytkownika, integrujący animowane wizualizacje przebiegu faz z wielowarstwową ścieżką dźwiękową — obsługującą tło muzyczne, wskazówki głosowe oraz generator dudnień różnicowych.

\vspace{0.5cm}
\noindent \textbf{Wnioski:} Opracowane rozwiązanie skutecznie łączy prostotę obsługi dla początkujących z elastycznością wymaganą przez użytkowników zaawansowanych, eliminując dostrzeżone na etapie analizy ograniczenia konkurencyjnych produktów. 

\vspace{1.5cm}
\noindent \textbf{Słowa kluczowe:} trening oddechowy, zdrowie cyfrowe, elastyczność, konfigurowalność, Flutter, aplikacja mobilna, fazy oddechu, oprawa dźwiękowa, personalizacja treningu
\vspace{0.5cm}

\noindent \textbf{Dziedzina nauki i techniki, zgodnie z wymogami OECD:} Nauki inżynieryjne i techniczne, inżynieria informatyczna
