\chapter{Projekt rozwiązania}

\section{Projekt architektury systemu \Hania\ \Ola}


\section{Projekt logiki aplikacji \Hania\ \Ola}

\section{Projekt interfejsu użytkownika \Hania\ \Ola}
\subsection{Projekt w Figma}\label{subsec:Figma}
\todoHania{tu bym opisała tylko figme}
Wstępny projekt interfejsu użytkownika został zaprojektowany w Figmie.

W narzędziu Canva zostało zaprojektowane logo aplikacji - połączenie dwóch różnych czcionek oraz dodatek grafiki podmuchu i detalu symbolizującego cząsteczkę powietrza nad literą “i” oddaje ducha ReSpire.

W oparciu o kolorystykę elementów w logo zostały wybrane 3 główne kolory aplikacji - są to odcienie niebieskiego. Kolor ten jest kojarzony z poczuciem spokoju, harmonii, stabilizacji i bezpieczeństwa.( tu cytat). Dzięki temu aplikacja jest spójna kolorystycznie, a także wywołuje przyjemne odczucia w użytkowniku, zapewniając spokój i harmonię. 

Jedna z czcionek użytych w logo  - “Glacial Indifference” została również wykorzystana w UI aplikacji, zapewniając jej unikatowy charakter.
\todoOla{Ogolnie gdzieś możnaby podać źródła do animacji może i napisać, że wszystkie uzyte materiały są legalne}
Białe kafelki na stronie głównej z asymetrycznymi zaokrągleniami oraz symbolami podmuchu stanowią prosty, ale modernistyczny element aplikacji. Przycisk dodawania własnego treningu nawiązuje do detalu w logo, czyniąc aplikację wyjątkową.

\begin{figure}[H]
\centering
\includegraphics[scale=0.4]{obrazki/projekt_rozwiazania/figma_home.png}
\caption{Pierwszy projekt aplikacji ReSpire - strona główna}
\label{img/respire_home}
\end{figure}

\begin{figure}[H]
\centering
\includegraphics[scale=0.45]{obrazki/projekt_rozwiazania/figma_training.png}
\caption{Pierwszy projekt aplikacji ReSpire - przebieg treningu}
\label{img/respire_training}
\end{figure}

\begin{figure}[H]
\centering
\includegraphics[scale=0.44]{obrazki/projekt_rozwiazania/figma_settings.png}
\caption{Pierwszy projekt aplikacji ReSpire - edycja treningu}
\label{img/respire_settings}
\end{figure}

\newpage
\subsection{Spójność aplikacji}
\todoHania{Tu bym dała kolory, logo i ewentualnie coś o czcionkach, może o użytych ikonach z jakiej biblioteki}
\todoOla{Oki}
Kolory użyte w aplikacji
\begin{itemize}
    \item Hex: \#1A93A8, RGB: 26, 147, 168
    \item Hex: \#32B7CF, RGB: 50, 183, 207
    \item Hex: \#7BDEF0, RGB: 123, 222, 240
\end{itemize}

Ikony chyba z biblioteki "material"

\begin{figure}[H]
    \centering
    \begin{minipage}{0.48\textwidth}
        \centering
        \includegraphics[width=\textwidth]{obrazki/projekt_rozwiazania/logo_biale_tlo.png}
        \caption{Logo ReSpire}
        \label{img/respire_logo}
    \end{minipage}
    \hfill
    \begin{minipage}{0.48\textwidth}
        \centering
        \includegraphics[width=\textwidth]{obrazki/projekt_rozwiazania/respire_colors.png}
        \caption{Kolory użyte w aplikacji}
        \label{img/respire_colors}
    \end{minipage}
\end{figure}



\section{Projekt struktury danych (treningi) \Jakub}

Model danych oparto na strukturze hierarchicznej, której fundament stanowią trzy kluczowe klasy: \texttt{Trening}, \texttt{EtapTreningu} oraz \texttt{FazaOddechowa}. Główna encja, \texttt{Trening}, składa się z co najmniej jednego \texttt{EtapuTreningu} (relacja jeden do wielu). Każdy etap charakteryzuje się zdefiniowaną liczbą cykli (powtórzeń) oraz parametrem przyrostu czasu. Na najniższym poziomie hierarchii znajduje się \texttt{FazaOddechowa}. Etapy składają się z sekwencji faz o określonym czasie trwania oraz typie, wybieranym ze zbioru zdefiniowanych wartości: wdech, wydech, regeneracja lub~wstrzymanie. Opisywany projekt przedstawia rysunek~\ref{img/training_structure}, natomiast rozwinięcie i implementację tej struktury danych opisano w sekcji \ref{subsec:training_classes}.

\begin{figure}[H]
    \centering
    \includegraphics[width=\textwidth]{obrazki/projekt_rozwiazania/Struktura_treningu.jpg}
    \caption{Struktura treningu}
    \label{img/training_structure}
\end{figure}