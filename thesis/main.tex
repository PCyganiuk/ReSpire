\documentclass[nostrict]{szablonPG}

%-------------------- Dodatkowe pakiety ---------------------
\usepackage{listing}
\renewcommand{\lstlistingname}{Kod}
\usepackage[x11names]{xcolor} % colorful text with the most amount of colors enabled (x11names)
\usepackage[backend=biber,
            style=ieee,
            language=polish,
            giveninits=true,
            sorting=none,
            url=true,
            doi=false,
            isbn=false]{biblatex} % pakiet pozwalający na konfigurację bibliografi   
\addbibresource{refs.bib}
%------------------------------------------------------------
%------------------Definicja własnych komend-----------------
\DeclareRobustCommand{\Hania}[0]{\textcolor{SlateBlue1}{Hania}}
\DeclareRobustCommand{\Jakub}[0]{\textcolor{green}{Jakub}}
\DeclareRobustCommand{\Karol}[0]{\textcolor{Tan1}{Karol}}
\DeclareRobustCommand{\Ola}[0]{\textcolor{Maroon0}{Ola}}

\newcommand{\todoHania}[1]{\todo[color=SlateBlue1!35]{Hania: #1}}
\newcommand{\todoKarol}[1]{\todo[color=Tan1!50]{Karol: #1}}
\newcommand{\todoJakub}[1]{\todo[color=green!50]{Jakub: #1}}
\newcommand{\todoOla}[1]{\todo[color=Maroon1!50]{Ola: #1}}
%------------------------------------------------------------
%------------------------------------------------------------
%			      Początek pracy dyplomowej  
%------------------------------------------------------------

\begin{document}

%------------------------------------------------------------
%  Dodanie strony tytułowej wygenerowanej z MojaPG oraz oświadczenia

\includepdf[pages=-]{meta/StronaTytulowa_193078.pdf}
%\includepdf{meta/oswiadczenie.pdf}

%------------------------------------------------------------
%  Dodanie streszczenia i abstract
%  						

%\chapter*{Streszczenie}

\noindent \textbf{Kontekst:} W obliczu rosnącej popularności technik oddechowych w sporcie, redukcji stresu oraz rehabilitacji pulmonologicznej, zidentyfikowano potrzebę stworzenia wysoce konfigurowalnego narzędzia wspierającego w ich praktykowaniu. Analiza aktualnego stanu wiedzy oraz istniejących rozwiązań pokazała, że obecnie dostępne rozwiązania często narzucają sztywne schematy ćwiczeń, ograniczając możliwość indywidualnego dostosowania treningu do potrzeb użytkownika. 

\vspace{0.5cm}
\noindent \textbf{Cel:} Głównym celem pracy inżynierskiej jest opracowanie rozwiązania oraz implementacja wieloplatformowej mobilnej aplikacji umożliwiającej przeprowadzanie treningu oddechowego umożliwiającej przeprowadzenie wysoce konfigurowalnych treningów oddechowych. Zakres prac obejmował analizę wymagań, projekt architektury i interfejsu oraz implementację oprogramowania.

\vspace{0.5cm}
\noindent \textbf{Metody:} Aplikacja została zrealizowana w oparciu o platformę Flutter w języku Dart, co umożliwiło wsparcie dla zarówno systemu Android jak i iOS przy wykorzystaniu wspólnego kodu źródłowego. Zastosowano modularną architekturę oprogramowania oraz lokalną, nierelacyjną bazę danych Hive do zapewnienia trwałości danych. Projektowanie poprzedzono analizą porównawczą istniejących rozwiązań na rynku.

\vspace{0.5cm}
\noindent \textbf{Wyniki:} W ramach prac stworzono aplikację “ReSpire”, wyposażoną w kompleksowy edytor umożliwiający tworzenie wieloetapowych sekwencji treningowych. Zaimplementowano audiowizualny system przeprowadzania użytkownika, integrujący animowane wizualizacje przebiegu faz z wielowarstwową ścieżką dźwiękową — obsługującą tło muzyczne, wskazówki głosowe oraz generator dudnień różnicowych.

\vspace{0.5cm}
\noindent \textbf{Wnioski:} Opracowane rozwiązanie skutecznie łączy prostotę obsługi dla początkujących z elastycznością wymaganą przez użytkowników zaawansowanych, eliminując dostrzeżone na etapie analizy ograniczenia konkurencyjnych produktów. 

\vspace{1.5cm}
\noindent \textbf{Słowa kluczowe:} trening oddechowy, zdrowie cyfrowe, elastyczność, konfigurowalność, Flutter, aplikacja mobilna, fazy oddechu, oprawa dźwiękowa, personalizacja treningu
\vspace{0.5cm}

\noindent \textbf{Dziedzina nauki i techniki, zgodnie z wymogami OECD:} Nauki inżynieryjne i techniczne, inżynieria informatyczna

%\chapter*{Abstract}

\noindent \textbf{Context:} In light of the increasing popularity of breathing techniques in sport, stress reduction, and pulmonary rehabilitation, the need to develop a highly configurable tool to facilitate users' practice has been identified. An analysis of the state of the art and existing market solutions revealed that currently available tools often impose rigid exercise patterns, thereby limiting the scope for individual customisation of training to meet user needs.


\vspace{0.5cm}
\noindent \textbf{Objective:} The primary aim of this engineering thesis is to design and implement a cross-platform mobile application capable of facilitating highly configurable breathing exercises. The scope of work included requirements analysis, architecture and interface design, and software implementation.

\vspace{0.5cm}
\noindent \textbf{Methods:} The application was developed using the Flutter framework and the Dart language, which enabled support for both Android and iOS systems via a single codebase. A modular software architecture was employed alongside Hive, a local non-relational database, to ensure data persistence. The design phase was preceded by a comparative analysis of existing solutions available on the market.


\vspace{0.5cm}
\noindent \textbf{Results:} The project resulted in the creation of 'ReSpire', an application featuring a comprehensive editor that allows for the construction of multi-stage training sequences. An audiovisual user guidance system was implemented, integrating animated visualisations of phase progression with a multi-layered audio track that supports background music, voice instructions, and a binaural beats generator.


\vspace{0.5cm}
\noindent \textbf{Conclusions:} The developed solution effectively combines the ease of use for beginners with the flexibility required by advanced users, successfully overcoming the limitations of competing products identified during the analysis stage.

\vspace{1.5cm}
\noindent \textbf{Keywords:} breathing training, digital health, flexibility, configurability, Flutter, mobile application, breathing phases, audio setting, training personalization
\vspace{0.5cm}


%------------------------------------------------------------
%   Lista komentarzy
\listoftodos

%------------------------------------------------------------
%	Utworzenie spisu treści pracy dyplomowej
\tableofcontents

%------------------------------------------------------------
%	Dodanie wykazu wazniejszych skrótów i oznaczeñ 
\input{meta/oznaczenia.tex}

%------------------------------------------------------------
%	Dodanie rozdziałów pracy dyplomowej 

\chapter{Wstęp i cel pracy} 

\section{Wprowadzenie (problemy) \Hania}
W ostatnich latach obserwuje się wzrost zainteresowania technikami wspierającymi zdrowie psychiczne oraz treningi fizjologiczne.Ważną rolę w tym obszarze odgrywają ćwiczenia oddechowe wykorzystywane w sporcie, terapii i codziennym dbaniu o dobre samopoczucie. Aplikacje mobilne pełnią w tym zakresie istotną rolę, sprzyjając ich popularności. Umożliwiają użytkownikom stały dostęp do zindywidualizowanych programów treningowych z dowolnego miejsca i w dowolnym czasie, pozwalając użytkownikowi dopasować przebieg ćwiczenia do własnych preferencji i celów.

Niniejsza praca koncentruje się na zaprojektowaniu aplikacji mobilnej, która umożliwia wykonywanie treningów oddechowych z wykorzystaniem konfigurowalnych wzorców oddechowych i czytelnej prezentacji ich przebiegu.

\section{Motywacje (korzyści) \Hania}
\todoHania{dodać biblo}
Motywacją do realizacji projektu jest stworzenie narzędzia ułatwiającego wykonywanie ćwiczeń oddechowych, które mogą wspierać zarówno zdrowie psychiczne, jak i fizyczne. Regularne treningi oddechowe są stosowane w różnych grupach użytkowników. Może on służyć:
\begin{itemize}
  \item sportowcom wykorzystującym je do poprawy wytrzymałości \cite{sport};
  \item pacjentom z chorobami układu oddechowego, takimi jak astma wykorzystującym je do poprawy stanu zdrowia \cite{asthma};
  \item osobom z zaburzeniami lękowymi lub doświadczającym przewlekłego stresu wykorzystującym je do ukojenia lęku i łagodzenie stresu \cite{stress}; 
  \item osobom zmagającym się z problemami ze snem wykorzystującym je do poprawy jakości snu \cite{sleep}; 
  \item każdej osobie chcącej pracować nad oddechem, która chce wesprzeć układ odpornościowy i zwiększyć ilość tlenu w organizmie.
\end{itemize}

Mobilny charakter aplikacji sprawia, że trening staje się bardziej dostępny. Użytkownik może wykonywać go samodzielnie, w dowolnym miejscu, w tempie dostosowanym do swoich potrzeb. To uzasadnia potrzebę stworzenia rozwiązania, które umożliwia wygodną konfigurację wzorców oddechowych, odpowiednie prowadzenie treningu oraz przejrzystą prezentację jego przebiegu.

\section{Cel pracy \Hania}
Celem pracy jest opracowanie projektu oraz implementacja wieloplatformowej aplikacji mobilnej służącej do przeprowadzania treningu oddechowego opartego na konfigurowalnych wzorcach oddechowych obejmujących wdech, retencję, wydech i regenerację, z możliwością dostosowania parametrów dźwiękowych oraz językowych (polski i angielski), a także z wbudowaną wizualizacją przebiegu treningu.

\section{Podział prac w zespole \Jakub}

Prace nad projektem realizowane były przy pełnym zaangażowaniu całego zespołu. Członkowie grupy dynamicznie reagowali na informacje zwrotne, sprawnie wdrażając sugestie opiekuna oraz nowe koncepcje funkcjonalne. Dzięki bieżącej eliminacji błędów i iteracyjnemu wprowadzaniu udoskonaleń zapewniono ciągłość procesu wytwórczego oraz stabilny rozwój systemu, a jasno określony zakres obowiązków przyczynił się do lepszej koordynacji zadań.

\subsection{Hanna Banasiak}
Hanna odpowiadała za projekt oraz implementację kluczowych komponentów służących do realizacji pobierania i odtwarzania sesji treningowych - \texttt{TrainingController} (\ref{subsec:TrainingController}) oraz \texttt{TrainingParser} (\ref{subsec:TrainingParser}). Opracowała również wstępny projekt interfejsu użytkownika w środowisku Figma (\ref{subsec:Figma}), który stanowił fundament dla dalszych prac deweloperskich. W warstwie widoku zaimplementowała moduły wizualizacji treningów (\ref{subsec:AnimatedCircle})\todoJakub{Dodać tu odniesienia do slidera (o ile będzie) lub zmienić moduły na moduł}, walidację danych wejściowych\todoJakub{odniesienie do zdjęcia jak pusty trening jest?}, a także stworzyła widok szczegółów treningu (\ref{sec:TrainingPage}), stanowiący element nawigacyjny między stroną główną a odtwarzaczem.

\subsection{Aleksandra Bujny}
Do zadań Aleksandry należało opracowanie kompletnej identyfikacji wizualnej systemu - zaprojektowanie logotypu (\ref{img/respire_logo}) oraz zdefiniowanie spójnego motywu graficznego aplikacji (\ref{img/respire_colors}), które stał się obowiązującym standardem dla wszystkich modułów aplikacji. W warstwie implementacyjnej przygotowała widok strony głównej (\ref{sec:HomePage}), a także wzbogaciła interfejs użytkownika o animacje, zwiększające dynamikę i interaktywność aplikacji. Dodatkowo odpowiadała za bieżące utrzymanie spójności i przejrzystości aplikacji - konsekwentnie usuwała błędy, korygowała niejednoznaczności wizualne i standaryzowała elementy interfejsu, co znacząco poprawiło czytelność oraz ogólną jakość systemu.

\subsection{Jakub Romanowski}
Zadania Jakuba związane były z zaprojektowaniem i implementacją całej warstwy obsługi dźwięku w aplikacji. Jego głównym zadaniem było stworzenie architektury silnika audio, opartej na komponentach \texttt{SoundManager}(\ref{subsec:SoundManager}), \texttt{SingleSoundManager}\todoJakub{Opisujemy to? - w sumie może warto, bo wykorzystujemy w SSM poole playerów, może to jest ciekawe} oraz \texttt{PlaylistManager}. Aby zapewnić płynność działania interfejsu i uniknąć blokowania głównego wątku aplikacji, zaimplementował on asynchroniczną obsługę odtwarzania multimediów. Ponadto zaprojektował kluczowe struktury danych (\ref{img/class_diagram}). Dodatkowo wdrożył mechanizm internacjonalizacji (obsługę wielu języków) oraz funkcjonalność pozwalającą użytkownikom na import własnych plików dźwiękowych. Dbał również o stosowanie optymalnych rozwiązań, ponowne wykorzystanie istniejących komponentów oraz utrzymanie wysokiej jakości i przejrzystości kodu. \todoJakub{GitHub?}

\subsection{Karol Zwierz}

Karol był odpowiedzialny za implementację mechanizmu modyfikacji treningów w sposób intuicyjny dla użytkowników, a także za funkcjonalność importu i eksportu treningów. W obszarze przetwarzania sygnałów opracował moduł generujący dudnienia różnicowe (binaural beats). Ponadto zajął się optymalizacją wydajności poprzez wdrożenie mechanizmu ładowania zasobów z~wyprzedzeniem przed uruchomieniem treningu. W warstwie interfejsu stworzył edytor list odtwarzania plików audio.

\section{Struktura rozdziałów \Hania}
W celu przejrzystego przedstawienia realizacji projektu i ułatwienia odbioru pracy, została ona podzielona na dziewięć rozdziałów opisanych poniżej.
	
W rozdziale pierwszym przedstawiono wprowadzenie do tematu pracy, przedstawiono problemy i motywacje podjęcia pracy, jej cel oraz podział obowiązków w zespole.
	
Rozdział drugi opisuje aktualny stan wiedzy, obejmujący istniejące rozwiązania aplikacji mobilnych wspierających trening oddechowy oraz aspekty medyczne i sportowe treningu oddechowego. Przedstawiono również opisy podsumowujący ważne elementy i brakujące funkcjonalności w tych rozwiązaniach.
	
W rozdziale trzecim określono główne założenia projektowe, w których przedstawiono grupę docelową, korzyści płynące z projektu oraz zakładane główne funkcjonalności aplikacji. Rozdział ten definiuje również przewidywanych użytkowników oraz podstawowe wymagania wobec systemu.
	
Rozdział czwarty zawiera analizę wymagań, w tym modele przypadków użycia, modele i diagramy klas oraz dźwięków wykorzystywanych w projekcie.

W rozdziale piątym przedstawiono projekt rozwiązania, obejmujący architekturę systemu, logikę aplikacji, interfejs użytkownika oraz strukturę danych.

Rozdział szósty opisuje implementację aplikacji, w tym strukturę plików i modułów oraz wybrane fragmenty kodu, obejmujące najważniejsze elementy projektu. Zwrócono w nim uwagę na zastosowane technologie i narzędzia programistyczne.

W rozdziale siódmym zamieszczono instrukcję użytkownika opisującą sposób korzystania z aplikacji.

Rozdział ósmy poświęcono testowaniu, obejmującemu testy jednostkowe, integracyji modułów, systemowe i użytkowe. Zaprezentowano w nim wyniki testów oraz ocenę poprawności działania aplikacji w różnych scenariuszach użycia.

Dziewiąty rozdział, zawiera podsumowanie pracy, omówienie osiągniętych rezultatów, napotkanych trudności oraz wnioski dotyczące projektu. Wskazano również możliwe kierunki przyszłych usprawnień i rozszerzeń funkcjonalnych systemu.


\chapter{Aktualny stan wiedzy (\Hania,\ \Ola,\ \Jakub,\ \Karol)}
Celem niniejszego rozdziału jest przegląd wybranych aplikacji mobilnych dedykowanych treningowi oddechowemu oraz przedstawienie naukowego uzasadnienia dla ich stosowania. W~kolejnych podrozdziałach opisano kontekst medyczny wykorzystania technologii mobilnych w rehabilitacji oddechowej, a następnie przeanalizowano indywidualne aplikacje, zwracając uwagę na~ich specyfikę funkcjonalną, zastosowane rozwiązania technologiczne, interfejs użytkownika oraz ograniczenia.

\section{Kontekst naukowy i skuteczność aplikacji mobilnych (\Karol)}
W dobie cyfryzacji opieki zdrowotnej, aplikacje mobilne stają się istotnym narzędziem wspierającym tradycyjne metody rehabilitacji. Przeglądy systematyczne wskazują, że samodzielna rehabilitacja pulmonologiczna wspierana przez aplikacje mobilne może być skuteczną alternatywą dla rehabilitacji stacjonarnej, szczególnie w przypadku pacjentów z przewlekłą obturacyjną chorobą płuc (POChP) \cite{MobileApp_Rehab}. Aplikacje te pozwalają na przezwyciężenie barier takich jak brak dostępu do specjalistycznych ośrodków, koszty czy problemy z transportem. Wykazano, że interwencje oparte na aplikacjach mogą prowadzić do poprawy wydolności fizycznej i łagodzenia objawów chorobowych, oferując jednocześnie spersonalizowane plany treningowe, co jest kluczowe dla utrzymania zaangażowania pacjenta.

\section{Istniejące rozwiązania wspierające trening oddechu (\Jakub)}\label{sec:istniejace_rozwiazania}
\begin{itemize}
    \item Wim Hof Method firmy WHM Services
    \item Breathwrk: Breathing Exercises firmy Breathwrk Inc.
    \item STAmina Apnea Trainer firmy Squarecrowd Apps SIA
    \item Breathe firmy Havabee
\end{itemize}

\section{Wim Hof Method firmy WHM Services (\Hania)}
Na szczególną uwagę wśród dostępnych rozwiązań zasługuje aplikacja Wim Hof Method \cite{WHM} stworzona przez firmę WHM Services. Wykorzystuje ona techniki oddechowe opracowane przez holenderskiego sportowca i trenera, Wima Hofa. Wyróżnia się ona kompleksowym podejściem, łącząc ćwiczenia z ekspozycją na zimno oraz medytacją. 

\subsection{Cel aplikacji}
Celem aplikacji jest udostępnienie użytkownikom dostępu do regularnych treningów oddechowych zgodnie z założeniami metody Wim Hofa. Została zaprojektowana z~myślą o~osobach szukających sposobów na poprawę zdrowia psychofizycznego, zwłaszcza poprzez codzienne ćwiczenia oddechowe i ekspozycję na zimno. Ze względu na zróżnicowany poziom wiedzy technicznej oraz wiedzy z zakresu zdrowia wśród użytkowników, kluczową cechą aplikacji jest prosty i intuicyjny interfejs oraz stopniowe, prowadzące użytkownika krok po kroku wdrożenie w proces ćwiczeń.

\subsection{Funkcjonalności}
Aplikacja podzielona jest na trzy główne zakładki: \textit{Results} (pol. Rezultaty), \textit{Home} (pol. Strona główna) oraz \textit{Community} (pol. Społeczność).

Celem zakładki \textit{Results} (pol. Rezultaty) jest przedstawienie wszystkich postępów użytkownika. Na podstronie \textit{Bar Chart} (pol. Wykres słupkowy) znajdują się statystyki dotyczące wykonanych \textit{Breathing sessions} (pol. sesji oddechowych), \textit{Cold showers} (pol. zimnych pryszniców) czy \textit{Ice baths} (pol. kąpieli w lodzie). Poniżej znajduje się także wykres słupkowy i podsumowanie ilościowe dotyczące konkretnego rodzaju ćwiczenia, które można wybrać z dostępnej listy. Aby zarejestrować zadanie, nie jest konieczne wykonywanie go wraz z aplikacją, można to zrobić, klikając w przycisk \textit{Log exercise} (pol. Zarejestruj ćwiczenie).

Na podstronie \textit{Calendar} (pol. Kalendarz) dostępny jest widok miesiąca z możliwością przeglądania szczegółów poszczególnych dni. Użytkownik może edytować lub usuwać zarejestrowane aktywności, jednak dodawanie nowych jest niedostępne. Istnieje także opcja dzielenia się wynikami ze społecznością.

Podstrona \textit{Badges} (pol. Odznaki) motywuje użytkownika do dalszej pracy poprzez system osiągnięć i nagród za regularność i stopień trudności ćwiczeń. Na rysunku \ref{img/whm_results} przedstawiono podstrony \textit{Bar Chart}, \textit{Calendar} i \textit{Badges} zakładki \textit{Results} aplikacji Wim Hof Method.

\begin{figure}[ht]
\centering
\includegraphics[scale=0.25]{obrazki/istniejace_rozwiazania/whm/whm_results.png}
\caption{Zakładka \textit{Results} (\textit{Bar Chart, Calendar, Badges}) w aplikacji Wim Hof Method}
\label{img/whm_results}
\end{figure}

Zakładka \textit{Home} (pol. Strona główna) oferuje użytkownikowi dostęp do różnych typów ćwiczeń: \textit{Breathing Exercises} (pol. Trening oddechowy), \textit{Cold Exposure} (pol. Ekspozycja na zimno) — obejmującej m.in. zimne prysznice i kąpiele w lodzie — oraz trening \textit{Power of the Mind} (pol. Siły umysłu), który w praktyce stanowi zestaw ćwiczeń fizycznych. Oprócz głównych modułów treningowych, użytkownik ma dostęp do treści dodatkowych, takich jak nagrania audio (np. internetowe publikacje Wima Hofa), sesje medytacyjne, materiały internetowe wyjaśniające zasady działania metody oraz różnego rodzaju wyzwania. Interfejs zakładki \textit{Home} (pol. Strona główna) przestawiono na rysunku \ref{img/whm_home}.

Zakładka z ćwiczeniami oddechowymi umożliwia użytkownikowi skorzystanie z gotowych zestawów treningowych oraz ich edycję. Po wybraniu konkretnego ćwiczenia dostępny jest szereg opcji personalizacji, takich jak: tempo oddechu, liczba rund, liczba oddechów przed wstrzymaniem, wybór podkładu muzycznego oraz wyłączenie głosu prowadzącego. 

Po uruchomieniu sesji użytkownik widzi animację w formie sześciokąta, który sygnalizuje pozostały czas do ukończenia etapu oddechu. Wyświetlane są także liczba rund i liczba oddechów. W dowolnym momencie istnieje możliwość przerwania ćwiczenia i powrotu do ekranu głównego — postęp zostaje mimo to zapisany jako ukończona sesja. Dodatkowo, podwójne kliknięcie w trakcie ćwiczenia pozwala na przejście do fazy wstrzymania oddechu. Proces treningu został przedstawiony na~rysunku \ref{img/whm_training}.

Zakładka \textit{Community} (pol. Społeczność) umożliwia śledzenie aktywności innych użytkowników aplikacji. Aby uzyskać dostęp do postępów konkretnej osoby, należy ją wyszukać i zaobserwować. Użytkownik może również udostępniać własne wyniki za pośrednictwem spersonalizowanego profilu, w którym można zamieścić takie informacje jak imię i nazwisko, zdjęcie profilowe oraz krótki opis. W profilu widoczne są również: lista obserwowanych i obserwujących użytkowników, zdobyte odznaki oraz liczba ukończonych treningów. Intrefejs zakładki \textit{Community} (pol. Społeczność) został przedstawiony na~rysunku \ref{img/whm_community}.

\begin{figure}[H]
    \centering
    \begin{minipage}{0.48\textwidth}
        \centering
        \includegraphics[scale=0.25]{obrazki/istniejace_rozwiazania/whm/whm_home.png}
        \caption{Zakładka \textit{Home} w aplikacji Wim Hof Method}
        \label{img/whm_home}
    \end{minipage}
    \hfill
    \begin{minipage}{0.48\textwidth}
        \centering
        \includegraphics[scale=0.25]{obrazki/istniejace_rozwiazania/whm/whm_community.png}
        \caption{Zakładka \textit{Community} w aplikacji Wim Hof Method}
        \label{img/whm_community}
    \end{minipage}
\end{figure}

\begin{figure}[H]
    \centering
    \includegraphics[scale=0.25]{obrazki/istniejace_rozwiazania/whm/whm_training.png}
    \caption{Trening oddechowy w aplikacji Wim Hof Method}
    \label{img/whm_training}
\end{figure}

\subsection{Ograniczenia}
Pomimo szerokiego zakresu funkcjonalności, aplikacja posiada również pewne ograniczenia, które z perspektywy użytkownika mogą mieć istotne znaczenie.

Pierwszym z nich jest dostęp do wielu funkcji wyłącznie w wersji płatnej \textit{Premium}. Ograniczenia te dotyczą m.in. wyboru rodzaju ćwiczeń, uczestnictwa w sesjach medytacyjnych oraz podejmowania wyzwań. Choć aplikacja jest dostępna i użyteczna także w wersji podstawowej, dla bardziej zaawansowanych użytkowników może nie oferować wystarczającego wsparcia rozwojowego. Warto jednak zauważyć, że model subskrypcyjny stanowi źródło przychodu.
\par
Drugim ograniczeniem jest brak możliwości tworzenia własnych, w pełni spersonalizowanych treningów. Aplikacja koncentruje się na ściśle określonych założeniach metody Wima Hofa, co ogranicza elastyczność i możliwość dostosowania ćwiczeń do indywidualnych potrzeb. Dostępna personalizacja ogranicza się głównie do zmiany długości ćwiczeń, liczby rund czy tempa oddychania. Dla bardziej zaawansowanych użytkowników może to stanowić istotne utrudnienie.


\section{Breathwrk: Breathing Exercises firmy Breathwrk Inc. (\Ola)}
Breathwrk \cite{Breathwrk} to aplikacja, która w 2022 roku zwyciężyła w kategorii \textit{Najlepsza aplikacja do rozwoju osobistego} w \textit{Google Google Play’s Best of 2022}. Jest reklamowana jako aplikacja numer jeden do ćwiczeń oddechowych. Posiada wysoką ocenę zarówno w \textit{Sklepie Play} --- 4.1, jak i~w~\textit{App Store} --- 4.8. Według twórców posiada około dwa miliony użytkowników.

\subsection{Cel aplikacji}
Aplikacja skupia się na pomocy użytkownikowi poprzez trening oddechowy w czterech głównych obszarach: śnie, stresie, koncentracji oraz energii. Oferuje setki ćwiczeń oraz zajęć, czyli sesji z doświadczonymi instruktorami, którzy przeprowadzają użytkownika  przez starannie dobraną serię ćwiczeń. Breathwrk ma także służyć jako pomoc w budowaniu nawyku regularnych treningów oddechowych, dzięki przypomnieniom według ustalonych harmonogramów. Elementem wyróżniającym są także codziennie dodawane, nowe zajęcia immersyjne oferujące ciekawe wrażenia wizualne. Interesującym dodatkiem są również wibracje telefonu podczas treningu, jednak ustawienie to nie jest dostępne na wszystkich urządzeniach. Możliwy jest dostęp offline do aplikacji, a jej językiem głównym jest angielski.

\subsection{Funkcjonalności}
Aplikacja składa się z czterech głównych zakładek dostępnych na pasku w~dole aplikacji: \textit{Now} (pol. Teraz), \textit{Explore} (pol. Odkrywaj), \textit{Schedule} (pol. Zaplanuj), \textit{Me} (pol. Ja).

W prawym, górnym rogu znajduje się okrągła ikona ze zdjęciem profilowym lub wybranym z~biblioteki zdjęciem zwierzęcia, liczbą oznaczającą poziom rankingowy oraz postęp w osiągnięciu kolejnego poziomu rankingowego. Poziom rankingowy można zwiększać poprzez wykonywanie ćwiczeń. Po kliknięciu na ikonę pokazuje się okno \textit{Shortcuts} (pol. Skróty) z dwoma zakładkami. W jednej z nich znajdują się treningi oznaczone jako ulubione, a w drugiej ostatnio wykonane treningi. Pokazuje się także ikona ołówka, po kliknięciu którego możliwa jest edycja zdjęcia profilu. Zakładki te przedstawione zostały na rysunku \ref{img/wrk_shortcuts}.

\begin{figure}[H]
    \centering
    \includegraphics[scale=0.55]{obrazki/istniejace_rozwiazania/breathwrk/wrk_shortcuts.png}
    \caption{Okno \textit{Shortcuts} w aplikacji Breathwrk}
    \label{img/wrk_shortcuts}
\end{figure}

Główna zakładka aplikacji to \textit{Now}. Na samej górze widoczny jest streak, a także zakładka \textit{favourites} (pol. ulubione).
Środkową część zajmuje karuzela z treningami oddechowymi, które są obecnie trendujące lub zostały wybrane dla użytkownika. W prawym górnym rogu dostępna jest opcja dodania treningu do ulubionych poprzez kliknięcie ikony serca, a także ikona informacyjna, która otwiera okno z opisem treningu. Dostępny jest ogólny opis treningu, propozycje użycia treningu oraz opis biologicznych procesów, które aktywuje trening. U góry okna umieszczone zostały piktogramy symbolizujące w jaki sposób wykonywać wdechy i wydechy oraz czas trwania poszczególnych etapów. Na rysunku \ref{img/wrk_opis} przedstawiono widok opisu trenigu.

\begin{figure}[H]
    \centering
    \includegraphics[scale=0.50]{obrazki/istniejace_rozwiazania/breathwrk/wrk_opis.png}
    \caption{Podzakładki strony opisu treningu}
    \label{img/wrk_opis}
\end{figure}

Po uruchomieniu treningu pokazuje się domyślna animacja --- koło. Po kliknięciu na środek ekranu pokazuje się czas do zakończenia treningu, ikona informacji,odnosząca do opisu treningu, przycisk pauzy oraz rezygnacji z treningu. Możliwe jest też dodanie treningu do ulubionych i jego udostępnienie. Dostępne są także opcje konfiguracji treningu --- wybór animacji (koło, małpka lub linia), dodanie nawyku, włączenie instrukcji głosowych, włączenie i wybór dźwięku w tle, a także wibracje telefonu. Widok tej strony przedstawiony został na rysunku \ref{img/wrk_trening}.
Po zakończeniu treningu pokazuje się okno ze statystykami oraz gratulacjami. Pokazuje się także przycisk do ponowienia treningu z wybranymi wcześniej ustawieniami.

\begin{figure}[H]
    \centering
    \includegraphics[scale=0.48]{obrazki/istniejace_rozwiazania/breathwrk/wrk_trening.png}
    \caption{Strona treningu w aplikacji Breathwrk}
    \label{img/wrk_trening}
\end{figure}
\begin{figure}[H]
    \centering
    \includegraphics[scale=0.85]{obrazki/istniejace_rozwiazania/breathwrk/wrk_immersje.png}
    \caption{Przykładowe treningi immersyjne w aplikacji Breathwrk}
    \label{img/wrk_immersje}
\end{figure}

Na dole zakładki znajduje się podsekcja \textit{Daily Classes} (pol. Codzienne Ćwiczenia) z zestawem 3 ćwiczeń immersyjnych, które zmieniają się co 24 godziny. Ćwiczenia te  charakteryzują się nietypowymi wrażeniami wizualnymi --- efekt promieniowania czy animacja wody w basenie. Przykładowe ćwiczenie przedstawione zostały na rysunku \ref{img/wrk_immersje}.

\textit{Explore}  to zakładka będąca biblioteką treningów, ćwiczeń, wyzwań, nawyków (opcja dla użytkowników  \textit{Breathwrk Premium}) oraz testów. Dostępne są także zakładki, po których można szybko filtrować treść: \textit{All}, \textit{Calming}, \textit{Nighttimel}, \textit{Energizing}, \textit{Perform} oraz \textit{Health}. 
Rysunek \ref{img/wrk_discover} przedstawia wygląd tej zakładki.

\begin{figure}[H]
    \centering
    \includegraphics[scale=0.5]{obrazki/istniejace_rozwiazania/breathwrk/wrk_discover.png}
    \caption{Zakładka \textit{Discover} w aplikacji Breathwrk)}
    \label{img/wrk_discover}
\end{figure}

\begin{figure}[H] 
    \centering
    \includegraphics[scale=0.5]{obrazki/istniejace_rozwiazania/breathwrk/wrk_me.png}
    \caption{Zakładka \textit{Me}}
    \label{img/wrk_me}
\end{figure}


Celem zakładki \textit{Schedule} jest zachęcenie użytkownika do systematycznego używania aplikacji. Możliwe jest ustawienie pojedynczego przypomnienia na trening lub regularnych przypomnień w wybrane dni, tak aby stworzyć nawyk.

 
Zakładka \textit{Me} składa się z trzech sekcji: statystyk użytkownika, testów udoskonalających oddech oraz tabeli najlepszych użytkowników, ocenianych na podstawie miesięcznego lub tygodniowego czasu spędzonego na aktywnościach oddechowych w aplikacji. Zakładka przedstawiona została na rysunku \ref{img/wrk_me}

\subsection{Ograniczenia}
Aplikacja jest przedstawiana jako umożliwiająca pełną personalizację ćwiczeń oddechowych, mimo to zawiera ograniczenia.

W aplikacji brakuje przede wszystkim możliwości wyboru długości trwania poszczególnych kroków. Możliwy jest jedynie wybór czasu trwania całego treningu, jednak jest on także ograniczony do trzech predefiniowanych wyborów dla mniejszej ilości interwałów. Dla większej liczby interwałów można je wybrać z dokładnością do jednego interwału, jest to natomiast funkcja dostępna jedynie w płatnym planie \textit{Breathwrk Premium}

Kolejnym ograniczeniem jest brak możliwości tworzenia własnych treningów. Do wyboru pozostają wyłącznie treningi oferowane przez aplikację. Dla użytkowników wersji \textit{Breathwrk Premium} wybór gotowych treningów i zajęć jest szeroki, natomiast dla pozostałych użytkowników jest mocno ograniczony --- po jednym lub nawet braku treningu spośród kilku z każdej z kategorii.

\section{STAmina Apnea Trainer firmy Squarecrowd Apps SIA (\Karol)} 
Kolejnym ważnym rozwiązaniem jest aplikacja STAmina Apnea Trainer \cite{Squarecrowd}, stworzona przez firmę Squarecrowd Apps SIA. To narzędzie mobilne wspomagające trening wstrzymywania oddechu, szczególnie przydatne dla freediverów, sportowców oraz osób praktykujących techniki oddechowe i relaksacyjne.

\subsection{Cel aplikacji}
STAmina Apnea Trainer to narzędzie przeznaczone przede wszystkim dla osób ćwiczących freediving, łowiectwo podwodne oraz inne dyscypliny wodne, w których kluczowe znaczenie ma umiejętność długiego wstrzymywania oddechu. Głównym założeniem aplikacji jest zapewnienie kompleksowego wsparcia w treningu \textit{static apnea} (pol. bezdech statyczny), wykorzystującego różnorodne tablice treningowe (między innymi ukierunkowane na tolerancję na niedotlenienie --- O$_2$, nadmiar dwutlenku węgla --- CO$_2$, a także kombinacje, techniki relaksacyjne i możliwość tworzenia własnych schematów).

\subsection{Funkcjonalności}
Aplikacja udostępnia pięć gotowych szablonów treningowych, które pozwalają systematycznie rozwijać poszczególne aspekty wstrzymywania oddechu.
Pierwszy z nich koncentruje się na niedoborze O$_2$, czyli stopniowym wydłużaniu czasu wstrzymania przy stałym czasie odpoczynku. 
Drugi skupia się na tolerancji CO$_2$, w którym użytkownik zmniejsza przerwy między kolejnymi sesjami wstrzymania oddechu, by przyzwyczaić organizm do wyższego poziomu CO$_2$.
Kolejny schemat, nazwany \textit{Wonka}, wprowadza pauzę po pierwszym odczuwalnym skurczu przepony, co pomaga uczyć się właściwej techniki. 
Schemat \textit{Mix} (pol. Mieszany) łączy oba podejścia (O$_2$/CO$_2$), zapewniając wszechstronne podejście do treningu, natomiast \textit{Pranayama} oferuje techniki oddechowe o charakterze relaksacyjnym, przydatne zarówno przed jak i po głównej części treningu. Na rysunku \ref{img/sta_schematy} przedstawiono gotowe schematy dostępne w~aplikacji.

\begin{figure}[ht]
\centering
\includegraphics[scale=0.25]{obrazki/istniejace_rozwiazania/STA/STA_schematy.png}
\caption{Gotowe schematy w aplikacji STAmina Apnea Trainer}
\label{img/sta_schematy}
\end{figure}

 Poza gotowymi programami każdy użytkownik może stworzyć trening \textit{Custom} (pol. własny) z pełną konfigurowalnością, dobierając liczbę powtórzeń, czasy wstrzymywania i przerw według indywidualnych potrzeb. Na podstawie dotychczasowego \textit{Personal Best} (pol. rekordu) aplikacja automatycznie generuje sesje w trzech stopniach zaawansowania --- \textit{Easy} (pol. łatwy), \textit{Normal} (pol. normalny), \textit{Hard} (pol. trudny), co ułatwia progresję i dostosowanie wysiłku do aktualnych możliwości. Tworzenie treningu przedstawiono na rysunku \ref{img/sta_generate}.
 
 Aby wspomóc systematyczne korzystanie, wbudowany system powiadomień przypomina o zaplanowanych treningach. Wszystkie sesje są zapisywane, a użytkownik może obserwować swoje postępy dzięki szczegółowym przeglądom rekordów i trendów w czasie. Dodatkowym wsparciem jest nawigacja głosowa prowadzona przez nagrania profesjonalnych lektorów w wersjach męskiej i żeńskiej, dostępna w kilku językach (m.in. angielskim, francuskim, niemieckim, włoskim i rosyjskim), co pozwala skupić się wyłącznie na ćwiczeniu, bez konieczności patrzenia na ekran. Integracja z Apple Health umożliwia rejestrowanie tętna i poziomu SpO$_2$ za pomocą Apple Watch lub innych urządzeń Bluetooth, co dostarcza cennych danych biometricznych podczas treningów. Przedstawiono to na rysunku \ref{img/sta_heart}.
 
 Synchronizacja z chmurą gwarantuje \textit{backup} (pol. kopia zapasowa) i przywracanie danych pomiędzy urządzeniami, co jest istotne dla osób korzystających z aplikacji na różnych sprzętach. Aplikacja oferuje także możliwość śledzenia momentu wystąpienia skurczów oddechowych, co pomaga w lepszym zrozumieniu własnych granic i rozwoju techniki. Dla lepszego komfortu dostępne są wibracyjne alerty oraz \textit{square breath} (pol. tryb oddechu pudełkowego), wspierające relaksację i stabilizację oddechu.

  \begin{figure}[H]
    \centering
    \begin{minipage}{0.48\textwidth}
        \centering
        \includegraphics[scale=0.15]{obrazki/istniejace_rozwiazania/STA/STA_generate.png}
\caption{Tworzenia treningu w aplikacji STAmina Apnea Trainer}
\label{img/sta_generate}
    \end{minipage}
    \hfill
    \begin{minipage}{0.48\textwidth}
        \centering
        \includegraphics[scale=0.15]{obrazki/istniejace_rozwiazania/STA/STA_serce.png}
        \caption{Wykresy poziomu SpO$_2$ w aplikacji STAmina Apnea Trainer}
        \label{img/sta_heart}
    \end{minipage}
\end{figure}
 
 Zakładka z ogólnymi pomocami dotyczącymi treningów oddechowych oraz freedivingu dostarcza wiedzy teoretycznej i praktycznych wskazówek, co pomaga użytkownikom zarówno początkującym, jak i bardziej zaawansowanym.

\subsection{Ograniczenia}
Mimo bogatego zestawu funkcji, aplikacja ma też swoje wady. Interfejs jest już nieco przestarzały, co może obniżać komfort użytkowania i sprawiać, że poruszanie się po aplikacji wydaje się mniej intuicyjne w porównaniu ze współczesnymi standardami. Brak wbudowanego przewodnika lub samouczka powoduje, że nowi użytkownicy mogą mieć trudności z rozpoczęciem treningów. Pomocne mogłoby okazać się interaktywne wskazówki dla początkujących czy wprowadzenie krok po kroku.

\section{Breathe firmy Havabee (\Jakub)}
Kolejną aplikacją wartą uwagi jest Breathe \cite{Havabee} firmy Havabee. Wyróżnia się spośród innych swoim minimalizmem i łatwością obsługi. 

\subsection{Cel aplikacji}
Celem aplikacji jest umożliwienie użytkownikom, niezależnie od wieku i zaawansowania technologicznego, wykonywania treningów oddechowych. Została ona zaprojektowana z~myślą o~osobach niewymagających wiele, co potwierdzają proste treningi dostarczone razem z aplikacją.

\subsection{Funkcjonalności}
Aplikacje w swoim przejrzystym interfejsie zawiera jedynie trzy podstrony: \textit{Domyślnie} (Strona główna), \textit{Własne} (Strona konfiguracji treningów) i \textit{Postęp} (Strona ze statystykami).

Podstrona \textit{Domyślnie} zawiera cztery zdefiniowane przez twórców aplikacji treningi, takie jak: \textit{Oddychanie równoważne}, \textit{Oddychanie pudełkowe}, \textit{Oddychanie sposobem 4-7-8} i \textit{Test wstrzymania oddechu}. Aby dowiedzieć się więcej o danym treningu, wystarczy kliknąć na ikonę informacji. Możliwa jest również zmiana trwania schematów poprzez kliknięcie w przycisk \textit{Zmień czas trwania}. Gotowe schematy zostały przedstawione na rysunku \ref{img/havabee_menu}. 

\begin{figure}[H]
    \centering
    \begin{minipage}{0.48\textwidth}
        \centering
        \includegraphics[scale=0.11]{obrazki/istniejace_rozwiazania/Havabee/Havabee_menu.jpg}
\caption{Schematy dostępne w~aplikacji Breathe}
\label{img/havabee_menu}
    \end{minipage}
    \hfill
    \begin{minipage}{0.48\textwidth}
        \centering
        \includegraphics[scale=0.10]{obrazki/istniejace_rozwiazania/Havabee/Havabee_trening.jpg}
        \caption{Trening w aplikacji Breathe}
        \label{img/havabee_trening}
    \end{minipage}
\end{figure}

Po wybraniu interesującego nas schematu przenoszeni jesteśmy do strony treningu. Aby przygotować użytkownika do ćwiczeń, w pierwszej kolejności odbywa się krótka sesja pozwalająca skupić się na własnym oddechu i przejść w stan świadomej pracy oddechowej. Po niej przechodzimy do właściwego ćwiczenia, które prowadzone jest zarówno wizualnie, jak i w formie audio. Liczba odbytych cykli wyświetlana jest w postaci zapełniającej się obwódki kółka, jednak brak bezpośredniego wyświetlania wartości liczbowych.
Dostępne jest również ustawienie poziomu dźwięku, jednakże brakuje możliwości zatrzymania treningu. Można jedynie powrócić do okna startowego, w dodatku bez ostrzeżenia użytkownika o natychmiastowym zakończeniu sesji. Trening został przedstawiony na rysunku \ref{img/havabee_trening}.

Celem kolejnej zakładki jest umożliwienie użytkownikom tworzenie własnych wzorów oddychania. Po kliknięciu w przycisk plusa (+), zostajemy przeniesieni na stronę konfiguracji treningu. Możliwe jest nadanie treningowi nazwy, a także wybranie czasu trwania każdej z faz oddechu (Wdech, Wstrzymanie, Wydech, Wstrzymanie) i liczby cykli w przedziale od jednego do dwustu pięćdziesięciu. Na rysunku \ref{img/havabee_custom} przedstawiono tę zakładkę. 

Na ostatniej stronie możemy podejrzeć czas, jaki spędziliśmy w aplikacji oraz liczbę odbytych sesji w danym dniu, tygodniu, miesiącu i roku. Poniżej ogólnych statystyk znajduje się także osobna sekcja do podglądu odbytych treningów wstrzymywania oddechu, na której można zaobserwować przeciętny i najdłuższy czas trwania takiego oddechu w tych samych ramach czasowych.

\begin{figure}[H]
    \centering
    \includegraphics[scale=0.20]{obrazki/istniejace_rozwiazania/Havabee/havabee_custom.png}
    \caption{Tworzenie własnych treningów w aplikacji Breathe}
    \label{img/havabee_custom}
\end{figure}

\subsection{Ograniczenia}
Wspomniana aplikacja, mimo swojej prostej szaty graficznej oraz oferowanych funkcjonalności, nie jest w stanie sprostać oczekiwaniom wszystkich użytkowników. Każdy z dostępnych wzorców treningowych umożliwia jedynie powtarzanie określonego cyklu oddechowego z góry ustaloną liczbę razy. Stanowi to istotne ograniczenie dla osób, które w swoich treningach wykorzystują zróżnicowane cykle oddechowe lub takie, które składają się z mniejszej liczby faz. Dodatkowym mankamentem jest brak możliwości ustawienia czasu wstrzymania oddechu na zero sekund lub bardzo krótkich wdechów, co skutkuje nieprawidłowym działaniem dźwięku towarzyszącego ćwiczeniu. Koniecznie należy wspomnieć o braku możliwości zatrzymania treningu i mało przejrzystego wyświetlania liczby cykli podczas treningu.

\section{Podsumowanie analizowanych aplikacji \Hania}

W tabeli \ref{tab:podsumowanie_aplikacji} zestawiono kluczowe cechy omówionych aplikacji. Pozwala to na szybkie porównanie ich funkcjonalności, zakresu personalizacji oraz ograniczeń. Zestawienie uwzględnia zarówno aplikacje o rozbudowanej strukturze, jak i rozwiązania minimalistyczne, skupione na prostych schematach oddechowych.

Z przeprowadzonej analizy wynika, że dostępne aplikacje wyraźnie dzielą się na dwa \linebreak typy --- jeden stawia na doświadczenie i wygląd, a drugi na szeroką możliwość konfiguracji. \linebreak W obu podejściach pojawiają się ograniczenia --- pierwsze oferują mało elastyczności, a drugie często mają uproszczoną formę. Żadna nie łączy pełnej personalizacji z nowoczesnym, lekkim podejściem, co pokazuje lukę i uzasadnia stworzenie bardziej elastycznego rozwiązania.

\begin{table}[H]
\centering
\hspace*{-1cm}
\renewcommand{\arraystretch}{1.4}
\begin{tabular}{|p{1.7cm}|p{4.7cm}|p{4.2cm}|p{4.5cm}|}
\hline
\textbf{Aplikacja} & \textbf{Najważniejsze funkcje} & \textbf{Personalizacja} & \textbf{Główne ograniczenia} \\ \hline

\textbf{Wim Hof \linebreak Method} &
Treningi oddechowe, \newline 
ekspozycja na zimno, \newline 
medytacje, statystyki, \newline 
odznaki, społeczność. &
Zmiana tempa oddechu, \newline
liczby rund, oddechów, \newline
tła dźwiękowego. \newline
Brak tworzenia własnych \newline
treningów. &
Wiele funkcji tylko w wersji \newline
Premium; \newline
ograniczona elastyczność \newline
konfiguracji.\\ 
\hline
\textbf{Breathwrk} &
Setki ćwiczeń, \newline 
zajęcia z instruktorami, \newline 
animacje, tryby immersyjne, \newline 
przypomnienia, statystyki. &
Wybór animacji, dźwięków,\newline
czasu trwania treningu, \newline 
podstawowe ustawienia \newline  
kroków (ograniczone). &
Brak możliwości tworzenia \newline 
własnych treningów; \newline 
ograniczenia czasu \newline 
trwania ćwiczeń; \newline 
wiele treści wyłącznie\newline 
w planie Premium. \\ 
\hline
\textbf{STAmina \linebreak Apnea \linebreak Trainer} &
Tablice CO$_2$/O$_2$, \newline 
bezdech statyczny, \newline 
śledzenie rekordów, \newline 
integracja z Apple Health, \newline 
analiza SpO$_2$, \newline 
głosowe instrukcje. &
Pełne tworzenie treningów; \newline  
regulacja czasów trwania \newline  
każdej fazy; \newline 
generowanie sesji \newline 
na podstawie rekordów. &
Starszy interfejs; \newline 
brak samouczka; \newline 
wyższy próg wejścia \newline 
dla początkujących. \\ 
\hline
\textbf{Breathe \linebreak (Havabee)} &
Schematy oddechowe, \newline
tworzenie własnych wzorców, \newline
statystyki sesji, \newline
intuicyjny interfejs. &
Możliwość definiowania faz \newline
oddechu i liczby cykli; \newline
tworzenie własnych \newline
treningów. &
Brak zatrzymania \newline
treningu; \newline
brak złożonych sekwencji \newline
oddechowych; \newline
ograniczenia minimalnych \newline
czasów faz; \newline
mniej czytelne wizualizacje postępu. \\ 
\hline
\end{tabular}
\caption{Porównanie funkcjonalności analizowanych aplikacji oddechowych}
\label{tab:podsumowanie_aplikacji}
\end{table}

\chapter{Założenia projektowe \Karol}

\section{Dla kogo rozwiązanie jest przeznaczone?  \Karol}

\section{Jakie korzyści ma dostarczać? \Karol}
\textbf{//na razie z RPI}

ReSpire to aplikacja mobilna wspierająca trening oddechowy. Umożliwia użytkownikom projektowanie i zarządzanie spersonalizowanymi sesjami treningu oddechowego. Dzięki szerokim możliwościom personalizacji, ReSpire pozwala na tworzenie unikalnych planów treningowych i ich realizację w intuicyjny sposób. 
Aplikacja wychodzi naprzeciw potrzebie złożonej konfiguracji ćwiczeń oddechowych. Służy ona jak najlepszemu dopasowaniu treningów do użytkowników i ich potrzeb, zapewniając dużą elastyczność i swobodę. To wyróżnia ją od innych aplikacji dostępnych na rynku, które nie pozwalają na dostosowywanie w tak znacznym stopniu. Jednocześnie zaspokaja również potrzeby osób, których nie interesuje personalizacja i wolą skorzystać       z gotowych, prostszych treningów oddechowych.

Celem projektu jest stworzenie tworzenie intuicyjnej i funkcjonalnej aplikacji mobilnej, umożliwiającej użytkownikom skuteczny trening oddechowy, która ma pomagać w poprawie kontroli oddechu, redukcji stresu oraz zwiększeniu świadomości oddechowej. Aplikacja zostanie opublikowana w Google Play i App Store, aby dotrzeć do jak najszerszego grona użytkowników w Polsce. Projekt stanowi część pracy inżynierskiej.

Zakres projektu obejmuje pełny cykl tworzenia aplikacji mobilnej ReSpire, od fazy projektowania po wdrożenie i publikację. 

W ramach prac zostanie zaprojektowany i zaimplementowany intuicyjny interfejs użytkownika, który umożliwi łatwe zarządzanie sesjami treningu oddechowego. Kluczowym aspektem aplikacji będzie szeroka personalizacja poprzez wybór parametrów takich jak typ kroku (wdech, wydech, przerwa), długość kroku, inkrementacja czy powtórzeń. Projekt zakłada również opracowanie mechanizmów interaktywnych, takich jak wizualizacje wspomagające kontrolę oddechu.
Zwieńczeniem projektu będzie publikacja aplikacji w sklepach Google Play i App Store, co umożliwi szeroką dostępność wśród użytkowników w Polsce, jak i za granicą.

\section{Jakie mają być główne funkcjonalności? \Karol}

\section{Jacy użytkownicy są przewidziani? \Karol}

\section{Czy i ew. z jakimi systemami ma współpracować? \Karol}

\section{Pozostałe założenia \Karol}
\chapter{Analiza wymagań \Hania\ \Ola\ \Jakub}

W niniejszym rozdziale przedstawione zostały wymagania projektowanej aplikacji. Użyto w~tym celu trzech modeli - przypadków użycia, klas oraz sterowania dźwiękami.

\section{Model przypadków użycia \Ola}
Celem modelu jest określenie głównych funkcji aplikacji, aktorów oraz relacji między nimi. W jego zakres wchodzą wszystkie funkcje dostępne dla użytkownika końcowego. Wyróżnionych zostało czterech aktorów, w tym jeden ożwiony: osoba korzystająca z aplikacji oraz trzech nieożywionych: lokalna baza danych, syntezator mowy oraz odtwarzacz dźwięków. Ze względu na~czytelność, diagram przypadków użycia został podzielony na trzy mniejsze części, gdzie zgrupowane są przypadki ze sobą powiązane. 

Pierwszy z diagramów został przedstawiony na rysunku \ref{img/przypadki_uzycia_1}. Zawiera podstawowe przypadki użycia, takie jak: wyświetlanie strony treningu \ref{tab:wyswietlenie_strony_treningu} lub ustawień, usuwanie \ref{tab:usuwanie_treningu}, eksport \ref{tab:eksport_treningu}, import \ref{tab:import_treningu}, edycję \ref{tab:edycja_treningu}, dodawanie \ref{tab:dodawanie_treningu}, odtwarzanie \ref{tab:odtwarzanie_treningu} lub wyświetlanie listy albo szczegółów treningów oddechowych. Dla edycji, dodawania oraz odtwarzania związki rozszerzenia (ang. extend) i~zawierania (ang. include) z innymi przypadkami użycia zostały przedstawione na osobnych fragmentach.
\begin{figure}[H]
\centering
\includegraphics[scale=0.17]{obrazki/analiza_wymagan/przypadki_uzycia_1.jpg}
\caption{Główne przypadki użycia dla aplikacji ReSpire}
\label{img/przypadki_uzycia_1}
\end{figure}

\begin{table}[H]
\centering
\renewcommand{\arraystretch}{1.3}
\begin{tabular}{|p{4cm}|p{11cm}|}
\hline
\textbf{Warunki początkowe} & Istnieje co najmniej jeden trening, wykonano przypadek użycia \textit{Wyświetlenie listy treningów} \\ \hline
\textbf{Przebieg} & 
1. Użytkownik zgłasza żądanie wyświetlenia strony wybranego treningu.\\
& 2. System wyświetla stronę zgodnie z żądaniem.\\
& 3. Użytkownik może zgłosić żądanie edycji \ref{tab:edycja_treningu}, usunięcia \ref{tab:usuwanie_treningu}, eksportu \ref{tab:eksport_treningu} lub odtworzenia treningu \ref{tab:odtwarzanie_treningu}. \\ \hline
\textbf{Przebieg alternatywny} & brak \\ \hline
\textbf{Warunki końcowe} & brak \\ \hline
\end{tabular}
\caption{Wyświetlenie strony treningu}
\label{tab:wyswietlenie_strony_treningu}
\end{table}

\begin{table}[H]
\centering
\renewcommand{\arraystretch}{1.3}
\begin{tabular}{|p{4cm}|p{11cm}|}
\hline
\textbf{Warunki początkowe} & Istnieje co najmniej jeden trening, wykonano przypadek użycia \textit{Wyświetlenie strony treningu} \ref{tab:wyswietlenie_strony_treningu}\\ \hline
\textbf{Przebieg} & 
1. Użytkownik zgłasza żądanie usunięcia treningu.\\
& 2. System wyświetla okno z prośbą o potwierdzenie akcji.\\
& 3. Użytkownik potwierdza żądanie.\\
& 4. Użytkownik zgłasza żądanie opuszczenia strony edycji. \\
& 5. System usuwa trening z bazy.\\\hline
\textbf{Przebieg alternatywny} 
& 3a. Użytkownik nie zgłasza żadnego żądania modyfikacji treningu. \\ 
& 3a1. Użytkownik zgłasza żądanie opuszczenia strony edycji.\\
& 3a2. Przypadek użycia zostaje przerwany. \\ \hline
\textbf{Warunki końcowe} & Trening jest poprawnie usunięty i nie wyświetla się na liście. \\ \hline
\end{tabular}
\caption{Usuwanie treningu}
\label{tab:usuwanie_treningu}
\end{table}

\begin{table}[H]
\centering
\renewcommand{\arraystretch}{1.3}
\begin{tabular}{|p{4cm}|p{11cm}|}
\hline
\textbf{Warunki początkowe} & Użytkownik znajduje się na stronie głównej\\ \hline
\textbf{Przebieg} & 
1. Użytkownik zgłasza żądanie importu treningu.\\
& 2. System wyświetla okno z eksploratorem plików.\\
& 3. Użytkownik wybiera plik w formacie JSON do importu.\\\hline
\textbf{Przebieg alternatywny} 
& 3a. Użytkownik wybiera plik o błędnym rozszerzeniu, złej strukturze lub przerywa akcję. \\ 
& 3a1. Przypadek użycia zostaje przerwany, system wyświetla odpowiedni komunikat. \\ \hline
\textbf{Warunki końcowe} & Trening lub treningi zostają poprawnie zaimportowane i pojawiają się na liście. \\ \hline
\end{tabular}
\caption{Import treningu}
\label{tab:import_treningu}
\end{table}

\begin{table}[H]
\centering
\renewcommand{\arraystretch}{1.3}
\begin{tabular}{|p{4cm}|p{11cm}|}
\hline
\textbf{Warunki początkowe} & Istnieje co najmniej jeden trening, wykonano przypadek użycia \textit{Wyświetlenie strony treningu} \ref{tab:wyswietlenie_strony_treningu} lub otwarta jest strona główna\\ \hline
\textbf{Przebieg} & 
1. Użytkownik zgłasza żądanie eksportu treningu (lub treningów).\\
& 2. System wyświetla okno eksploratora plików.\\
& 3. Użytkownik wybiera tytuł pliku oraz jego lokalizację i potwierdza żądanie zapisu. \\
& 4. System wyświetla potwierdzenie zapisu.\\\hline
\textbf{Przebieg alternatywny} 
& 3a. Użytkownik przerywa akcję zapisu. \\ 
& 3a1. Przypadek użycia zostaje przerwany, system wyświetla odpowiedni komunikat. \\ \hline
\textbf{Warunki końcowe} & Trening w formacie JSON zostaje zapisany na urządzeniu użytkownika. \\ \hline
\end{tabular}
\caption{Eksport treningu}
\label{tab:eksport_treningu}
\end{table}


Drugi diagram został przedstawiony na rysunku \ref{img/przypadki_uzycia_2}. Przedstawia wspomniane wcześniej: edycję treningu \ref{tab:edycja_treningu} oraz dodawnie treningu \ref{tab:dodawanie_treningu} wraz z powiązanymi z nim przypadkami.

\begin{figure}[H]
\centering
\includegraphics[scale=0.21]{obrazki/analiza_wymagan/przypadki_uzycia_2.jpg}
\caption{Przypadki użycia dla aplikacji ReSpire - edycja treningu}
\label{img/przypadki_uzycia_2}
\end{figure}

\begin{table}[H]
\centering
\renewcommand{\arraystretch}{1.3}
\begin{tabular}{|p{4cm}|p{11cm}|}
\hline
\textbf{Warunki początkowe} & Istnieje co najmniej jeden trening, widok ze szczegółami treningu jest otwarty \\ \hline
\textbf{Przebieg} & 
1. Użytkownik zgłasza żądanie edycji treningu.\\
& 2. System przenosi użytkownika na stronę edytora.\\
& 3. Użytkownik może zgłosić żądania: edycji opisu, tytułu, dźwięków w~tle lub zmiany, dodania czy usunięcia etapu (wraz z fazami).\\
& 4. Użytkownik zgłasza żądanie opuszczenia strony edytora. \\
& 5. System dokonuje aktualizacji treningu o wprowadzone zmiany\\ \hline
\textbf{Przebieg alternatywny} 
& 3a. Użytkownik nie zgłasza żadnego żądania modyfikacji treningu. \\ 
& 3a1. Użytkownik zgłasza żądanie opuszczenia strony edytora.\\
& 3a2. Przypadek użycia zostaje przerwany. \\ \hline
\textbf{Warunki końcowe} & Trening jest poprawnie zaktualizowany o wprowadzone zmiany. \\ \hline
\end{tabular}
\caption{Edycja treningu}
\label{tab:edycja_treningu}
\end{table}

\begin{table}[H]
\centering
\renewcommand{\arraystretch}{1.3}
\begin{tabular}{|p{4cm}|p{11cm}|}
\hline
\textbf{Warunki początkowe} & Istnieje co najmniej jeden trening, widok ze szczegółami treningu jest otwarty \\ \hline
\textbf{Przebieg} & 
1. Użytkownik zgłasza żądanie dodania treningu.\\
& 2. System przenosi użytkownika na stronę edytora.\\
& 3. Użytkownik zgłasza żądania dodania etapów wraz z fazami. \\
& 4. Użytkownik może także zgłosić żądania: edycji opisu, tytułu, dźwięków w tle lub usunięcia etapów/faz.\\
& 5. Użytkownik zgłasza żądanie opuszczenia strony edytora.\\
& 6. System dokonuje aktualizacji listy treningów.\\ \hline
\textbf{Przebieg alternatywny} 
& 3a. Użytkownik nie zgłasza żądania dodania żadnego elementu treningu.\\ 
& 3a1. Użytkownik zgłasza żądanie opuszczenia strony edytora.\\
& 3a2. System wyświetla komunikat o pustym treningu.\\
& 3a3. Użytkownik potwierdza chęć przerwania dodawania treningu i~przypadek użycia zostaje przerwany lub wraca do punktu 3.\\ 
& 3b. Użytkownik zgłasza żądanie dodania etapów bez dodania faz.\\ 
& 3b1. System wyświetla komunikat o niepełnym treningu.\\
& 3b2. Użytkownik uzupełnia trening o brakujące elementy, a następnie wraca~do punktu 4 lub rezygnuje z dodania treningu i~przypadek użycia zostaje przerwany.\\ \hline
\textbf{Warunki końcowe} & Trening został dodany do listy. \\ \hline
\end{tabular}
\caption{Dodawanie treningu}
\label{tab:dodawanie_treningu}
\end{table}


Ostatni, trzeci diagram został przedstawiony na rysunku \ref{img/przypadki_uzycia_3}.
\begin{figure}[H]
\centering
\includegraphics[scale=0.8]{obrazki/analiza_wymagan/przypadki_uzycia_3.png}
\caption{Przypadki użycia dla aplikacji ReSpire - odtwarzanie treningu}
\label{img/przypadki_uzycia_3}
\end{figure}

\begin{table}[H]
\centering
\renewcommand{\arraystretch}{1.3}
\begin{tabular}{|p{4cm}|p{11cm}|}
\hline
\textbf{Warunki początkowe} & Istnieje co najmniej jeden trening, wykonano przypadek użycia \textit{Wyświetlenie strony treningu} \ref{tab:wyswietlenie_strony_treningu}\\ \hline
\textbf{Przebieg} & 
1. Użytkownik zgłasza żądanie odtworzenia treningu.\\
& 2. System odtwarza trening z wizualizacją.\\
& 3. Jeśli zostały ustwione odpowiednie parametry w edytorze syntezator mowy może zażądać odtworzenia instrukcji głosowych, a odtwarzacz dźwięków - odtwarzania dźwięków lub wybranych częstotliwości. \\
& 4. Użytkownik może zażądać wstrzymania treningu.\\
& 5. System zakańcza odtwarzanie treningu po upływie określonego czasu. \\ \hline
\textbf{Przebieg alternatywny} 
& 2a. Użytkownik zgłasza żądanie opuszczenia treningu. \\ 
& 3a1. System wyświetla okno z zapytaniem o potwierdzenie akcji.\\
& 3a2. Użytkownik potwierdza żądanie opuszczenia treningu i przypadek użycia zostaje przerwany lub analuje akcje i wraca do~punktu 2. \\ \hline
\textbf{Warunki końcowe} & Następuje powrót na stronę szczegółów treningu. \\ \hline
\end{tabular}
\caption{Otwarzanie treningu}
\label{tab:odtwarzanie_treningu}
\end{table}


\section{Modele klas \Jakub}
\subsection{Model serwisów}
\subsection{Model treningu}

\section{Model sterowania dźwiękami \Hania}

oś czasu z zaznaczonymi zdarzeniami
cel: żeby pokazać logikę ich odtwarzania w czasie
opis diagramu i logiki
diagram przedtawiono na rysunku \ref{img/dzwieki_w_tle}, \ref{img/dzwieki_w_etapie}
\begin{figure}[ht]
\centering
\hspace*{-2cm}
\includegraphics[scale=0.25]{obrazki/analiza_wymagan/dzwieki_w_tle.jpg}
\caption{Diagram sterowania dźwiękami w tle}
\label{img/dzwieki_w_tle}
\end{figure}

\begin{figure}[ht]
\centering
\hspace*{-2cm}
\includegraphics[scale=0.25]{obrazki/analiza_wymagan/dzwieki_w_etapie.jpg}
\caption{Diagram sterowania dźwiękami w etapie}
\label{img/dzwieki_w_etapie}
\end{figure}
\chapter{Projekt rozwiązania}

\section{Projekt architektury systemu \Hania\ \Ola}


\section{Projekt logiki aplikacji \Hania\ \Ola}

\section{Projekt interfejsu użytkownika \Hania\ \Ola}
\subsection{Projekt w Figma}\label{subsec:Figma}
\todoHania{tu bym opisała tylko figme}
Wstępny projekt interfejsu użytkownika został zaprojektowany w Figmie.

W narzędziu Canva zostało zaprojektowane logo aplikacji - połączenie dwóch różnych czcionek oraz dodatek grafiki podmuchu i detalu symbolizującego cząsteczkę powietrza nad literą “i” oddaje ducha ReSpire.

W oparciu o kolorystykę elementów w logo zostały wybrane 3 główne kolory aplikacji - są to odcienie niebieskiego. Kolor ten jest kojarzony z poczuciem spokoju, harmonii, stabilizacji i bezpieczeństwa.( tu cytat). Dzięki temu aplikacja jest spójna kolorystycznie, a także wywołuje przyjemne odczucia w użytkowniku, zapewniając spokój i harmonię. 

Jedna z czcionek użytych w logo  - “Glacial Indifference” została również wykorzystana w UI aplikacji, zapewniając jej unikatowy charakter.
\todoOla{Ogolnie gdzieś możnaby podać źródła do animacji może i napisać, że wszystkie uzyte materiały są legalne}
Białe kafelki na stronie głównej z asymetrycznymi zaokrągleniami oraz symbolami podmuchu stanowią prosty, ale modernistyczny element aplikacji. Przycisk dodawania własnego treningu nawiązuje do detalu w logo, czyniąc aplikację wyjątkową.

\begin{figure}[H]
\centering
\includegraphics[scale=0.4]{obrazki/projekt_rozwiazania/figma_home.png}
\caption{Pierwszy projekt aplikacji ReSpire - strona główna}
\label{img/respire_home}
\end{figure}

\begin{figure}[H]
\centering
\includegraphics[scale=0.45]{obrazki/projekt_rozwiazania/figma_training.png}
\caption{Pierwszy projekt aplikacji ReSpire - przebieg treningu}
\label{img/respire_training}
\end{figure}

\begin{figure}[H]
\centering
\includegraphics[scale=0.44]{obrazki/projekt_rozwiazania/figma_settings.png}
\caption{Pierwszy projekt aplikacji ReSpire - edycja treningu}
\label{img/respire_settings}
\end{figure}

\newpage
\subsection{Spójność aplikacji}
\todoHania{Tu bym dała kolory, logo i ewentualnie coś o czcionkach, może o użytych ikonach z jakiej biblioteki}
\todoOla{Oki}
Kolory użyte w aplikacji
\begin{itemize}
    \item Hex: \#1A93A8, RGB: 26, 147, 168
    \item Hex: \#32B7CF, RGB: 50, 183, 207
    \item Hex: \#7BDEF0, RGB: 123, 222, 240
\end{itemize}

Ikony chyba z biblioteki "material"

\begin{figure}[H]
    \centering
    \begin{minipage}{0.48\textwidth}
        \centering
        \includegraphics[width=\textwidth]{obrazki/projekt_rozwiazania/logo_biale_tlo.png}
        \caption{Logo ReSpire}
        \label{img/respire_logo}
    \end{minipage}
    \hfill
    \begin{minipage}{0.48\textwidth}
        \centering
        \includegraphics[width=\textwidth]{obrazki/projekt_rozwiazania/respire_colors.png}
        \caption{Kolory użyte w aplikacji}
        \label{img/respire_colors}
    \end{minipage}
\end{figure}



\section{Projekt struktury danych (treningi) \Jakub}

Model danych oparto na strukturze hierarchicznej, której fundament stanowią trzy kluczowe klasy: \texttt{Trening}, \texttt{EtapTreningu} oraz \texttt{FazaOddechowa}. Główna encja, \texttt{Trening}, składa się z co najmniej jednego \texttt{EtapuTreningu} (relacja jeden do wielu). Każdy etap charakteryzuje się zdefiniowaną liczbą cykli (powtórzeń) oraz parametrem przyrostu czasu. Na najniższym poziomie hierarchii znajduje się \texttt{FazaOddechowa}. Etapy składają się z sekwencji faz o określonym czasie trwania oraz typie, wybieranym ze zbioru zdefiniowanych wartości: wdech, wydech, regeneracja lub~wstrzymanie. Opisywany projekt przedstawia rysunek~\ref{img/training_structure}, natomiast rozwinięcie i implementację tej struktury danych opisano w sekcji \ref{subsec:training_classes}.

\begin{figure}[H]
    \centering
    \includegraphics[width=\textwidth]{obrazki/projekt_rozwiazania/Struktura_treningu.jpg}
    \caption{Struktura treningu}
    \label{img/training_structure}
\end{figure}
\chapter{Implementacja aplikacji mobilnej ReSpire}
Projekt został wykonany w środowisku Flutter za pomocą zestawu narzędzi dla programistów Flutter w wersji 3.27.2 wspierającego język Dart w wersji 3.6.1. Ten zestaw narzędzi dla programistów umożliwia tworzenie aplikacji na platformy Android, iOS, Linux, macOS oraz Windows z wykorzystaniem jednej bazy kodu źródłowego. Wybór Fluttera był podyktowany jego wydajnością, bogatym zestawem narzędzi do tworzenia interfejsów użytkownika oraz dużą dostępnością wsparcia. Zapewnie trwałości danych między uruchomieniami zostało zrealizowane przy użyciu lokalnej, nierelacyjnej bazy danych Hive w wersji 2.2.3, przygotowanej pod środowisko Flutter oraz język Dart.

\section{Technologie}
Aplikacja została napisana w jęzku Dart we frameworku Flutter.

\section{Schemat plików/klas/modułów \Hania\ \Ola\ \Jakub\ \Karol}
Aplikacja ReSpire charakteryzuje się modułową budową, na którą składają się łącznie 63 pliki źródłowe w języku Dart. Projekt obejmuje 80 klas oraz ponad 400 metod realizujących logikę biznesową i interfejs użytkownika. Całkowity rozmiar projektu wynosi nieco ponad 10 kLOC (ang. Kilo Lines of Code – tysięcy linii kodu), z czego blisko 8900 stanowi tzw. czysty kod, określany jako SLOC (ang. Source Lines of Code). Ze względu na taką złożoność struktury, diagramy klas i architektury zostały w niniejszej dokumentacji podzielone na logiczne fragmenty, aby zachować ich czytelność.

\subsection{Klasy treningu} \label{subsec:training_classes}

Na trening składa się 12 klas, z czego główną z nich jest \texttt{Training}. Zawiera ona główne informacje o ćwiczeniu takie jak tytuł, czy opis, a także listę etapów, dźwięki i ustawienia. Struktura dźwięków została opisana w rodziale \ref{sec:Sounds}. 

\begin{figure}[H]
  \centering
  \includegraphics[width=\textwidth]{diagramy/class_diagram.png}
  \caption{Diagram klas treningu}
\end{figure}


\section{Edytor treningów}

Edytor treningów został zaimplementowany jako dedykowany widok \texttt{TrainingEditorPage}, który operuje bezpośrednio na obiekcie \texttt{Training}. Struktura treningu przechowywana jest w modelach Hive (\texttt{TrainingStage}, \texttt{BreathingPhase}, \texttt{Sounds}, \texttt{Settings}), dzięki czemu każda modyfikacja wprowadzona w interfejsie natychmiast trafia do tej samej instancji danych, którą aplikacja później serializuje. Cała logika edytora została podzielona na trzy zakładki sterowane przez komponent \texttt{CustomSlidingSegmentedControl}, co pozwoliło oddzielić edycję przebiegu treningu, konfigurację dźwięków oraz ustawienia uzupełniające, utrzymując jednocześnie wspólną nawigację i kontrolę zapisów. Sam \texttt{CustomSlidingSegmentedControl} jest komponentem zainspirowanym szeroko wykorzystywanym w systemie mobilnym Apple iOS oraz platformie technologicznej SwiftUI komponenetem \texttt{segmented}, który umożliwia przełączanie się między różnymi widokami w obrębie jednej strony aplikacji przy jednoczesnej informacji o dostępnych panelach oraz aktualnie włączonym panelu. Wpasowuje się to idealnie w nasze zapotrzebowanie, ponieważ każda zakładka reprezentuje inny, lecz integralny aspekt edycji treningu.

Pierwsza zakładka "Trening" odpowiada za logiczny układ sesji oddechowej. Reprezentowana jest przez \texttt{ReorderableListView}, w którym wyświetlane są kolejne etapy treningu \\(\texttt{TrainingStageTile}). Każdy kafel umożliwia zmianę nazwy, liczby powtórzeń i przyrostu czasów, a także usuwanie całego etapu treningu po potwierdzeniu w oknie dialogowym. Wewnątrz każdego etapu treningu umieszczona jest lista faz oddechu (wdechu, zatrzymania, wydechu czy regeneracji) reprezentowana przez sekwencję komponentów \texttt{BreathingPhaseTile}. Zarówno etapy treningu w obrębie całego treningu jak i fazy oddechu w obrębie etapu można uporządkować w kolejności zgodnej z upodobaniem użytkownika w trybie „przeciągnij i upuść”. Pola numeryczne panelu edycji etapu treningu (liczba powtórzeń, przyrost) oraz fazy oddechu (czas trwania) wykorzystują \texttt{TextEditingController} oraz \texttt{FocusNode}, aby zatwierdzać poprawne wartości dopiero po utracie skupienia na danym elemencie wprowadzania danych, a dodatkowe przyciski +/- umożliwiają inkrementalne korekty w krokach od 0{,}1 do 0{,}5 s. Dodawanie nowych faz i etapów wywołuje przewinięcie listy oraz zdejmuje fokus z aktywnych pól, co zapobiega konfliktom z klawiaturą ekranową. Przy próbie opuszczenia ekranu edycji treningu komponent opakowujący \texttt{WillPopScope} sprawdza, czy wszystkie etapy treningu zawierają przynajmniej jedną fazę, i w razie potrzeby wyświetla komunikat z możliwością wyboru automatycznego usunięcia pustych etapów treningu lub powrotem do edycji.

Druga zakładka skupia się na warstwie dźwiękowej i korzysta ze współdzielonego obiektu \texttt{Sounds}. Użytkownik może ustawić sygnał odliczania, sposób odtwarzania komunikatów o nadejściu kolejnej fazy (globalnie, per faza) oraz zdefiniować tło muzyczne w jednym z trzech zakresów: globalnym, per faza albo per etap. W trybie globalnym wykorzystywany jest komponent \texttt{PlaylistEditor}, który obsługuje listy odtwarzania poprzez przeciąganie pozycji, usuwanie oraz dodawanie nowych ścieżek z okna \texttt{AudioSelectionPopup}. Dla wariantu per etap zastosowano \texttt{StagePlaylistsEditor}, mapujący listy na identyfikatory etapów (\texttt{TrainingStage.id}). Użytkownik może importować własne pliki audio poprzez \texttt{file\_picker}; pliki są zapisywane w lokalnej bazie i natychmiast dostępne w oknie wyboru, gdzie można je odsłuchiwać dzięki \texttt{SingleSoundMana-} \texttt{ger}. Dodatkowo przewidziano osobne pola dla muzyki przygotowawczej i końcowej oraz sekcję kontroli dudnień binauralnych, która po aktywacji blokuje klasyczne tło i pozwala regulować częstotliwości składowych suwakami.

Zakładka „Inne” obejmuje elementy opisowe: pole \texttt{TextField} do wprowadzenia opisu, a także licznik czasu przygotowawczego z własnym formatterem cyfr oraz walidacją wartości minimalnej. Wartości te trafiają do obiektu \texttt{Settings}, dzięki czemu są dostępne zarówno w podglądzie treningu, jak i w trakcie wykonywania sesji. Cały edytor jest przetłumaczony z użyciem \texttt{TranslationProvider}, co umożliwia dynamiczne zmiany języka i konsekwentne stosowanie lokalnych etykiet w dialogach, walidatorach czy komunikatach ostrzegawczych. Dzięki temu moduł edytora zachowuje spójność wizualną i logiczną z resztą aplikacji oraz zapewnia użytkownikowi poczucie kontroli nad każdym aspektem treningu bez konieczności przełączania kontekstu.

\section{Importowanie oraz eksportowanie treningów}

Funkcjonalność importu i eksportu danych w aplikacji ReSpire została zaprojektowana w celu umożliwienia użytkownikom łatwej wymiany konfiguracji treningowych pomiędzy różnymi urządzeniami oraz tworzenia kopii zapasowych swoich ustawień. Moduł ten jest kluczowy dla zapewnienia przenośności danych w systemie, który nie polega na centralnym serwerze chmurowym do synchronizacji treści.

Architektura rozwiązania opiera się na dedykowanym serwisie \texttt{TrainingImportExportService}, który pełni rolę fasady dla wszystkich operacji wejścia-wyjścia związanych z plikami treningowymi. Implementacja wykorzystuje bibliotekę \texttt{file\_picker} do obsługi natywnych, systemowych okien dialogowych wyboru i zapisu plików, zapewniając spójne doświadczenie użytkownika niezależnie od platformy (Android/iOS). Do serializacji i deserializacji obiektów domenowych wykorzystano standardową bibliotekę \texttt{dart:convert}, wspieraną przez pomocniczą klasę \texttt{TrainingJsonConverter}.

Proces eksportu danych może zostać zainicjowany w dwóch miejscach w aplikacji. Pierwsza opcja to naciśnięcie przycisku eksportu dostępnego z poziomu widoku szczegółów treningu (\texttt{TrainingPage}), pozwala na wyeksportowanie pojedynczej konfiguracji. W tym przypadku nazwa pliku jest generowana dynamicznie na podstawie tytułu treningu, po uprzedniej sanityzacji znaków specjalnych. Drugą możliwością jest, zaimplementowany na ekranie głównym (\texttt{HomePage}), przycisk umożliwiający masowy eksport wielu treningów jednocześnie. Użytkownik, korzystając z trybu wyboru, zaznacza poprzez dotknięcie interesujące go pozycje, które następnie są pakowane w zbiorczą strukturę JSON. Proces wybierania został celowo zaprojektowany w sposób zbliżony do zaznaczania plików w managerach plików systemów moiblnych. Plik wynikowy otrzymuje nazwę zawierającą znacznik czasu, co ułatwia katalogowanie kopii zapasowych.

Struktura pliku eksportowego JSON jest kompletnym odzwierciedleniem modelu danych aplikacji. Zawiera ona metadane (tytuł, opis), pełną hierarchię etapów (\texttt{TrainingStage}) wraz z fazami oddechowymi (\texttt{BreathingPhase}) i ich parametrami czasowymi, a także szczegółową konfigurację ustawień (\texttt{Settings}) oraz mapę dźwięków (\texttt{Sounds}). Dzięki temu, wyeksportowany plik jest samowystarczalną jednostką informacji, możliwą do odtworzenia w dowolnej innej instancji aplikacji.

Proces importu charakteryzuje się elastycznością dzięki dwóm wariantom eksportu. Metody parsujące w klasie \texttt{TrainingJsonConverter} zostały zaprojektowane tak, aby rozpoznawać i poprawnie przetwarzać zarówno pojedyncze obiekty treningów, jak i listy obiektów lub struktury opakowane (używane przy eksporcie masowym). Po wczytaniu danych następuje proces walidacji oraz integracji, w ramach którego odtwarzane są powiązania do zasobów dźwiękowych metodą \texttt{updateSounds()}. Poprawnie zweryfikowane treningi są następnie dodawane do lokalnej bazy danych Hive, a interfejs użytkownika jest natychmiastowo aktualizowany.


\subsection{Menu - trening}
\subsection{Menu - dźwięki}
\subsection{Menu - inne}

\section{Przebieg treningu \Hania}
\subsection{Klasa TrainingParser}\label{subsec:TrainingParser}
Klasa \texttt{TrainingParser} powstała w celu przekształcenia hierarchicznych danych treningowych pobranych z lokalnej bazy danych Hive w ciąg występujących po sobie faz, oparty na skonfigurowanym uprzednio przez użytkownika wzorcu oddychania. Jej zadaniem jest zwracanie kolejnych faz do obiektu \texttt{TrainingController}. Dzięki temu logika przełączania faz i powtórzeń jest odseparowana od interfejsu. 

Konstruktor jako parametr przyjmuje obiekt klasy \texttt{Training} i zapisuje do zmiennej \textit{currentTrainingStage} pierwszy etap treningu. Fragment realizujący tą funkcjonalność przedstawiono poniżej w kodzie \ref{code/parser/constructor}.

\begin{figure}[h]
\centering
\begin{lstlisting}[caption={TrainingParser - konstruktor}, label={code/parser/constructor}]
  Training training;
  TrainingStage currentTrainingStage;
  late breathing_phase.BreathingPhase currentBreathingPhase;

  TrainingParser({required this.training})
      : currentTrainingStage = training.trainingStages[0];
\end{lstlisting}
\end{figure}

\newpage
Zadaniem funkcji \texttt{nextInstruction}, przedstawionej na kodzie \ref{code/parser/nextInstruction}, jest zwrócenie danych dotyczącej kolejnej fazy oddechowej w postaci mapy \texttt{Map<String, dynamic>} lub wartości \texttt{null} w~przypadku zakończenia całego treningu. 

W pierwszej kolejności analizowany jest aktualny indeks fazy oddechowej (\textit{breathingPhaseID}). Jeżeli wskazuje on ostatnią fazę w bieżącym etapie, oznacza to zakończenie jednego pełnego cyklu etapu. W takim przypadku indeks fazy jest zerowany (\textit{breathingPhaseID = 0}), a licznik wykonanych powtórzeń w etapie (\textit{doneReps}) zwiększany jest o jeden. Następnie sprawdzana jest liczba wykonanych powtórzeń w odniesieniu do wartości zdefiniowanej w obiekcie etapu (\textit{currentTrainingStage.reps}). W przypadku jej osiągnięcia następuje przejście do kolejnego etapu treningu poprzez inkrementację indeksu \textit{trainingStageID}. Jeżeli po tej operacji indeks ten osiągnie wartość równą liczbie wszystkich etapów w strukturze treningu, funkcja zwraca \textit{null}, sygnalizując zakończenie sesji. W przeciwnym razie wczytywany jest nowy etap (\textit{currentTrainingStage = training.trainingStages[trainingStageID]}), a licznik \textit{doneReps} zostaje zresetowany do zera. Gdy aktualna faza nie była ostatnią w cyklu, indeks \textit{breathingPhaseID} jest jedynie zwiększany o jeden. Po ustaleniu poprawnego indeksu do zmiennej \textit{currentBreathingPhase} przypisywana jest odpowiadająca mu faza oddechowa. Kolejnym etapem jest obliczenie rzeczywistego czasu trwania fazy z uwzględnieniem mechanizmu progresji. 

Na podstawie obliczonego czasu tworzona jest zmienna \textit{progressedBreathingPhase}, w której pole \textit{duration} przyjmuje wartość \textit{durationSeconds}, natomiast pozostałe atrybuty (\textit{breathingPhaseType}, \textit{breathType}, \textit{breathDepth}, \textit{sounds}) są kopiowane z obecnej fazy oddechowej.

Funkcja zwraca mapę zawierającą następujące klucze:
\begin{itemize}
    \item \textit{breathingPhase} - pełny obiekt fazy,
    \item \textit{remainingTime} - czas trwania fazy wyrażony w milisekundach,
    \item \textit{trainingStageName} - nazwę aktualnego etapu treningu.
\end{itemize}

W ten sposób \textit{nextInstruction()} pełni rolę centralnego mechanizmu sterującego przebiegiem treningu oddechowego, zapewniając poprawne przechodzenie pomiędzy fazami i etapami oraz automatyczne zwiększanie trudności zgodnie z zaimplementowanym modelem progresji liniowej.

\newpage
\begin{figure}[h]
\centering
\begin{lstlisting}[caption={TrainingParser - pobranie instrukcji}, label={code/parser/nextInstruction}]
  Map<String, dynamic>? nextInstruction() {
    if (breathingPhaseID == currentTrainingStage.breathingPhases.length - 1) {
      breathingPhaseID = 0;
      doneReps++;

      if (doneReps == currentTrainingStage.reps) {
        trainingStageID++;
        if (trainingStageID == training.trainingStages.length) {
          return null;
        } else {
          currentTrainingStage = training.trainingStages[trainingStageID];
          doneReps = 0;
        }
      }
    } else {
      breathingPhaseID++;
    }

    currentBreathingPhase = currentTrainingStage.breathingPhases[breathingPhaseID];

    double durationSeconds = currentBreathingPhase.duration + (currentTrainingStage.increment * doneReps);

    final progressedBreathingPhase = breathing_phase.BreathingPhase(
      duration: durationSeconds,
      breathingPhaseType: currentBreathingPhase.breathingPhaseType,
      breathType: currentBreathingPhase.breathType,
      breathDepth: currentBreathingPhase.breathDepth,
      sounds: currentBreathingPhase.sounds,
    );

    return {
      "breathingPhase": progressedBreathingPhase,
      "remainingTime": (durationSeconds * 1000).truncate(),
      "trainingStageName": currentTrainingStage.name,
    };
  }
\end{lstlisting}
\end{figure}

\subsection{TrainingController} \label{subsec:TrainingController}
\subsection{AnimatedCircle} \label{subsec:AnimatedCircle}
Komponent \texttt{AnimatedCircle} odpowiada za wizualizację przebiegu fazy oddechowej w postaci animowanego koła. Wdech powoduje zwiększanie jego promienia, wydech - zmniejszanie, natomiast fazy retencji i regeneracji utrzymują stały rozmiar.

Obiekt przyjmuje dwa parametry - obiekt typu \texttt{BreathingPhase?}, który reprezentuje aktualną fazę oddechową bądź wartość null w przypadku zakończenia treningu oraz obiekt typu \texttt{bool} \textit{isPaused}, reprezentujący stan wstrzymania treningu. Framgment tworzenia klasy przedstawiony został w kodzie \ref{code/circle/constructor}.

\begin{figure}[h]
\centering
\begin{lstlisting}[caption={AnimatedCircle - konstruktor}, label={code/circle/constructor}]
  final breathing_phase.BreathingPhase? breathingPhase;
  final bool isPaused;

  const AnimatedCircle({super.key, required this.breathingPhase, required this.isPaused});
\end{lstlisting}
\end{figure}

Po utworzeniu obiektu na początku zostaje obliczona wartość początkowa czasu trwania animacji, która zapisywana jest w zmiennej \textit{duration} na podstawie czasu trwania danej fazy. Następnie inicjowany jest kontroler animacji \textit{controller} oraz animacja zmiany promienia koła \textit{circleAnimation}. Kontrolwe ustawiany jest na stan początkowy (\textit{controller.value=0.0}). Jeżeli trening nie jest wstrzymany i dostępna jest faza oddechowa, uruchamiana jest odpowiednia animacja - rosnąca dla wdechu lub malejąca dla wydechu. Fragment funkcji init przedstawiono w kodzie \ref{code/circle/init}.

\begin{figure}[h]
\centering
\begin{lstlisting}[caption={AnimatedCircle - fragment funkcji init}, label={code/circle/init}]
  @override
  void initState() {
    super.initState();

    duration = widget.breathingPhase == null ? 0 : (widget.breathingPhase!.duration * 1000).toInt();

    _controller = AnimationController(
      duration: Duration(milliseconds: duration),
      vsync: this,
    );

    _circleAnimation = Tween<double>(begin: 125.0, end: 300.0).animate(
      CurvedAnimation(parent: _controller, curve: Curves.easeInOut),
    );
    _controller.duration = Duration(milliseconds: duration);

    _controller.value = 0.0;

    if (!widget.isPaused && widget.breathingPhase != null) {
      if (widget.breathingPhase!.breathingPhaseType == breathing_phase.BreathingPhaseType.inhale) {
        _controller.forward(from: 0.0);
      } else if (widget.breathingPhase!.breathingPhaseType == breathing_phase.BreathingPhaseType.exhale) {
        _controller.reverse(from: 1.0);
      }
    }
  }
\end{lstlisting}
\end{figure}

Zachowanie koła zależne jest od jego poprzedniego stanu i zmiany parametrów wejściowych. Jeżeli faza oddechowa sie zmieniła względem ostatniego stanu następuje reakcja zmiany animacji. Ponownie obliczany jest czas trwania animacji (zmienna \textit{duration}) i ustawiane zostaje poprawne zachowanie - wzrost promienia dla wdechu i jego zmniejszenie dla wydechu. Aktualizację stanu przedstawiono w kodzie \ref{code/circle/update}

\begin{figure}[h]
\centering
\begin{lstlisting}[caption={AnimatedCircle - aktualizacja stanu}, label={code/circle/update}]
    if (widget.breathingPhase != oldWidget.breathingPhase && widget.breathingPhase != null) {
      log("${widget.breathingPhase?.breathingPhaseType.name}");

      duration = widget.breathingPhase == null ? 0 : (widget.breathingPhase!.duration * 1000).toInt();
      _controller.duration = Duration(milliseconds: duration);

      if (!widget.isPaused && widget.breathingPhase != null) {
        if (widget.breathingPhase!.breathingPhaseType == breathing_phase.BreathingPhaseType.inhale) {
          _controller.forward(from: 0.0);
        } else if (widget.breathingPhase!.breathingPhaseType == breathing_phase.BreathingPhaseType.exhale) {
          _controller.reverse(from: 1.0);
        }
      } else {
        _controller.stop();
      }
    }
\end{lstlisting}
\end{figure}

Koło \texttt{AnimatedCircle} reaguje również na zmianę parametru \textit{isPaused}, która określa czy trening został zatrzymany. Jeśli tak, animacja zostaje wstrzymana, w innym wypadku jeśli trening był zatrzymany i zostawł wznowiony, animacja zostaje kontynuowana w kierunku wynikającym z~bieżącej fazy. Fragment ten przedstawiono w kodzie \ref{code/circle/pause}.

\begin{figure}[h]
\centering
\begin{lstlisting}[caption={AnimatedCircle - reakcja na pauzę i wznowienie}, label={code/circle/pause}]
if (widget.isPaused && !oldWidget.isPaused) {
      _controller.stop();
    } else if (!widget.isPaused && oldWidget.isPaused) {
      if (widget.breathingPhase != null) {
        if (widget.breathingPhase!.breathingPhaseType == breathing_phase.BreathingPhaseType.inhale) {
          _controller.forward();
        } else if (widget.breathingPhase!.breathingPhaseType == breathing_phase.BreathingPhaseType.exhale) {
          _controller.reverse();
        }
      }
    }
\end{lstlisting}
\end{figure}

Dodatkowo utworzony został obiekt łączący \texttt{AnimatedCircle} oraz dwa koła statyczne typu \texttt{Container} wizualizujące maksymalną i minimalną wartość promienia koła \texttt{AnimatedCircle}, by umożliwić użytkownikowi lepiej ocenić przebieg wdechu i wydechu.

\subsection{InstructionSlider}
\texttt{InstructionSlider} to animowany komponent odpowiadający za wizualizację ciągu instrukcji dla użytkownika w postaci trzech kafelków reprezentujących:
\begin{itemize}
    \item poprzednią fazę oddechową,
    \item obecną fazę oddechową,
    \item nadchodzącą fazę oddechową.
\end{itemize}
Gdy zachodzi zmiana fazy, cała lista przesuwa się w lewo, a na końcu pojawia się nowy kafelek z~kolejną instrukcją.

Obiekt przyjmuje trzy parametry - \textit{preparationTime}, czyli czas trwania fazy przygotowania przed treningiem, kolejkę \textit{breathingPhasesQueue}, która jest ciągiem faz oddechowych w treningu oraz \textit{change}, która jest wyznacznikiem zmiany obecnie trwającej fazy. Framgment tworzenia klasy przedstawiony został w kodzie \ref{code/slider/constructor}.

\begin{figure}[h] 
\centering
\begin{lstlisting}[caption={InstructionSlider - konstruktor}, label={code/slider/constructor}]
double preparationTime;
Queue<breathing_phase.BreathingPhase?> breathingPhasesQueue = Queue<breathing_phase.BreathingPhase?>();
int change; 

InstructionSlider({super.key,required this.preparationTime,  required this.breathingPhasesQueue, required this.change});
\end{lstlisting}
\end{figure}

Podczas inicjalizacji obiekt tworzy pierwszy kafelek reprezentujący fazę przygotowania. Następnie dodawane są dwie fazy z kolejki faz treningu w tej samej formie . Każdy kafelek przechowuje tekst w postaci instrukcji dla użytkowanika i informację, na której pozycji ma się znaleźć. Fragment dodawania obiektów przedstawiono w kodzie \ref{code/slider/init}.

\begin{figure}[h] 
\centering
\begin{lstlisting}[caption={InstructionSlider - fragment funkcji init}, label={code/slider/init}]
_blocks.add(
  InstructionBlock(
    text: translationProvider.getTranslation(
      "BreathingPage.InstructionSlider.get_ready_block_text") + "\n${widget.preparationTime} s", 
    position: 0.0)
);

addNewBreathingPhase(widget.breathingPhasesQueue.elementAt(1));
addNewBreathingPhase(widget.breathingPhasesQueue.elementAt(2));
\end{lstlisting}
\end{figure}

\texttt{InstructionSlider} korzysta z \texttt{AnimationController}, który przy każdej zmianie fazy przesuwa wszystkie kafelki o jedną pozycję w lewo. Każdy kafelek ma przypisaną pozycję z przedziału od -2 do 2. Wartości skrajne (-2 oraz 2) znajdują się poza obszarem widocznym dla użytkownika i służą jedynie do zapewnienia płynniejszej i estetczniejszej animacji. W centrum ekranu znajdują się trzy środkowe pozycje: -1 odpowiada fazie poprzedniej, 0 fazie bieżącej, a 1 fazie nadchodzącej. Gdy kafelek na pozycji -2 ma zostać przesunięty zostaje usunięty w celach optymalizacyjnych. Dokładną implementację animacji przedstawiono w~kodzie \ref{code/slider/animation}. Dodatkowo wykorzystywany jest mechanizm skalowania środkowego kafelka w celu wizualnego podkreślenia aktualnie trwającej fazy.

\begin{figure}[h] 
\centering
\begin{lstlisting}[caption={InstructionSlider - animacja}, label={code/slider/animation}]
_controller = AnimationController(vsync: this, duration: duration);
_animation = Tween<double>(begin: 0.0, end: -1.0).animate(
  CurvedAnimation(parent: _controller, curve: Curves.easeInOut),
);
_controller.addStatusListener((status) {
  if (status == AnimationStatus.completed) {
    _controller.reset();
    setState(() {
      final removed = _blocks.where((b) => b.position <= -2).toList();
      _blocks.removeWhere((b) => b.position <= -2);
      _blocks.forEach((b) => b.position += _animation.value); 
      _blocks.forEach((b) => b.position -= 1);
    });
  }
});
\end{lstlisting}
\end{figure}

Każda zmiana fazy powoduje uruchomienie aktualizacji stanu komponentu. Mechanizm opiera się na porównaniu poprzedniej i bieżącej wartości parametru \textit{change}. Dzięki temu komponent wykrywa moment rozpoczęcia nowej fazy. Obliczany jest wtedy czas trwania nowej fazy, wywoływana jest animacja oraz dodawany nowy kafelek z instrukcją. Funkcję aktualizacji przedstawiono w kodzie \ref{code/slider/update}.

\begin{figure}[h] 
\centering
\begin{lstlisting}[caption={InstructionSlider - aktualizacja stanu}, label={code/slider/update}]
if(oldWidget.change != widget.change) {
  final int phaseDuration = (widget.breathingPhasesQueue.elementAt(0)?.duration.toInt() != null)
    ? (widget.breathingPhasesQueue.elementAt(0)!.duration * 1000).toInt() - 50 
    : 400;
  _controller.duration = Duration(milliseconds: min(phaseDuration,400));
  _controller.forward();
  if(_blocks.last.text!=translationProvider.getTranslation(
    "BreathingPage.InstructionSlider.ending_tile_text")) {
    addNewBreathingPhase(widget.breathingPhasesQueue.elementAt(2));
  }
}
\end{lstlisting}
\end{figure}

\section{Dźwięki \Jakub} \label{sec:Sounds}

Wysoki stopień konfigurowalności warstwy audio przez użytkownika wymusił dekompozycję logiki biznesowej aplikacji na klasy odpowiedzialne za poszczególne przypadki użycia, postępując zgodnie z zasadą pojedynczej odpowiedzialności. 


\begin{figure}[H]
  \centering
  \includegraphics[width=\textwidth]{diagramy/soundManagers_diagram.png}
  \caption{Diagram klas dźwiękowych}
\end{figure}

\subsection{SoundManager} \label{subsec:SoundManager}
Centralnym elementem modułu audio jest klasa \texttt{SoundManager}. Ze względu na konieczność zapewnienia globalnego punktu dostępu do listy załadowanych plików oraz potrzebę współdzielenia jednej instancji obiektu przez wiele komponentów aplikacji, zastosowano w niej wzorzec projektowy Singleton. W polach wspomnianej instancji przechowywane są listy dźwięków krótkich, jak i długich. Rozróżnienie na typy dźwięków było niezbędne ze względu na odmienne ładowanie i odtwarzanie plików. Krótkie efekty dźwiękowe wymagają minimalnej latencji (czasu reakcji) przy odtwarzaniu, natomiast długie ścieżki dźwiękowe, charakteryzujące się rzadszą rotacją i mniejszą dynamiką zmian, nie podlegają tak rygorystycznym wymogom czasowym przy inicjalizacji. Do osiągnięcia tego wymagania należało zastosować oddzielne konfiguracje odtwarzaczy dźwięków (komponentów klasy \texttt{AudioPlayer}) — low latency mode (tryb minimalnego czasu reakcji) lub media player mode (tryb odtwarzania media). Mimo zastosowania tych trybów wysoki narzut obliczeniowy związany z  częstą inicjalizacją odtwarzaczy uniemożliwił odtwarzanie sygnałów dźwiękowych z wymaganą precyzją. Ograniczenie to zmusiło nas do wprowadzenia mechanizmu puli obiektów — \texttt{AudioPlayerPool} \ref{subsec:AudioPlayerPool}.

 \todoJakub{Rozpisz poniższe}
- fade in/out
- ustawienie by dźwięki mogły lecieć razem

\subsection{SingleSoundManager} \label{subsec:SingleSoundManager}
\subsection{PlaylistManager} \label{subsec:PlaylistManager}
\subsection{AudioPlayerPool} \label{subsec:AudioPlayerPool}
\subsection{textToSpeech} \label{subsec:TTS}

\section{Języki} \label{sec:Languages}

\section{Baza danych}

\chapter{Instrukcja użytkowania}
Celem tego rozdziału jest przedstawienie zbioru funkcji aplikacji ReSpire. W kolejnych podrozdziałach opisano sposób instalacji aplikacji oraz wszystkie ekrany aplikacji zaprezentowane poprzez zrzuty ekranu. Aplikacja jest opisywana i prezentowana w polskiej wersji językowej.

\section{Instalacja aplikacji \Ola}
Na urządzenie mobilne z systemem Android należy pobrać plik instalacyjny aplikacji w formacie \textit{.apk}. Następnie należy uruchomić pobrany plik poprzez ppojedyncze kliknięcie i postępować zgodnie z instrukcjami na ekranie. Na ekranie telefonu może się wtedy pojawić okno z zapytaniem o zezwolenie na instalację aplikacji spoza Sklepu Google Play, jeśli to ustawienie jest wyłączone. Należy wtedy kliknąć przycisk \textit{Ustawienia}, a następnie zezwolić na instalację przesuwając przycisk w prawo. Można tego dokonać również poprzez wejście w odpowiednią zakładkę bezpośrednio z poziomu aplikacji ustawień (\textit{Aplikacje} -> \textit{Specjalny dostęp do aplikacji} -> \textit{Instalowanie nieznanych aplikacji} -> \textit{Files by Google lub inny eksplorator plików zainstalowany na urządzeniu}). Po pojawieniu się okna z zapytaniem o instalację należy kliknąć przycisk \textit{Zainstaluj} i poczekać na zakończenie instalacji. Aplikacja powinna następnie pojawić się na liście aplikacji na urządzeniu i być gotowa do uruchomienia.

\section{Strona główna}
Ekran główny aplikacji składa się z dwóch sekcji: u góry znajduje się pasek z opcjami, a~resztę ekranu zajmują kafelki z~dostępnymi treningami, zgodnie z~RYSUNEK. Na dole ekranu widoczna jest animacja fal.

Na środku paska znajduje się logo aplikacji. Po lewej stronie umieszczona jest ikona symbolizująca tryb zaznaczania, który jest aktywowany po kliknięciu. Tryb zmienia wygląd paska - po lewej stronie znajduje się symbol krzyżyka pozwalający opuścić tryb. Obok znajduje się informacja, że tryb zaznaczania jest aktywowany. Użytkownik może wybrać dowolne treningi z listy poprzez kliknięcie kafelków z~nazwami treningów - zaznaczone elementy oznaczone są ciemną, pogrubioną obwódką lub kilknąć przycisk \textit{Zaznacz wszystkie}, który zaznaczy automatycznie wszystkie dostępne treningi. Po wybraniu co najmniej jednego treningu pojawia się informacja o~ilości wybranych treningów. Treningi można także odznaczyć, jeśli użytkownik zmieni zdanie. Jeśli wybór treningów do eksportu jest satysfakcjonujący należy nacisnąć ikonę z~prawej strony paska. Po wybraniu miejsca eksportu w~eksploratorze plików, który się otworzy, opcjonalnie zmianie nazwy pliku i~kliknięciu przycisku \textit{Zapisz} plik w~formacie JSON zostanie zapisany na urządzeniu. W aplikacji wyświetli się także odpowiedni komunikat mówiący o~tym, czy zapis się powiódł. (RYSUNEK). W przypadku eksportu kilku treningów, zostaną one zapisane w jednym pliku. Opcja eksportu treningu dostępna jest także z poziomu strony treningu, jednak w przeciwności do strony głównej, nie pozwala na grupowy eksport, a jedynie indywidualnego treningu. 

W prawej części paska znajdują się dwie ikony: koło zębate przenoszące użytkownika na stronę ustawień oraz symbol ze~strzałką w~dół umożliwiający import zapisanych treningów. Po klikniknięciu w~ikonę importu następuje przeniesienie do eksploratora plików na urządzeniu z~domyślnie otworzonym folderem \textit{Pobrane} na urządzeniu, skąd można wybrać plik z~zapisanym wcześniej treningiem/treningami.

Na stronie widoczna jest lista treningów użytkownika w postaci klikalnych kafelków z~nazwą treningu i~ozdobnym detalem symbolizującym podmuch powietrza. Naciśnięcie kafelka przenosi na stronę treningu, która zawiera szczegóły, możliwość edycji, usunięcia, eskportu czy rozpoczęcia treningu. Domyślnie, po zainstalowaniu aplikacji, dostępne są 3 predefiniowane treningi, widoczne na RYSUNEK. W przypadku, gdy użytkownik chce dodać swój trening, może to zrobić klikając ikonę plusa w~białym okręgu z~niebieskim cieniem. Zostanie on~wtedy przeniesiony na~stronę konfiguracji treningu, która opisana jest w późniejszym podrozdziale LINK. Po dodaniu indywidualnie skonfigurowanego treningu wyświetli się on na liście.

\section{Strona szczegółów treningu}
Na stronę można przejść, klikając w kafelek z wybranym treningiem na stronie głównej. Na pasku u góry znajduje się tytuł treningu oraz strzałka umożliwiająca powrót na stronę główną. Niżej znajdują się klikalne ikony pozwalające na akcje związane z treningiem. Z lewej strony umieszczona została opcja eksportu, która działa analogicznie do strony głównej, z tą różnicą, że eksportujemy tylko jeden trening. Po prawej natomiast znajduje się opcja usunięcia treningu (oznaczona symbolem kosza na śmieci), a także możliość edycji treningu (symbol ołówka). Przy akcji usuwania wyświetla się okno z prośbą o jej potwierdzenie, co zapobiega przypadkowemu usunięciu treningu. Natomiast po wybraniu opcji edycji aplikacja przenosi użytkownika do opisywanej w kolejnym podrozdziale strony. LINK

Niżej znajduje się sekcja opisu treningu. Można go edytować lub dodać na stronie konfiguracji treningu. Predefiniowane treningi zawierają opisy, jednak nie jest to element wymagany.

Następnym elementem jest \textit{Przegląd treningu}, który po naciśnięciu strzałki po prawej stronie rozwija się, pokazując wszystkie etapy treningu wraz z ich nazwami, fazami oddechowymi i długościami ich trwania, a także liczbę powtórzeń oraz przyrost. Dzięki takiemu podsumowaniu użytkownik może łatwo przejrzeć strukturę treningu.

Na stronie znajduje się również przycisk \textit{Rozpocznij trening}, który przenosi użytkownika na stronę treningu oddechowego, rozpoczynając ćwiczenie.

Dodatkowym elementem strony jest animacja delikatnie falującej łódki.

\section{Strona konfiguracji treningu}
U góry strony konfiguracji treningu, na pasku, umieszczona jest nazwa treningu (przy tworzeniu nowego treningu jest ona nadawana automatycznie), strzałka umożliwiająca powrót do strony ze szczegółami treningu oraz ikona ołówka, po kliknięciu której wyświetla się okno edycji nazwy treningu. Po wprowadzeniu wybranej nazwy należy ją zatwierdzić przyciskiem “Zapisz”, można także wyjść bez zapisu po kliknięciu “Anuluj”. Długość nazwy treningu ograniczona jest do 15 znaków, po osiągnięciu tego limitu dalsze znaki nie będą wpisywane w pole edycji. Na dole, po jego prawej stronie, znajduje się licznik znaków, tak aby użytkownik wiedział, ile znaków pozostało jeszcze do wykorzystania. 

Konfigurator składa się z trzech zakładek: \textit{Trening}, \textit{Dźwięki} oraz \textit{Inne}, tak jak na widocznym RYSUNEK.

\subsection{Zakładka Trening}
Pierwsza z zakładek - \textit{Trening} - służy do tworzenia struktury nowego treningu lub edycji istniejącego. Przyciśnięcie przycisku \textit{Dodaj etap treningu} umożliwia użytkownikowi utworzenie i dołączenie nowego etapu do treningu, co skutkuje pojawieniem się na ekranie kafelka reprezentującego uwtorzony etap pod istniejącymi etapami. Można zmienić kolejność kafelków poprzez naciśnięcie i przytrzymanie symbolu dwóch równoległych kresek umieszczonego po lewej stronie, a następnie przesunięcie palcem po ekranie w górę lub dół, do momentu gdy kafelek znajdzie się w pożądanym miejscu. Każdy kafelek zawiera nazwę etapu, która jest dodawana domyślnie przy jego tworzeniu. Może ona zostać edytowana poprzez naciśnięcie na pole z nazwą znajdującą się w zaokrąglonej ramce, a następnie wprowadzenie zmian przy pomocy klawiatury, która pokaże się na ekranie. Nazwa etapu jest ograniczona do 25 znaków. Po osiągnięciu limitu, użytkownik nie będzie miał możliwości wpisania więcej znaków. Licznik znaków wyświetlany jest pod polem z nazwą. Usunięcie etapu jest możliwe poprzez kliknięcie ikony kosza na śmieci znajdującej się po prawej stronie pola z nazwą treningu, a następnie potwierdzenie wykonania akcji klikając przycisk \textit{Usuń} w okienku, które się pokaże. Usuwanie można także anulować klikając przycisk \textit{Anuluj}. Poniżej znajdują się pola dedykowane liczbie powtórzeń danego etapu oraz przyrostowi wyrażonemu w sekundach. Liczba powtórzeń definiuje, ile razy po sobie będzie odtwarzany dany etap, natomiast przyrost - o ile sekund będzie dłuższa każda faza oddechowa w kolejnych powtórzeniach. Wartości można edytować poprzez naciśnięcie ikon plusa / minusa (odpowiednio zwiększenie / zmniejszenie wartości) lub klikając w pole z aktualną liczbą, co spowoduje pokazanie klawiatury na ekranie i umożliwi użytkownikowi wpisanie wybranej wartości. W przypadku używania przycisków plusa lub minusa dla powtórzeń wartość będzie się zmieniać o 1, natomiast dla przyrostu - o 0,1 sekundy w przedziale od 0 do 1 sekundy lub o 1 sekundę dla wartości większych niż 1 sekunda. Za pomocą przycisku \textit{Dodaj fazę oddechu} użytkownik może dodać fazy oddechu do etapu, które wyświetlane są w postaci listy kafelków. Każda faza oddechu zawiera parametry takie jak czas trwania oraz typ. Czas trwania wyrażany jest w sekundach i może być zmodyfikowany poprzez naciśnięcie symbolu plus / minus (odpowiednio zwiększenie / zmniejszenie) lub kliknięcie w pole z aktualną wartością i wpisanie żadanej wartości z klawiatury, która się pokaże na ekranie. Wartości zmieniają się o 0,1 sekundy w zakresie od 0 do 1 sekundy, natomiast później o pół sekundy. W przypadku wpisywania z klawiatury, jeśli podana wartość ma większą dokładność niż obsługiwana, wartość jest automatycznie zaokrąglana. Przykładowo, jeśli użytkownik wpisze wartość 4,8 sekundy, zostanie ona zaokrąglona do wartości 5 sekund. Typ można ustawić natomiast poprzez kliknięcie w pole pod napisem \textit{Typ} - pojawia się wówczas rozwijana lista, z której użytkownik może wybrać żądany typ po kliknięciu w wybraną opcję. Fazę można usunąć poprzez kliknięcie ikony kosza na śmieci umiejszonego po prawej stronie kafelka, a następnie potwierdzenia akcji za pomocą przycisku \textit{Usuń}. Można także zmienić kolejność faz poprzez naciśnięcie i przytrzymanie symbolu dwóch równoległych linii umieszczonego po lewej stronie kafelka, a następnie przesunięcie palcem w górę lub dół ekranu, tak aby kafelek znalazł się w pożądanym miejscu, analogicznie jak w przypadku etapów. Dostępna jest także opcja zwinięcia kafelka za pomocą strzałki w górę umieszczonej z jego prawej strony - nie wyświetlana jest wtedy lista faz oddechu dla danego treningu. Kafelek można z powrotem rozwinąć klikając strzałkę w dół znajdującą się w tym samym miejscu, co uprzednio strzałka w górę.

\subsection{Zakładka Dźwięki}
Zakładka \textit{Dźwięki} zawiera zbiór ustawień związanych z oprawą dźwiękową treningu. Składa się z trzech sekcji: \textit{Dźwięki treningu}, \textit{Muzyka treningu} oraz \textit{Dźwięki binauralne}. Przy tworzeniu nowego treningu ustawione są dźwięki domyślne, natomiast można je zmienić, tak jak zostało opisane poniżej.

Sekcja \textit{Dźwięki treningu} składa się z trzech części. Pierwsza z nich to \textit{Odliczanie}, zawierająca dźwięk odliczania czasu trwania faz oddechowych, przygotowania oraz zakończenia. Po kliknięciu w napis obok symbolu nutki 

W sekcji\textit{Muzyka treningu} również znajdują się trzy części.

Sekcja \textit{Dźwięki binauralne} domyślnie jest wyłączona. Można ją aktywować za pomocą kliknięcia w przełącznik znajdujący się po prawej stronie napisu \textit{Włącz dźwięki binauralne}. Istotne jest fakt, że w przypadku włączenia tej opcji automatycznie nastąpi wyłączenie opcji muzyki w tle - zostanie ona zablokowana w interfejsie i oznaczona jaśniejszym kolorem, jak na RYSUNEK. Po aktywacji, sekcja się rozwija i pokazują się dwa suwaki, za pomocą których użytkownik może ustawić częstwotliwości dudnienia synchronicznego (czyli dźwięku binauralnego) dla prawego oraz lewego ucha. Sumaryczna częstotliwość uderzenia wyświetlana jest na dole sekcji.

\subsection{Zakładka Inne}
W zakładce \textit{Inne} znajdują się pozostałe ustawienia - możliwe jest dodanie nowego lub modyfikacja istniejącego opisu treningu oraz ustawianie długości trwania przygotowania. Należy to zrobić używając przycisków plusa / minusa (odpowiednio zwiększenie / zmniejszenie wartości) lub klikając w okno z aktualnie ustawioną liczbą, a następnie wpisując wybraną liczbę z klawiatury, która się pokaże.

Wszystkie zmiany wprowadzone do konfiguratora zostaną zapisane automatycznie po jego opuszczeniu (czyli kliknięciu strzałki w lewym górnym rogu paska strony). W przypadku próby zapisania treningu z pustymi etapami użytkownik otrzyma komunikat widoczny na RYSUNEK.

\section{Strona treningu oddechowego}
Podczas wczytywania strony treningu oddechowego może się pojawić komunikat informujący o ładowaniu dźwięków. U góry strony widoczny jest tytuł odtworzonego treningu, a także strzałka umożliwiająca powrót do strony szczegółów treningu oraz po prawej stronie przycisk pauzy, jeśli użytkownik chce wstrzymać trening. W celu wznowienia treningu należy nacisnąć ikonę odtwarzania, która pojawi się w miejsce ikony pauzy lub napis \textit{Wznów} w środku opisanej poniżej animacji.

Podczas treningu wyświetlana jest nazwa aktualnego etapu treningu, a poniżej niej - informacja na którym etapie z ilu łącznie jest użytkownik. Na ekranie znajduje się także karuzela z kafelkami, które przesuwają się wraz z przbiegiem treningu. Kafelki zawierają nazwę kroku (fazy oddechowej, rozpoczęcia lub zakończenia treningu) oraz czas jego trwania. Centralny, największy kafelek symbolizuje obecnie trwający krok, kafelek na lewo od niego - poprzedni krok, a kafelek na prawo od centralnego - kolejny krok. Dzięki temu użytkownik może śledzić przebieg treningu oraz przygotować się na nadchodzący krok. Poniżej karuzeli znajduje się licznik wszystkich kroków, co pozwala mniej więcej zorientować się, jak daleko jest od początku treningu.

Głównym elementem strony jest animacja koła, które zmienia się zgodnie z przebiegiem treningu. Podczas wdechu koło powiększa się, a podczas wydechu - pomniejsza. W czasie trwania fazy regneracji lub wstrzymania koło jest natomiast statyczne. Obrazuje to użytkownikowi, jaką fazę oddechową powinien obecnie wykonywać. Na środku animacji znajduje się także licznik czasu przeznaczonego na dany krok.

Jeśli użytkownik kliknie strzałkę powrotu do strony szczegółów treningu, zostanie zapytany o powterdzenie swojej akcji, w celu zapobiegnięcia przypadkowemu opszczeniu treningu.

\section{Strona ustawień}
Strona ustawień zawiera dwie sekcje: wybór języka aplikacji oraz notatkę o ReSpire informującą, jaki jest cel aplikacji. W celu zmiany języka aplikacji należy nacisnąć strzałkę obok informacji o obecnie ustawionym języku, a następnie dokonać wyboru poprzez kliknięcie jednej z dwóch opcji - języka angielskiego lub języka polskiego. RYSUNEK wersja polska aplikacji, a następnie wersja angielska po zmianie języka na przykładzie strony ustawień.

\chapter{Testowanie aplikacji i wnioski}
\todoHania{SIEROTY!!!!}
W tym rozdziale przedstawione zostały sposoby testowania tworzonej aplikacji mobilnej. Terminologia i podział technik testowych są zgodne z aktualnym syllabusem ISTQB (\textit{ang. International Software Testing Qualifications Board}) Foundation Level (wersja 4.0) oraz polskimi odpowiednikami zawartymi w Glosariuszu SJSI. Ze względu na mały zespół i charakter projektu większość czynności miała charakter manualny i opierała się na wzajemnej weryfikacji członków zespołu i opiekuna pracy. Opis przeprowadzonego testowania podzielono na dwie części - testowanie statyczne i testowanie dynamiczne.

\section{Testowanie statyczne}
Testowanie statyczne stanowiło jeden z kluczowych elementów zapewnienia jakości w niniejszym projekcie. W odróżnieniu od testowania dynamicznego, wszystkie czynności statyczne realizowano bez konieczności uruchamiania kodu wykonawczego. Zamiast tego analizie poddane były artefakty wytwarzane na każdym etapie, w tym: kod źródłowy, modele i diagramy oraz wymagania. Dzięki temu możliwe było wczesne wykrycie błędów, niezgodności, problemów z jakością czy naruszenia standardów. 

Główną formę testowania statycznego stosowanego w projekcie stanowiły przeglądy kodu (\textit{ang. Code Review}). Każdy fragment kodu, niezależnie od tego, czy była to nowa funkcjonalność czy drobna poprawka, musiał przejść nieformalny przegląd przez co najmniej jedną, a najczęściej dwie inne osoby z zespołu. Czynności te odbywały się na bieżąco po każdym przyroście przed dołączeniem (\textit{ang. merge}) do głównej gałęzi roboczej \textit{dev}. Ich celem było wykrycie błędów przed połączeniem z już działającym kodem.

Często odbywały się również przeglądy techniczne, których celem było podjęcie decyzji w sprawie problemu technicznego. W spotkaniach tego typu brał udział cały zespół, co skutkowało większą liczbą wykreowanych pomysłów i prowadziło do osiągnięcia konsensusu. Zdarzało się jednak, iż przegląd odbywał się w formie przeglądu parami (\textit{ang. pair reviewing}), gdy problem wymagał osób dysponujących specyficznymi kwalifikacjami technicznymi lub problem nie wymagał obecności wszystkich członków zespołu.

Wszystkie inne artefakty, takie jak wymagania, modele czy diagramy były wielokrotnie omawiane i poprawiane w trakcie wspólnych sesji. Często dochodziło do kilku iteracji, aż cały zespół w pełni rozumiał i akceptował przedstawione rozwiązanie. Szczególną uwagę zwracano na spójność, brak sprzeczności oraz jednoznaczność opisów.

Dzięki małemu zespołowi i kulturze otwartej komunikacji każdy mógł w dowolnym momencie zweryfikować kod innego członka zespołu i zadać pytanie, zasugerować zmiany lub wskazać potencjalny problem. Ta nieformalna, ale niezwykle częsta sytuacja była jednym z najskuteczniejszych mechanizmów wczesnego wykrywania defektów.
\chapter{Podsumowanie \Karol}

\section{Co się udało}
Napisać inżynierkę B-)

\section{Czego się nie udało}
Wyspać się

\section{Wnioski na przyszłość}
Zacząć szybciej

% ---------------------- Bibliografia -----------------------
\nocite{*} % dodajemy całą zawartość refs.bib
\printbibliography[heading=bibintoc,title={Bibliografia}]
% -----------------------------------------------------------

%------------------------------------------------------------
%	Dodanie wykazu rysunków oraz tabeli

% \renewcommand{\baselinestretch}{1.0}\normalsize	% interlinia w sekcji wykazów
% \addcontentsline{toc}{chapter}{\listfigurename}	% dodanie wykazu rysunków do spisu treści
% \listoffigures									% generacja wykazu rysunków

% \addcontentsline{toc}{chapter}{\listtablename}	% dodanie wykazu tabel do spisu treści
% \listoftables									% generacja wykazu tabel
% \renewcommand{\baselinestretch}{1.3}\normalsize	% powrót do interlinii 1.5


% ---------------------- DODATKI -----------------------
%\chapter*{Dodatek A}
%\addcontentsline{toc}{chapter}{Dodatek A}
% Zależnie od dodatku, zaleca się dodawnie dodatków jako
% dokumenty PDF
% \includepdf{dodatki/dodatek1.pdf}

% -----------------------------------------------------------
\end{document}
%------------------------------------------------------------
			 	%	Koniec pracy dyplomowej  %
%------------------------------------------------------------
