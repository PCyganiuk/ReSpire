\chapter{Implementacja aplikacji ReSpire}

Projekt: Aplikacja została zaprojektowana z myślą o prostocie i intuicyjności użytkowania. 
Projekt w Figmie

\section{Technologie}
Technologie:
Aplikacja została napisana w jęzku Dart we frameworku Flutter. 
Lorem ipsum dolor sit amet, consectetur adipiscing elit. Vivamus elementum arcu nec blandit aliquam. Integer eros dolor, molestie eget dictum quis, luctus sit amet sapien. Proin dignissim felis in ornare volutpat. Morbi vulputate rutrum efficitur. Ut vehicula vehicula metus, et iaculis tortor mattis vel. Nam blandit, arcu quis ultricies blandit, libero ante commodo augue, in accumsan dui leo at orci. Phasellus in augue et velit pulvinar malesuada ut et sem. Nulla vehicula nibh eu odio sollicitudin sagittis. Praesent condimentum semper neque, tincidunt luctus nisl scelerisque sed. Orci varius natoque penatibus et magnis dis parturient montes, nascetur ridiculus mus.

\section{Schemat plików/klas/modułów \Hania\ \Ola\ \Jakub\ \Karol}

\section{Podział prac w zespole \Jakub}
\section{Edycja treningu \Ola}
\subsection{Menu - trening}
\subsection{Menu - dźwięki}
\subsection{Menu - inne}

\section{Przebieg treningu \Hania}
\subsection{Klasa TrainingParser}
Klasa \texttt{TrainingParser} powstała w celu przekształcenia hierarchicznych danych treningowych pobranych z lokalnej bazy danych Hive w ciąg występujących po sobie faz, oparty na skonfigurowanym uprzednio przez użytkownika wzorcu oddychania. Jej zadaniem jest zwracanie kolejnych faz do obiektu \texttt{TrainingController}. Dzięki temu logika przełączania faz i powtórzeń jest odseparowana od interfejsu. 

Konstruktor jako parametr przyjmuje obiekt klasy \texttt{Training} i zapisuje do zmiennej \textit{currentTrainingStage} pierwszy etap treningu. Fragment realizujący tą funkcjonalność przedstawiono poniżej w kodzie \ref{code/parser/constructor}.

\begin{figure}[h]
\centering
\begin{lstlisting}[caption={TrainingParser - konstruktor}, label={code/parser/constructor}]
  Training training;
  TrainingStage currentTrainingStage;
  late breathing_phase.BreathingPhase currentBreathingPhase;

  TrainingParser({required this.training})
      : currentTrainingStage = training.trainingStages[0];
\end{lstlisting}
\end{figure}

\newpage
Zadaniem funkcji \texttt{nextInstruction}, przedstawionej na kodzie \ref{code/parser/nextInstruction}, jest zwrócenie danych dotyczącej kolejnej fazy oddechowej w postaci mapy \texttt{Map<String, dynamic>} lub wartości \texttt{null} w~przypadku zakończenia całego treningu. 

W pierwszej kolejności analizowany jest aktualny indeks fazy oddechowej (\textit{breathingPhaseID}). Jeżeli wskazuje on ostatnią fazę w bieżącym etapie, oznacza to zakończenie jednego pełnego cyklu etapu. W takim przypadku indeks fazy jest zerowany (\textit{breathingPhaseID = 0}), a licznik wykonanych powtórzeń w etapie (\textit{doneReps}) zwiększany jest o jeden. Następnie sprawdzana jest liczba wykonanych powtórzeń w odniesieniu do wartości zdefiniowanej w obiekcie etapu (\textit{currentTrainingStage.reps}). W przypadku jej osiągnięcia następuje przejście do kolejnego etapu treningu poprzez inkrementację indeksu \textit{trainingStageID}. Jeżeli po tej operacji indeks ten osiągnie wartość równą liczbie wszystkich etapów w strukturze treningu, funkcja zwraca \textit{null}, sygnalizując zakończenie sesji. W przeciwnym razie wczytywany jest nowy etap (\textit{currentTrainingStage = training.trainingStages[trainingStageID]}), a licznik \textit{doneReps} zostaje zresetowany do zera. Gdy aktualna faza nie była ostatnią w cyklu, indeks \textit{breathingPhaseID} jest jedynie zwiększany o jeden. Po ustaleniu poprawnego indeksu do zmiennej \textit{currentBreathingPhase} przypisywana jest odpowiadająca mu faza oddechowa. Kolejnym etapem jest obliczenie rzeczywistego czasu trwania fazy z uwzględnieniem mechanizmu progresji. 

Na podstawie obliczonego czasu tworzona jest zmienna \textit{progressedBreathingPhase}, w której pole \textit{duration} przyjmuje wartość \textit{durationSeconds}, natomiast pozostałe atrybuty (\textit{breathingPhaseType}, \textit{breathType}, \textit{breathDepth}, \textit{sounds}) są kopiowane z obecnej fazy oddechowej.

Funkcja zwraca mapę zawierającą następujące klucze:
\begin{itemize}
    \item \textit{breathingPhase} - pełny obiekt fazy,
    \item \textit{remainingTime} - czas trwania fazy wyrażony w milisekundach,
    \item \textit{trainingStageName} - nazwę aktualnego etapu treningu.
\end{itemize}

W ten sposób \textit{nextInstruction()} pełni rolę centralnego mechanizmu sterującego przebiegiem treningu oddechowego, zapewniając poprawne przechodzenie pomiędzy fazami i etapami oraz automatyczne zwiększanie trudności zgodnie z zaimplementowanym modelem progresji liniowej.

\newpage
\begin{figure}[h]
\centering
\begin{lstlisting}[caption={TrainingParser - pobranie instrukcji}, label={code/parser/nextInstruction}]
  Map<String, dynamic>? nextInstruction() {
    if (breathingPhaseID == currentTrainingStage.breathingPhases.length - 1) {
      breathingPhaseID = 0;
      doneReps++;

      if (doneReps == currentTrainingStage.reps) {
        trainingStageID++;
        if (trainingStageID == training.trainingStages.length) {
          return null;
        } else {
          currentTrainingStage = training.trainingStages[trainingStageID];
          doneReps = 0;
        }
      }
    } else {
      breathingPhaseID++;
    }

    currentBreathingPhase = currentTrainingStage.breathingPhases[breathingPhaseID];

    double durationSeconds = currentBreathingPhase.duration + (currentTrainingStage.increment * doneReps);

    final progressedBreathingPhase = breathing_phase.BreathingPhase(
      duration: durationSeconds,
      breathingPhaseType: currentBreathingPhase.breathingPhaseType,
      breathType: currentBreathingPhase.breathType,
      breathDepth: currentBreathingPhase.breathDepth,
      sounds: currentBreathingPhase.sounds,
    );

    return {
      "breathingPhase": progressedBreathingPhase,
      "remainingTime": (durationSeconds * 1000).truncate(),
      "trainingStageName": currentTrainingStage.name,
    };
  }
\end{lstlisting}
\end{figure}

\subsection{TrainingController}
\subsection{Kółko}
\subsection{Instrukcja}

\section{Dźwięki \Jakub}
\subsection{SoundManager}
\subsection{textToSpeach}

\section{Języki}

\section{Baza danych}

\section{Import i export \Karol}







