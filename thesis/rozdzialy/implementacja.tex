\chapter{Implementacja aplikacji mobilnej ReSpire}
Projekt został wykonany w środowisku Flutter za pomocą zestawu narzędzi dla programistów Flutter w wersji 3.27.2 wspierającego język Dart w wersji 3.6.1. Ten zestaw narzędzi dla programistów umożliwia tworzenie aplikacji na platformy Android, iOS, Linux, macOS oraz Windows z wykorzystaniem jednej bazy kodu źródłowego. Wybór Fluttera był podyktowany jego wydajnością, bogatym zestawem narzędzi do tworzenia interfejsów użytkownika oraz szeroką społecznością wsparcia. Zapewnie trwałości danych między uruchomieniami zostało zrealizowane przy użyciu lokalnej, nierelacyjnej bazy danych Hive w wersji 2.2.3, skrojonej pod środowisko Flutter oraz język Dart.

\section{Schemat plików/klas/modułów \Hania\ \Ola\ \Jakub\ \Karol}

\section{Ciekawe fragmenty kodu \Hania\ \Ola\ \Jakub\ \Karol}

\section{Edytor treningów}

Edytor treningów został zaimplementowany jako dedykowany widok \texttt{TrainingEditorPage}, który operuje bezpośrednio na obiekcie \texttt{Training}. Struktura treningu przechowywana jest w modelach Hive (\texttt{TrainingStage}, \texttt{BreathingPhase}, \texttt{Sounds}, \texttt{Settings}), dzięki czemu każda modyfikacja wprowadzona w interfejsie natychmiast trafia do tej samej instancji danych, którą aplikacja później serializuje. Cała logika edytora została podzielona na trzy zakładki sterowane przez komponent \texttt{CustomSlidingSegmentedControl}, co pozwoliło oddzielić edycję przebiegu treningu, konfigurację dźwięków oraz ustawienia uzupełniające, utrzymując jednocześnie wspólną nawigację i kontrolę zapisów. Sam \texttt{CustomSlidingSegmentedControl} jest komponentem zainspirowanym szeroko wykorzystywanym w systemie mobilnym Apple iOS oraz platformie technologicznej SwiftUI komponenetem \texttt{segmented}, który umożliwia przełączanie się między różnymi widokami w obrębie jednej strony aplikacji przy jednoczesnej informacji o dostępnych panelach oraz aktualnie włączonym panelu. Wpasowuje się to idealnie w nasze zapotrzebowanie, ponieważ każda zakładka reprezentuje inny, lecz integralny aspekt edycji treningu.

Pierwsza zakładka "Trening" odpowiada za logiczny układ sesji oddechowej. Reprezentowana jest przez \texttt{ReorderableListView}, w którym wyświetlane są kolejne etapy treningu \\(\texttt{TrainingStageTile}). Każdy kafel umożliwia zmianę nazwy, liczby powtórzeń i przyrostu czasów, a także usuwanie całego etapu treningu po potwierdzeniu w oknie dialogowym. Wewnątrz każdego etapu treningu umieszczona jest lista faz oddechu (wdechu, zatrzymania, wydechu czy regeneracji) reprezentowana przez sekwencję komponentów \texttt{BreathingPhaseTile}. Zarówno etapy treningu w obrębie całego treningu jak i fazy oddechu w obrębie etapu można uporządkować w kolejności zgodnej z upodobaniem użytkownika w trybie „przeciągnij i upuść”. Pola numeryczne panelu edycji etapu treningu (liczba powtórzeń, przyrost) oraz fazy oddechu (czas trwania) wykorzystują \texttt{TextEditingController} oraz \texttt{FocusNode}, aby zatwierdzać poprawne wartości dopiero po utracie skupienia na danym elemencie wprowadzania danych, a dodatkowe przyciski +/- umożliwiają inkrementalne korekty w krokach od 0{,}1 do 0{,}5 s. Dodawanie nowych faz i etapów wywołuje przewinięcie listy oraz zdejmuje fokus z aktywnych pól, co zapobiega konfliktom z klawiaturą ekranową. Przy próbie opuszczenia ekranu edycji treningu komponent opakowujący \texttt{WillPopScope} sprawdza, czy wszystkie etapy treningu zawierają przynajmniej jedną fazę, i w razie potrzeby wyświetla komunikat z możliwością wyboru automatycznego usunięcia pustych etapów treningu lub powrotem do edycji.

Druga zakładka skupia się na warstwie dźwiękowej i korzysta ze współdzielonego obiektu \texttt{Sounds}. Użytkownik może ustawić sygnał odliczania, sposób odtwarzania komunikatów o nadejściu kolejnej fazy (globalnie, per faza) oraz zdefiniować tło muzyczne w jednym z trzech zakresów: globalnym, per faza albo per etap. W trybie globalnym wykorzystywany jest komponent \texttt{PlaylistEditor}, który obsługuje listy odtwarzania poprzez przeciąganie pozycji, usuwanie oraz dodawanie nowych ścieżek z okna \texttt{AudioSelectionPopup}. Dla wariantu per etap zastosowano \texttt{StagePlaylistsEditor}, mapujący listy na identyfikatory etapów (\texttt{TrainingStage.id}). Użytkownik może importować własne pliki audio poprzez \texttt{file\_picker}; pliki są zapisywane w lokalnej bazie i natychmiast dostępne w oknie wyboru, gdzie można je odsłuchiwać dzięki \texttt{SingleSoundMana-} \texttt{ger}. Dodatkowo przewidziano osobne pola dla muzyki przygotowawczej i końcowej oraz sekcję kontroli dudnień binauralnych, która po aktywacji blokuje klasyczne tło i pozwala regulować częstotliwości składowych suwakami.

Zakładka „Inne” obejmuje elementy opisowe: pole \texttt{TextField} do wprowadzenia opisu, a także licznik czasu przygotowawczego z własnym formatterem cyfr oraz walidacją wartości minimalnej. Wartości te trafiają do obiektu \texttt{Settings}, dzięki czemu są dostępne zarówno w podglądzie treningu, jak i w trakcie wykonywania sesji. Cały edytor jest przetłumaczony z użyciem \texttt{TranslationProvider}, co umożliwia dynamiczne zmiany języka i konsekwentne stosowanie lokalnych etykiet w dialogach, walidatorach czy komunikatach ostrzegawczych. Dzięki temu moduł edytora zachowuje spójność wizualną i logiczną z resztą aplikacji oraz zapewnia użytkownikowi poczucie kontroli nad każdym aspektem treningu bez konieczności przełączania kontekstu.

\section{Importowanie oraz eksportowanie treningów}

Funkcjonalność importu i eksportu danych w aplikacji ReSpire została zaprojektowana w celu umożliwienia użytkownikom łatwej wymiany konfiguracji treningowych pomiędzy różnymi urządzeniami oraz tworzenia kopii zapasowych swoich ustawień. Moduł ten jest kluczowy dla zapewnienia przenośności danych w systemie, który nie polega na centralnym serwerze chmurowym do synchronizacji treści.

Architektura rozwiązania opiera się na dedykowanym serwisie \texttt{TrainingImportExportService}, który pełni rolę fasady dla wszystkich operacji wejścia-wyjścia związanych z plikami treningowymi. Implementacja wykorzystuje bibliotekę \texttt{file\_picker} do obsługi natywnych, systemowych okien dialogowych wyboru i zapisu plików, zapewniając spójne doświadczenie użytkownika niezależnie od platformy (Android/iOS). Do serializacji i deserializacji obiektów domenowych wykorzystano standardową bibliotekę \texttt{dart:convert}, wspieraną przez pomocniczą klasę \texttt{TrainingJsonConverter}.

Proces eksportu danych może zostać zainicjowany w dwóch miejscach w aplikacji. Pierwsza opcja to naciśnięcie przycisku eksportu dostępnego z poziomu widoku szczegółów treningu (\texttt{TrainingPage}), pozwala na wyeksportowanie pojedynczej konfiguracji. W tym przypadku nazwa pliku jest generowana dynamicznie na podstawie tytułu treningu, po uprzedniej sanityzacji znaków specjalnych. Drugą możliwością jest, zaimplementowany na ekranie głównym (\texttt{HomePage}), przycisk umożliwiający masowy eksport wielu treningów jednocześnie. Użytkownik, korzystając z trybu wyboru, zaznacza poprzez dotknięcie interesujące go pozycje, które następnie są pakowane w zbiorczą strukturę JSON. Proces wybierania został celowo zaprojektowany w sposób zbliżony do zaznaczania plików w managerach plików systemów moiblnych. Plik wynikowy otrzymuje nazwę zawierającą znacznik czasu, co ułatwia katalogowanie kopii zapasowych.

Struktura pliku eksportowego JSON jest kompletnym odzwierciedleniem modelu danych aplikacji. Zawiera ona metadane (tytuł, opis), pełną hierarchię etapów (\texttt{TrainingStage}) wraz z fazami oddechowymi (\texttt{BreathingPhase}) i ich parametrami czasowymi, a także szczegółową konfigurację ustawień (\texttt{Settings}) oraz mapę dźwięków (\texttt{Sounds}). Dzięki temu, wyeksportowany plik jest samowystarczalną jednostką informacji, możliwą do odtworzenia w dowolnej innej instancji aplikacji.

Proces importu charakteryzuje się elastycznością dzięki dwóm wariantom eksportu. Metody parsujące w klasie \texttt{TrainingJsonConverter} zostały zaprojektowane tak, aby rozpoznawać i poprawnie przetwarzać zarówno pojedyncze obiekty treningów, jak i listy obiektów lub struktury opakowane (używane przy eksporcie masowym). Po wczytaniu danych następuje proces walidacji oraz integracji, w ramach którego odtwarzane są powiązania do zasobów dźwiękowych metodą \texttt{updateSounds()}. Poprawnie zweryfikowane treningi są następnie dodawane do lokalnej bazy danych Hive, a interfejs użytkownika jest natychmiastowo aktualizowany.