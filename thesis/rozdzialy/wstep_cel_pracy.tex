\chapter{Wstęp i cel pracy} 

\section{Wprowadzenie (problemy) \Hania}
W ostatnich latach obserwuje się wzrost zainteresowania technologiami wspierającymi samopomoc, zdrowie psychiczne oraz treningi fizjologiczne, w tym trening oddechu. Aplikacje mobilne pełnią w tym zakresie istotną rolę, umożliwiając użytkownikom łatwy dostęp do zindywidualizowanych programów treningowych, monitorowania postępów oraz technik relaksacyjnych.

\section{Motywacje (korzyści) \Hania}
Wsparcie zdrowia psychicznego i fizycznego poprzez ułatwienie wykonywania ćwiczeń oddechowych. 
Możliwość wykorzystania dostępności aplikacji mobilnych 
Trening oddechowy może mieć korzystny wpływ na zdrowie i być pomocny dla szerokiego grona osób. Może on służyć:
\begin{enumerate}
  \item sportowcom - poprawienie wytrzymałości;
  \item pacjentom z zespołami bólowymi i chorobami układu oddechowego takimi jak astma, przewlekła obturacyjna choroba płuc czy wady klatki piersiowej - zwiększenie tolerancji wysiłku, ogólnej wydolności fizycznej oraz maksymalnego poboru tlenu;
  \item osobom z zaburzeniami lękowymi lub doświadczającym przewlekłego stresu - ukojenie lęku i łagodzenie stresu;
  \item osobom zmagającym się z problemami ze snem - poprawa jakości snu;
  \item instrumentalistom dętym - polepszenie zadęcia;
  \item każdej osobie chcącej pracować nad oddechem - polepszenie układu odpornościowego, zwiększenie ilości tlenu w organizmie.
\end{enumerate}

\section{Cel pracy \Hania}
Celem pracy jest opracowanie projektu oraz implementacja wieloplatformowej aplikacji mobilnej służącej do przeprowadzania treningu oddechowego opartego na konfigurowalnych wzorcach oddechowych obejmujących wdech, retencję, wydech i regenerację, z możliwością dostosowania parametrów dźwiękowych oraz językowych (polski i angielski), a także z wbudowaną wizualizacją przebiegu treningu.

\section{Podział prac w zespole \Jakub}



\section{Struktura rozdziałów \Hania}
W celu przejrzystego przedstawienia realizacji projektu i ułatwienia odbioru pracy, została ona podzielona na dziewięć rozdziałów opisanych poniżej.
	
W rozdziale pierwszym przedstawiono wprowadzenie do tematu pracy, przedstawiono problemy i motywacje podjęcia pracy, jej cel oraz podział obowiązków w zespole.
	
Rozdział drugi opisuje aktualny stan wiedzy, obejmujący istniejące rozwiązania aplikacji mobilnych wspierających trening oddechowy oraz aspekty medyczne i sportowe treningu oddechowego. Przedstawiono również opisy podsumowujący ważne elementy i brakujące funkcjonalności w tych rozwiązaniach.
	
W rozdziale trzecim określono główne założenia projektowe, w których przedstawiono grupę docelową, korzyści płynące z projektu oraz zakładane główne funkcjonalności aplikacji. Rozdział ten definiuje również przewidywanych użytkowników oraz podstawowe wymagania wobec systemu.
	
Rozdział czwarty zawiera analizę wymagań, w tym modele przypadków użycia, modele i diagramy klas oraz dźwięków wykorzystywanych w projekcie.

W rozdziale piątym przedstawiono projekt rozwiązania, obejmujący architekturę systemu, logikę aplikacji, interfejs użytkownika oraz strukturę danych.

Rozdział szósty opisuje implementację aplikacji, w tym strukturę plików i modułów oraz wybrane fragmenty kodu, obejmujące najważniejsze elementy projektu. Zwrócono w nim uwagę na zastosowane technologie i narzędzia programistyczne.

W rozdziale siódmym zamieszczono instrukcję użytkownika opisującą sposób korzystania z aplikacji.

Rozdział ósmy poświęcono testowaniu, obejmującemu testy jednostkowe, integracyji modułów, systemowe i użytkowe. Zaprezentowano w nim wyniki testów oraz ocenę poprawności działania aplikacji w różnych scenariuszach użycia.

Dziewiąty rozdział, zawiera podsumowanie pracy, omówienie osiągniętych rezultatów, napotkanych trudności oraz wnioski dotyczące projektu. Wskazano również możliwe kierunki przyszłych usprawnień i rozszerzeń funkcjonalnych systemu.

