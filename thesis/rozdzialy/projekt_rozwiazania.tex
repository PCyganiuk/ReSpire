\chapter{Projekt rozwiązania \Hania\ \Ola\ \Jakub}

\section{Projekt architektury systemu \Ola}


\section{Projekt logiki aplikacji \Hania}

\section{Projekt interfejsu użytkownika \Hania\ \Ola}
Podrozdział ten poświęcony jest projektowaniu interfejsu użytkownika opracowanej aplikacji mobilnej. Głównym celem było stworzenie intuicyjnej, wizualnie estetycznej oraz w pełni responsywnej warstwy użytkownika, zapewniającego pozytywne doświadczenia przy jednoczesnym spełnieniu wcześniej zdefiniowanych wymagań funkcjonalnych i niefunkcjonalnych.
Do~kluczowych wymagań funkcjonalnych interfejsu należały:
\begin{itemize}
    \item przeglądanie i zarządzanie listą treningów,
    \item przebieg treningu oddechowego z licznikiem czasu i wyświetlanymi instrukcjami,
    \item zaawansowane tworzenie oraz edycja treningów oddechowych.
\end{itemize}
Spośród wymagań niefunkcjonalnych szczególnie uwzględniono:
\begin{itemize}
    \item pełną responsywność projektu oraz podejście \textit{mobile-first},
    \item wsparcie dla dwóch wersji językowych (polski i angielski),
    \item wysoką czytelność i estetykę wizualną,
    \item częściową zgodność z wytycznymi dostępności WCAG~2.1 na poziomie~AA.
\end{itemize}

Przy projektowaniu świadomie kierowano się wytycznymi dostępności WCAG~2.1 jako punktem odniesienia. Większość kluczowych zaleceń tego standardu - w tym odpowiedni kontrast tekstu (minimum 4.5:1), widoczny wskaźnik aktywnej klawiatury czy tekstowe alternatywy dla elementów nietekstowych - została zastosowana. Ze względu na prototypowy charakter projektu wykonanego w narzędziu Figma, pełna zgodność z WCAG~2.1~AA nie była celem, a jedynie wskazówką projektową.

Proces projektowania oparto na następujących metodykach i podejściach:
\begin{itemize}
    \item \textit{Mobile-first} - projekt został stworzony skupiając się w pierwszej kolejności na użytkownikach urządzeń mobilnych,
    \item \textit{Design System} - zbudowano spójny system projektowy obejmujący paletę kolorów, typografię i komponenty, gwarantując jednolity wygląd i zachowanie aplikacji,
    \item \textit{Atomic Design} - interfejs skonstruowano w oparcie o hierarchię komponentów.
\end{itemize}

\subsection{Projekt w narzędziu Figma} \label{subsec:Figma}
Do zaprojektowania interfejsu użytkownika wybrano narzędzie Figma, które oferuje zaawansowane prototypowanie, w tym reakcję zmiany widoków po kliknięciu w dany obiekt. Dodatkowo narzędzie zostało wybrane również ze względu na jego znajomość przez członków zespołu, możliwość współpracy i łatwego udostępniania gotowego projektu.

W pierwotnej wersji ekran główny aplikacji stanowiła lista treningów oddechowych prezentowana w formie kafelków. Dodawanie nowego treningu realizowane było za pomocą przycisku pływającego, umieszczonego pod kafelkami. Dostępne były również opcje w prawym górnym rogu. Każdy kafelek treningu obsługiwał gesty przesunięcia w lewo, dając możliwość jego usunięcia, i w prawo - dając możliwość jego edycji. Projekt można zobaczyć na rysunku \ref{img/respire_home}. 
Mechanizm przesuwalnych kafelków okazał się jednak nieintuicyjny dla użytkowników i postanowiono z niego zrezygnować. W finalnej wersji aplikacji każdy kafelek treningu po kliknięciu przenosi użytkownika na dedykowany ekran szczegółów treningu, na którym w wyraźny sposób udostępniono przyciski do edycji, usunięcia i włączenia treningu.

\begin{figure}[H]
\centering
\includegraphics[scale=0.4]{obrazki/projekt_rozwiazania/figma_home.png}
\caption{Pierwszy projekt aplikacji ReSpire - Strona Główna}
\label{img/respire_home}
\end{figure}

Przebieg treningu oddechowego w prototypie aplikacji ReSpire zaprojektowano w sposób maksymalnie intuicyjny i czytelny dla użytkownika. W centralnej części ekranu znajduje się element animowany w postaci koła, które płynnie zmienia swój promień i wyświetla pozostały czas do końca obecnej fazy oddechowej. Bezpośrednio nad nim znajduje się komponent wyświetlający instrukcje w postaci trzech kafelków, przedstawiających kolejno od lewej fazę poprzednią, obecną i następną. Dodatkowo w tym samym obszarze prezentowany był globalny postęp treningu - liczba ukończonych cykli w stosunku do zaplanowanej liczby wszystkich cykli. Kliknięcie w~centralne koło miało powodować natychmiastowe wstrzymanie treningu, a ponowne jego kliknięcie wznawiało sesję od miejsca przerwania. Przy próbie opuszczenia ekranu treningu aplikacja miała wyświetlać potwierdzenie z informacją, iż bieżąca sesja zostanie przerwana i jej postęp nie zostanie zapisany. Projekt przebiegu treningu przedstawiono na rysunku \ref{img/respire_training}. 

\begin{figure}[H]
\centering
\includegraphics[scale=0.45]{obrazki/projekt_rozwiazania/figma_training.png}
\caption{Pierwszy projekt aplikacji ReSpire - przebieg treningu}
\label{img/respire_training}
\end{figure}

Ekran edycji treningu w pierwotym projekcie zawiera w górnej części tytuł, a w dolnej przycisk zapisu \textit{Save}, z którego w wersji ostatecznej zrezygnowano, ze względu na wprowadzenie mechanizmu automatycznego zapisu zmian. Główną część ekranu stanowi panel z trzema zakładkami:
\begin{itemize}
    \item \textit{Training} - sekcja służąca do tworzenia i edycji własnych schematów oddechowych,
    \item \textit{Sound} - ustawienia muzyki i sygnałów dźwiękowych,
    \item \textit{Other} - pozostałe preferencje dotyczące treningu.
\end{itemize}
Wstępny projekt nie zakładał opisu treningu, który został dodany dopiero na późniejszych etapach rozwoju aplikacji. Prototyp przedstawiono na rysunku \ref{img/respire_settings}.

\begin{figure}[H]
\centering
\includegraphics[scale=0.42]{obrazki/projekt_rozwiazania/figma_settings.png}
\caption{Pierwszy projekt aplikacji ReSpire - edycja treningu}
\label{img/respire_settings}
\end{figure}

\newpage
\subsection{Identyfikacja wizualna aplikacji}
W celu zapewnienia spójności wyglądu oraz estetyki aplikacji ReSpire przygotowana została identyfikacja wizualna obejmująca logo, kolory oraz czcionkę.

Logo aplikacji zostało zaprojektowane w narzędziu \textit{Canva}, a część jego elementów była narysowana w edytorze graficznym \textit{AdobeFresco}. Połączenie dwóch różnych czcionek o odmiennych grubościach oraz dodatek grafiki podmuchu i detalu symbolizującego cząsteczkę powietrza nad literą \textit{i} oddaje ducha ReSpire. Logo używane w ekranie ładowania aplikacji oraz ikonie posiada napis umieszczony w dwóch liniach, jak przedstawiono na rysunku \ref{img/respire_logo}. Pomysł umieszczenia go na pasku na ekranie głównym spowodował, że powstała także druga wersja, w której całość mieści się w jednej linii, co pokazano na rysunku \ref{img/logo_w_jednej_linii}.
\begin{figure}[H]
    \centering
    \begin{minipage}{0.48\textwidth}
        \centering
        \includegraphics[scale=0.45]{obrazki/projekt_rozwiazania/logo_biale_tlo.png}
        \caption{Logo ReSpire w oryginalnej wersji}
        \label{img/respire_logo}
    \end{minipage}
    \hfill
    \begin{minipage}{0.48\textwidth}
        \centering
        \includegraphics[scale=0.45]{obrazki/projekt_rozwiazania/logo_poziom.png}
        \caption{Logo przystosowane do użycia na ekranie głównym aplikacji}
        \label{img/logo_w_jednej_linii}
    \end{minipage}
\end{figure}


W oparciu o kolorystykę elementów w logo zostały wybrane 3 główne kolory aplikacji. Są to~odcienie niebieskiego. Kolor ten jest kojarzony z poczuciem spokoju i bezpieczeństwa \cite{blue}. Dzięki temu aplikacja jest spójna kolorystycznie, a także wywołuje przyjemne odczucia w użytkowniku, zapewniając wyciszenie i harmonię. 
Poniżej przedstawiono listę kolorów zapisanych w~dwóch powszechnie stosowanych systemach zapisu barw oraz ich wizualizację:
\definecolor{colA}{HTML}{1A93A8}
\definecolor{colB}{HTML}{32B7CF}
\definecolor{colC}{HTML}{7BDEF0}

\noindent
\colorbox{colA}{\phantom{XXXX}} \quad \textbf{HEX:} \#1A93A8, \textbf{RGB:} (26, 147, 168) \\[3pt]
\colorbox{colB}{\phantom{XXXX}} \quad \textbf{HEX:} \#32B7CF, \textbf{RGB:} (50, 183, 207) \\[3pt]
\colorbox{colC}{\phantom{XXXX}} \quad \textbf{HEX:} \#7BDEF0, \textbf{RGB:} (123, 22, 240)

Jedna z dwóch czcionek użytych w logo - \textit{Glacial Indifference} \cite{font} - została również wykorzystana w~aplikacji, zapewniając jej unikatowy charakter. Jest to darmowy, otwartoźródłowy krój pisma zaprojektowany przez \textit{Hanken Design Co.} z~licencją \textit{SIL OPEN FONT LICENSE Version 1.1} \cite{font_license} pozwalającą na użytek zarówno prywatny, jak i komercyjny. Przykładowy tekst oraz znaki zapisane użytą czcionką przedstawione zostały na rysunku \ref{img/czcionki}. Wykorzystana została także domyślna czcionka \textit{Fluttera} dla aplikacji na systemie \textit{Android} - \textit{Roboto}.
\begin{figure}[H]
    \centering
    \includegraphics[scale=0.25]{obrazki/projekt_rozwiazania/czcionka_pogrubiona.jpg}
    \includegraphics[scale=0.25]{obrazki/projekt_rozwiazania/czcionka_regularna_tekst.jpg}
    \caption{Czcionka \textit{GlacialIndeifference} użyta w logo oraz aplikacji - po lewej w wersji pogrubionej, a~po~prawej w standardowej}
    \label{img/czcionki}
\end{figure}

Elementy charakterystyczne dla wyglądu ReSpire przedstawione zostały na rysunku \ref{img/kafelek_i_przycisk}. Białe kafelki z nazwami treningów na stronie głównej z asymetrycznymi zaokrągleniami oraz symbolami podmuchu stanowią prosty, ale modernistyczny element aplikacji. Przycisk dodawania własnego treningu nawiązuje do detalu cząstki powietrza w logo, czyniąc aplikację wyjątkową. 
\begin{figure}[H]
    \centering
    \includegraphics[scale=0.25]{obrazki/projekt_rozwiazania/Kafelek_i_przycisk.png}
    \caption{Kafelek z nazwą treningu oraz przycisk dodawania nowego poniżej}
    \label{img/kafelek_i_przycisk}
\end{figure}


Wszytskie ikony używane w aplikacji pochodzą z otwartoźródłowej biblioteki \textit{Material Icons} \cite{icons} należącej do \textit{Google}. Są intuicyjne i minimalistyczne, dzięki czemu wygląd aplikacji jest spójny oraz przejrzysty dla użytkownika. Można je wykorzystywać na podstawie licencji \textit{Apache License, Version 2.0} \cite{font_license}.
\begin{figure}[H]
    \centering
    \includegraphics[scale=0.35]{obrazki/projekt_rozwiazania/ikony.png}
    \caption{Przykładowe ikony z biblioteki \textit{Material Icons}}
    \label{img/ikony}
\end{figure}

W aplikacji zastosowano lekkie, łatwo skalowalne animacje w formacie Lottie \cite{animations}. Umożliwia on zapis w formie plików JSON, są bazowane na grafice wektorowej i idealnie nadaje się do aplikacji mobilnych, zapewniając wysoką płynność i niskie wykorzystanie zasobów. Wykorzystane zostały dwie darmowe animacje - fala \cite{wave_animation} wykonana przez \textit{CosmoYo} oraz łódka \cite{boat_animation}, której autorem jest \textit{Sergey Riznyk} na podstawie licencji \textit{Lottie Simple License (FL 9.13.21)} \cite{animations_license}. Zostały one dostosowane do potrzeb aplikacji poprzez między innymi: zmianę kolorów, usunięcie niektórych elementów, spowolnienie tempa odtwarzania czy zmianę orientacji. Animacje ożywiają warstwę prezentacji, dopełniając przestrzenie ekranu głównego i strony ze szczegółami treningu. Dzięki powolnemu tempu wprowadzają spokój.


\section{Projekt struktury danych (treningi) \Jakub}

Model danych oparto na strukturze hierarchicznej, której fundament stanowią trzy kluczowe klasy: \texttt{Trening}, \texttt{EtapTreningu} oraz \texttt{FazaOddechowa}. Główna encja, \texttt{Trening}, składa się z co najmniej jednego \texttt{EtapuTreningu} (relacja jeden do wielu). Każdy etap charakteryzuje się zdefiniowaną liczbą cykli (powtórzeń) oraz parametrem przyrostu czasu. Na najniższym poziomie hierarchii znajduje się \texttt{FazaOddechowa}. Etapy składają się z sekwencji faz o określonym czasie trwania oraz typie, wybieranym ze zbioru zdefiniowanych wartości: wdech, wydech, regeneracja lub~wstrzymanie. Opisywany projekt przedstawia rysunek~\ref{img/training_structure}, natomiast rozwinięcie i implementację tej struktury danych opisano w sekcji \ref{subsec:training_classes}.

\begin{figure}[H]
    \centering
    \includegraphics[width=\textwidth]{obrazki/projekt_rozwiazania/Struktura_treningu.jpg}
    \caption{Struktura treningu}
    \label{img/training_structure}
\end{figure}