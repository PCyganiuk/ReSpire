\chapter{Instrukcja użytkowania}
Celem tego rozdziału jest przedstawienie zbioru funkcji aplikacji ReSpire. W kolejnych podrozdziałach opisano sposób instalacji aplikacji oraz wszystkie ekrany aplikacji zaprezentowane poprzez zrzuty ekranu. Aplikacja jest opisywana i prezentowana w polskiej wersji językowej.

\section{Instalacja aplikacji \Ola}
Na urządzenie mobilne z systemem Android należy pobrać plik instalacyjny aplikacji w formacie \textit{.apk}. Następnie należy uruchomić pobrany plik poprzez ppojedyncze kliknięcie i postępować zgodnie z instrukcjami na ekranie. Na ekranie telefonu może się wtedy pojawić okno z zapytaniem o zezwolenie na instalację aplikacji spoza Sklepu Google Play, jeśli to ustawienie jest wyłączone. Należy wtedy kliknąć przycisk \textit{Ustawienia}, a następnie zezwolić na instalację przesuwając przycisk w prawo. Można tego dokonać również poprzez wejście w odpowiednią zakładkę bezpośrednio z poziomu aplikacji ustawień (\textit{Aplikacje} -> \textit{Specjalny dostęp do aplikacji} -> \textit{Instalowanie nieznanych aplikacji} -> \textit{Files by Google lub inny eksplorator plików zainstalowany na urządzeniu}). Po pojawieniu się okna z zapytaniem o instalację należy kliknąć przycisk \textit{Zainstaluj} i poczekać na zakończenie instalacji. Aplikacja powinna następnie pojawić się na liście aplikacji na urządzeniu i być gotowa do uruchomienia.

\section{Strona główna}
Ekran główny aplikacji składa się z dwóch sekcji: u góry znajduje się pasek z opcjami, a~resztę ekranu zajmują kafelki z~dostępnymi treningami, zgodnie z~RYSUNEK. Na dole ekranu widoczna jest animacja fal.

Na środku paska znajduje się logo aplikacji. Po lewej stronie umieszczona jest ikona symbolizująca tryb zaznaczania, który jest aktywowany po kliknięciu. Tryb zmienia wygląd paska - po lewej stronie znajduje się symbol krzyżyka pozwalający opuścić tryb. Obok znajduje się informacja, że tryb zaznaczania jest aktywowany. Użytkownik może wybrać dowolne treningi z listy poprzez kliknięcie kafelków z~nazwami treningów - zaznaczone elementy oznaczone są ciemną, pogrubioną obwódką lub kilknąć przycisk \textit{Zaznacz wszystkie}, który zaznaczy automatycznie wszystkie dostępne treningi. Po wybraniu co najmniej jednego treningu pojawia się informacja o~ilości wybranych treningów. Treningi można także odznaczyć, jeśli użytkownik zmieni zdanie. Jeśli wybór treningów do eksportu jest satysfakcjonujący należy nacisnąć ikonę z~prawej strony paska. Po wybraniu miejsca eksportu w~eksploratorze plików, który się otworzy, opcjonalnie zmianie nazwy pliku i~kliknięciu przycisku \textit{Zapisz} plik w~formacie \textit{JSON} zostanie zapisany na urządzeniu. W aplikacji wyświetli się także odpowiedni komunikat mówiący o~tym, czy zapis się powiódł. (RYSUNEK). W przypadku eksportu kilku treningów, zostaną one zapisane w jednym pliku. Opcja eksportu treningu dostępna jest także z poziomu strony treningu, jednak w przeciwności do strony głównej, nie pozwala na eksport grupowy, a jedynie indywidualnego treningu. Istotny jest także fakt, że treningi eksportowane są z domyślnymi dźwiękami, a nie tymi wybranymi przez użytkownika. Eksport zachowuje jedynie strukturę treningu.

W prawej części paska znajdują się dwie ikony: koło zębate przenoszące użytkownika na stronę ustawień oraz symbol ze~strzałką w~dół umożliwiający import zapisanych treningów. Po klikniknięciu w~ikonę importu następuje przeniesienie do eksploratora plików na urządzeniu z~domyślnie otworzonym folderem \textit{Pobrane} na urządzeniu, skąd można wybrać plik z~zapisanym wcześniej treningiem/treningami.

Na stronie widoczna jest lista treningów użytkownika w postaci klikalnych kafelków z~nazwą treningu i~ozdobnym detalem symbolizującym podmuch powietrza. Naciśnięcie kafelka przenosi na stronę treningu, która zawiera szczegóły, możliwość edycji, usunięcia, eskportu czy rozpoczęcia treningu. Domyślnie, po zainstalowaniu aplikacji, dostępne są trzy predefiniowane treningi, widoczne na RYSUNEK. W przypadku, gdy użytkownik chce dodać swój trening, może to zrobić klikając ikonę plusa w~białym okręgu z~niebieskim cieniem. Zostanie on~wtedy przeniesiony na~stronę konfiguracji treningu, która opisana jest w późniejszym podrozdziale LINK. Po dodaniu indywidualnie skonfigurowanego treningu wyświetli się on na liście.

\section{Strona szczegółów treningu}
Na stronę można przejść, klikając w kafelek z wybranym treningiem na stronie głównej. Na pasku u góry znajduje się tytuł treningu oraz strzałka umożliwiająca powrót na stronę główną. Niżej znajdują się klikalne ikony pozwalające na akcje związane z treningiem. Z lewej strony umieszczona została opcja eksportu, która działa analogicznie do strony głównej, z tą różnicą, że eksportujemy tylko jeden trening. Po prawej natomiast znajduje się opcja usunięcia treningu (oznaczona symbolem kosza na śmieci), a także możliość edycji treningu (symbol ołówka). Przy akcji usuwania wyświetla się okno z prośbą o jej potwierdzenie, co zapobiega przypadkowemu usunięciu treningu. Natomiast po wybraniu opcji edycji aplikacja przenosi użytkownika do opisywanej w kolejnym podrozdziale strony. LINK

Niżej znajduje się sekcja opisu treningu. Można go edytować lub dodać na stronie konfiguracji treningu. Predefiniowane treningi zawierają opisy, jednak nie jest to element wymagany.

Następnym elementem jest \textit{Przegląd treningu}, który po naciśnięciu strzałki po prawej stronie rozwija się, pokazując wszystkie etapy treningu wraz z ich nazwami, fazami oddechowymi i długościami ich trwania, a także liczbę powtórzeń oraz przyrost. Dzięki takiemu podsumowaniu użytkownik może łatwo przejrzeć strukturę treningu.

Na stronie znajduje się również przycisk \textit{Rozpocznij trening}, który przenosi użytkownika na stronę treningu oddechowego, rozpoczynając ćwiczenie.

Dodatkowym elementem strony jest animacja delikatnie falującej łódki.

\section{Strona konfiguracji treningu}
U góry strony konfiguracji treningu, na pasku, umieszczona jest nazwa treningu (przy tworzeniu nowego treningu jest ona nadawana automatycznie), strzałka umożliwiająca powrót do strony ze szczegółami treningu oraz ikona ołówka, po kliknięciu której wyświetla się okno edycji nazwy treningu. Po wprowadzeniu wybranej nazwy należy ją zatwierdzić przyciskiem \textit{Zapisz}, można także wyjść bez zapisu po kliknięciu \textit{Anuluj}. Długość nazwy treningu ograniczona jest do piętnastu znaków, po osiągnięciu tego limitu dalsze znaki nie będą wpisywane w pole edycji. Na dole, po jego prawej stronie, znajduje się licznik znaków, tak aby użytkownik wiedział, ile znaków pozostało jeszcze do wykorzystania. 

Konfigurator składa się z trzech zakładek: \textit{Trening}, \textit{Dźwięki} oraz \textit{Inne}, tak jak na widocznym RYSUNEK.

\subsection{Zakładka Trening}
Pierwsza z zakładek - \textit{Trening} - służy do tworzenia struktury nowego treningu lub edycji istniejącego. Przyciśnięcie przycisku \textit{Dodaj etap treningu} umożliwia użytkownikowi utworzenie i dołączenie nowego etapu do treningu, co skutkuje pojawieniem się na ekranie kafelka reprezentującego uwtorzony etap pod istniejącymi etapami. Można zmienić kolejność kafelków poprzez naciśnięcie i przytrzymanie symbolu dwóch równoległych kresek umieszczonego po lewej stronie, a następnie przesunięcie palcem po ekranie w górę lub dół, do momentu gdy kafelek znajdzie się w pożądanym miejscu. Każdy kafelek zawiera nazwę etapu, która jest dodawana domyślnie przy jego tworzeniu. Może ona zostać edytowana poprzez naciśnięcie na pole z nazwą znajdującą się w zaokrąglonej ramce, a następnie wprowadzenie zmian przy pomocy klawiatury, która pokaże się na ekranie. Nazwa etapu jest ograniczona do dwudziestu pięciu znaków. Po osiągnięciu limitu, użytkownik nie będzie miał możliwości wpisania więcej znaków. Licznik znaków wyświetlany jest pod polem z nazwą. Usunięcie etapu jest możliwe poprzez kliknięcie ikony kosza na śmieci znajdującej się po prawej stronie pola z nazwą treningu, a następnie potwierdzenie wykonania akcji klikając przycisk \textit{Usuń} w okienku, które się pokaże. Usuwanie można także anulować klikając przycisk \textit{Anuluj}. Poniżej znajdują się pola dedykowane liczbie powtórzeń danego etapu oraz przyrostowi wyrażonemu w sekundach. Liczba powtórzeń definiuje, ile razy po sobie będzie odtwarzany dany etap, natomiast przyrost - o ile sekund będzie dłuższa każda faza oddechowa w kolejnych powtórzeniach. Wartości można edytować poprzez naciśnięcie ikon plusa / minusa (odpowiednio zwiększenie / zmniejszenie wartości) lub klikając w pole z aktualną liczbą, co spowoduje pokazanie klawiatury na ekranie i umożliwi użytkownikowi wpisanie wybranej wartości. W przypadku używania przycisków plusa lub minusa dla powtórzeń wartość będzie się zmieniać o jedno powtórzenie, natomiast dla przyrostu - o jedną dziesiątą sekundy w przedziale od zera do jednej sekundy lub o jedną sekundę dla wartości większych niż jedna sekunda. Domyślnymi ustawieniami są trzy powtórzenia etapu oraz brak przyrostu (przyrost ustawiony na zero sekund). Za pomocą przycisku \textit{Dodaj fazę oddechu} użytkownik może dodać fazy oddechu do etapu, które wyświetlane są w postaci listy kafelków. Każda faza oddechu zawiera parametry takie jak czas trwania oraz typ. Czas trwania wyrażany jest w sekundach i może być zmodyfikowany poprzez naciśnięcie symbolu plus / minus (odpowiednio zwiększenie / zmniejszenie) lub kliknięcie w pole z aktualną wartością i wpisanie żadanej wartości z klawiatury, która się pokaże na ekranie. Wartości zmieniają się o jedną dziesiątą sekundy w zakresie od zera do jednej sekundy, natomiast później o pół sekundy. W przypadku wpisywania z klawiatury, jeśli podana wartość ma większą dokładność niż obsługiwana, wartość jest automatycznie zaokrąglana. Przykładowo, jeśli użytkownik wpisze wartość cztery sekundy i osiem dziesiątych, zostanie ona zaokrąglona do wartości 5 sekund. Typ można ustawić natomiast poprzez kliknięcie w pole pod napisem \textit{Typ} - pojawia się wówczas rozwijana lista, z której użytkownik może wybrać żądany typ po kliknięciu w wybraną opcję. Fazę można usunąć poprzez kliknięcie ikony kosza na śmieci umiejszonego po prawej stronie kafelka, a następnie potwierdzenia akcji za pomocą przycisku \textit{Usuń}. Można także zmienić kolejność faz poprzez naciśnięcie i przytrzymanie symbolu dwóch równoległych linii umieszczonego po lewej stronie kafelka, a następnie przesunięcie palcem w górę lub dół ekranu, tak aby kafelek znalazł się w pożądanym miejscu, analogicznie jak w przypadku etapów. Dostępna jest także opcja zwinięcia kafelka za pomocą strzałki w górę umieszczonej z jego prawej strony - nie wyświetlana jest wtedy lista faz oddechu dla danego treningu. Kafelek można z powrotem rozwinąć klikając strzałkę w dół znajdującą się w tym samym miejscu, co uprzednio strzałka w górę.

\subsection{Zakładka Dźwięki}
Zakładka \textit{Dźwięki} zawiera zbiór ustawień związanych z oprawą dźwiękową treningu. Składa się z trzech sekcji: \textit{Dźwięki treningu}, \textit{Muzyka treningu} oraz \textit{Dźwięki binauralne}. Sekcje \textit{Dźwięki treningu} oraz \textit{Muzyka treningu} podzielone są na podsekcje, które są oznaczone niebieską, zaokrągloną ramką. Każda z tych podsekcji odpowiada za inny element oprawy dźwiękowej treningu. Przy tworzeniu nowego treningu domyślnie ustawione są ściśle określone dźwięki i muzyka wybrane przez twórców aplikacji przy jej tworzeniu, natomiast można je zmienić lub wyłączyć postępując zgodnie z instrukcjami znajdującymi się w dalszym opisie. Użytkownik może wybierać spośród dźwięków dostępnych w aplikacji lub własnych, które samodzielnie doda. W niektórych podsekcjach dostępny jest także lektor.

Istotnym faktem, który jest kluczowy dla zrozumienia sposobu działania aplikacji, jest istnienie trzech zestawów dźwięków. Dźwięki zostały podzielone na długie (muzyka) - dla sekcji \textit{Muzyka treningu} oraz krótkie - dla sekcji \textit{Dźwięki treningu}, za wyjątkiem podsekcji \textit{Odliczanie} tej sekcji, która posiada trzeci, osobny zestaw. Dodanie własnego pliku nagrania dźwiękowego do jednego z tych zestawów spowoduje, że będzie on wyświetlany na liście w każdej podsekcji, która korzysta z danego zestawu dźwięków. Ważne, aby pamiętać, że w przypadku dźwięków krótkich najlepiej dodawać maksymalnie kilkusekundowe nagranie dźwiękowe, w celu uzyskania jak najlepszych wrażeń z korzystania z aplikacji - dłuższe dźwięki będą bowiem ucinane. 

W każdej podsekcji, nazwa obecnie ustawionego nagrania dźwiękowego (lub informacja o jego braku czy wyborze lektora) wyświetlana jest po lewej stronie ikony nuty. Po kliknięcie w napis lub ikonę nuty otwiera się okno wyboru dźwięku lub muzyki zawierające listę dostępnych opcji. Szarym cieniem zaznaczony jest obecny wybór, tak jak na RYSUNEK. Jeśli użytkownik chce odsłuchać, jak brzmi dane nagranie dźwiękowe może to zrobić po kliknięciu w zieloną ikonę odtwarzania. Przycisk \textit{Anuluj} przerwa akcję wyboru dźwięku / muzyki, natomiast przycik \textit{Dodaj własny dźwięk} otwiera okno z plikami na telefonie użytkownika (domyślnie w folderze \textit{Pobrane}, jednak można przejść także do pozostałych folderów) i pozwala na wybór pliku nagrania dźwiękowego w formacie \textit{.mp3}. Dodany przez użytkownika dźwięk lub muzyka pojawi się następnie pod listą domyślnych dźwięków, pod napisem \textit{Dźwięki użytkownika} - RYSUNEK. W celu usunięcia dodanego przez użytkownika pliku należy nacisnąć czerwoną ikonę kosza na śmieci, a następnie potwierdzić tą akcję przyciskiem \textit{Usuń} w oknie, które się pojawi - RYSUNEK. Skutkiem będzie usunięcie dźwięku z listy dźwięków użytkownika. Aby wyłączyć dźwięk lub muzykę w danej podsekcji należy wybrać opcję \textit{Brak} z listy dostępnych dźwięków lub muzyki. W przypadku podsekcji \textit{Następna faza oddechu} lub \textit{Muzyka treningu}, które są bardziej rozbudowane, należy wybrać również opcję \textit{Brak}, natomiast znajduje się ona na liście, którą można otworzyć poprzez kliknięcie niebieskiej strzałki w dół w miejscu, gdzie w pozostałych podsekcjach znajduje się ikona nuty. 

Sekcja \textit{Dźwięki treningu} korzysta z dwóch zestawów dźwięków krótkich i składa się z trzech podsekcji. Pierwsza z nich to \textit{Odliczanie}, zawierająca dźwięk odliczania czasu trwania faz oddechowych, przygotowania oraz zakończenia. Posiada ona własny zestaw dźwięków. Składa się on z dwóch dźwięków oraz lektora. RYSUNEK Po dodaniu własnego dźwięku przez użytkownika do tego zestawu nie pojawi się on na listach w pozostałych podsekcjach, ponieważ obie korzystają z osobnego zestawu dwunastu dźwięków. Kolejną, bardziej rozbudowaną podsekcją, jest \textit{Następna faza oddechu}. Zawiera trzy opcje do wyboru. Są one osiąglne z listy rozwijanej po kliknięciu niebieskiej strzałki. Opisywana podsekcja konfiguratora dźwięków definiuje dźwięk odtwarzany po zmianie fazy na inną. Można wyłączyć ten dźwięk klikając opcję \textit{Brak} lub wybrać globalny dźwięk na całą fazę (wspomniany wcześniej zestaw dwunastu krótkich dźwięków wzbogacony jest o lektora) lub osobne dźwięki dla każdej z czterech faz oddechu. Obie opisywane opcje przedstawione są na RYSUNEK. Ostatnią podsekcją jest \textit{Zmiana etapu}, w której użytkownik może wybrać dźwięk sygnalizujący zmianę etapu treningu.

W sekcji\textit{Muzyka treningu} również znajdują się trzy podsekcje. Każda z nich korzysta z zestawu dźwięków długich (muzyki), co oznacza, że po dodaniu przez użytkownika własnej muzyki wyświetli się na listach we wszystkich podsekcjach. Na liście oprócz nazwy nagrania wyświetlany jest także jego czas trwania. Pierwsza z podsekcji opisywanej sekcji - \textit{Muzyka w tle} - jest bardziej rozbudowana, analogicznie do podsekcji \textit{Następna faza oddechu}. Po rozwinięciu listy za pomocą kliknięcia w niebieską strzałkę pozwala ustawić muzykę w tle treningu na różnym poziomie szczegółowości lub ją wyłączyć (opcja \textit{Brak}). Dostępne poziomy szczegółwości (zaczynając od najmniej szczegółowego) to: muzyka na cały trening, muzyka zdefiniowna dla kolejnych etapów treningu lub muzyka dla konkretnych faz oddechowych. Można wybrać więcej niż jeden plik dźwiękowy poprzez stworzenie kolejki z muzyką (ang. playlista). W tym celu należy kliknąć przycisk \textit{Dodaj muzykę}, wybrać nagranie z listy, a następnie powtórzyć te kroki żądaną ilość razy. W celu usunięcia muzyki z kolejki należy kliknąć ikonę kosza na śmieci. Symbol dwóch równoległych kresek służy natomiast, podobnie jak w przypadku kafelków z etapami czy fazami, do zmieniania kolejności. Przytrzymując symbol, a następnie przesuwając palcem w górę lub dół ekranu użytkownik może ułożyć nagrania w dowolnej kolejności. Jeśli użytkownik nie doda żadnego nagrania do kolejki otrzyma o tym stosowny komunikat w miejscu, gdzie wyświetlany jest aktualny stan kolejki RYSUNEK. W trakcie treningu muzyka będzie odtwarzana według kolejności określonej w kolejce. Jeśli kolejka zawiera tylko jeden plik z muzyką, będzie on odtwarzany w zapętleniu. Podsekcje \textit{Przygotowanie} oraz \textit{Zakończenie} definiują muzykę (lub jej brak) grającą odpowiednio podczas kroku przygotowania do treningu lub sygnalizują jego zakończenie.

Sekcja \textit{Dźwięki binauralne} domyślnie jest wyłączona. Można ją aktywować za pomocą kliknięcia w przełącznik znajdujący się po prawej stronie napisu \textit{Włącz dźwięki binauralne}. Istotny jest fakt, że w przypadku włączenia tej opcji automatycznie nastąpi wyłączenie opcji muzyki w tle - zostanie ona zablokowana w interfejsie i oznaczona jaśniejszym kolorem, jak na RYSUNEK. Po aktywacji, sekcja się rozwija i pokazują się dwa suwaki, za pomocą których użytkownik może ustawić częstotliwości dudnienia synchronicznego (czyli dźwięku binauralnego) dla prawego oraz lewego ucha. Sumaryczna częstotliwość uderzenia wyświetlana jest na dole sekcji. Dudnienie synchroniczne odtwarzane jest w tle treningu, zamiast muzyki w tle.

\subsection{Zakładka Inne}
W zakładce \textit{Inne} znajdują się pozostałe ustawienia - możliwe jest dodanie nowego lub modyfikacja istniejącego opisu treningu oraz ustawianie długości trwania przygotowania. Należy to zrobić używając przycisków plusa / minusa (odpowiednio zwiększenie / zmniejszenie wartości) lub klikając w okno z aktualnie ustawioną liczbą, a następnie wpisując wybraną liczbę z klawiatury, która się pokaże.

Wszystkie zmiany wprowadzone do konfiguratora zostaną zapisane automatycznie po jego opuszczeniu (czyli kliknięciu strzałki w lewym górnym rogu paska strony). W przypadku próby zapisania treningu z pustymi etapami użytkownik otrzyma komunikat widoczny na RYSUNEK.

\section{Strona treningu oddechowego}
Podczas wczytywania strony treningu oddechowego może się pojawić komunikat informujący o ładowaniu dźwięków. U góry strony widoczny jest tytuł odtworzonego treningu, a także strzałka umożliwiająca powrót do strony szczegółów treningu oraz po prawej stronie przycisk pauzy, jeśli użytkownik chce wstrzymać trening. W celu wznowienia treningu należy nacisnąć ikonę odtwarzania, która pojawi się w miejsce ikony pauzy lub napis \textit{Wznów} w środku opisanej poniżej animacji.

Podczas treningu wyświetlana jest nazwa aktualnego etapu treningu, a poniżej niej - informacja na którym etapie z ilu łącznie jest użytkownik. Na ekranie znajduje się także karuzela z kafelkami, które przesuwają się wraz z przbiegiem treningu. Kafelki zawierają nazwę kroku (fazy oddechowej, rozpoczęcia lub zakończenia treningu) oraz czas jego trwania. Centralny, największy kafelek symbolizuje obecnie trwający krok, kafelek na lewo od niego - poprzedni krok, a kafelek na prawo od centralnego - kolejny krok. Dzięki temu użytkownik może śledzić przebieg treningu oraz przygotować się na nadchodzący krok. Poniżej karuzeli znajduje się licznik wszystkich kroków, co pozwala mniej więcej zorientować się, jak daleko jest od początku treningu.

Głównym elementem strony jest animacja koła, które zmienia się zgodnie z przebiegiem treningu. Podczas wdechu koło powiększa się, a podczas wydechu - pomniejsza. W czasie trwania fazy regneracji lub wstrzymania koło jest natomiast statyczne. Obrazuje to użytkownikowi, jaką fazę oddechową powinien obecnie wykonywać. Na środku animacji znajduje się także licznik czasu przeznaczonego na dany krok.

Jeśli użytkownik kliknie strzałkę powrotu do strony szczegółów treningu, zostanie zapytany o powterdzenie swojej akcji, w celu zapobiegnięcia przypadkowemu opszczeniu treningu.

\section{Strona ustawień}
Strona ustawień zawiera dwie sekcje: wybór języka aplikacji oraz notatkę o ReSpire informującą, jaki jest cel aplikacji. W celu zmiany języka aplikacji należy nacisnąć strzałkę obok informacji o obecnie ustawionym języku, a następnie dokonać wyboru poprzez kliknięcie jednej z dwóch opcji - języka angielskiego lub języka polskiego. RYSUNEK wersja polska aplikacji, a następnie wersja angielska po zmianie języka na przykładzie strony ustawień.
