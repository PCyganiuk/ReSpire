\chapter{Podsumowanie \Karol}

\section{Co się udało} \Ola{Zrenamowałabym}
W trakcie trwania naszego inżynierskiego projektu dyplomowego udało nam sie zrealizować w pełni funkcjonalną aplikację umożliwiającą kompleksową konfigurację skomplikowanych treningów oddechowych. Aplikacja umożliwia importowanie gotowych treningów lub grup treningów, tworzenie treningów, elastyczną konfigurację ich struktury oraz wielowarstwową edycję audio dla każdego elementu treningu. Następnie nasze rozwiązanie umożliwia odtworzenie treningu z wykorzystaniem wszystkich skofngiurowanych wskazówek audiowizualnych lub eksport jednego lub wybranej przez użytkownika grupy treningów do pliku JSON.

\section{Czego się nie udało}
Wyspać się

\section{Wnioski na przyszłość}

\section{Porównanie opracowanego rozwiązania z istniejącymi rozwiązaniami}

\section{Możliwości rozwoju}
Aplikacja jest obecnie w pełni funkcjonalna i stablina, natomiast posiada perspektywy dalszego rozwoju. Jedną z nich jest integracja z modelem oceniającym w czasie rzeczywistym na podstawie nagrań audio dokładności wykonywanych treningów i wyświetlanie ich w formie przyjaznych statystyk. System możnaby także wzbogacić o mechanizm powiadomień, aby użytkownik pamiętał o regularnych treningach oddechowych, a także wprowadzić system gratyfikacji motywujący do używania ReSpire. Kolejnym potencjalnym obszarem rozwoju jest wprowadzenie innych wersji językowych - obecnie aplikacja obsługuje jedynie polski oraz angielski. Pomimo tego, że aplikacja posiada silną i spójną identyfikację wizualną, warto rozważyć także dodanie motywu ciemnego lub innych motywów kolorystycznych. Etapem pozwalającym udostępnić aplikację szerszemu gronu użytkowników byłoby wprowadzenie jej do sklepu \textit{GooglePlay} i/lub \textit{AppStore}, co z kolei stanowiłoby podstawę do dodania funkcjonalności społeczności do ReSpire - na przykład udostępniania treningów między użytkownikami bezpośrednio przez platformę.
