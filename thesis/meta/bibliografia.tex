                      % styl bibliografii
\begin{thebibliography}{3}                      % początek środowiska
\addcontentsline{toc}{chapter}{Wykaz literatury}    % dodaje bibliografię do spisu treści
\small              % spisy i bibliografie składamy mniejszym stopniem pisma

% następna pozycja
\bibitem{Brucemax}
Freediving Apnea Trainer App, https://freedivingapp.pro (data dostępu 23.05.2025 r.).

% następna pozycja
\bibitem{Breathwrk}
Breathwrk: Breathing Exercises App, https://www.breathwrk.com (data dostępu 23.05.2025 r.).

% następna pozycja
\bibitem{Squarecrowd}
STAmina Apnea Trainer App, https://getstamina.app (data dostępu 23.05.2025 r.).

% następna pozycja
\bibitem{Havabee}
Breathe App, https://havabee.com/\#portfolio (data dostępu 13.06.2025 r.).

% następna pozycja
\bibitem{Fired} 
R. Fired: \emph{Biofeedback and Self-Regulation, Integrating Music in Breathing Training
and Relaxation: II. Applications 
}, New York, 1990

% następna pozycja
\bibitem{Chittaro} 
L. Chittaro, R. Sioni: \emph{Computers in Human Behavior, Evaluating mobile apps for breathing training: The effectiveness of visualization 
}, Udine, 2014

% następna pozycja
\bibitem{Fallon} 
B. Fallon: \emph{A (blue) nt: Beyond the Symbology of the Colour Blue}, Literature & Aesthetics 24.2, 2014
  
\end{thebibliography}                           % koniec środowiska